\documentclass[a4paper]{article}
% Kodowanie latain 2
%\usepackage[latin2]{inputenc}
\usepackage[T1]{fontenc}
% Można też użyć UTF-8
\usepackage[utf8]{inputenc}

% Język
\usepackage[polish]{babel}
% \usepackage[english]{babel}

% Rózne przydatne paczki:
% - znaczki matematyczne
\usepackage{amsmath, amsfonts}
% - wcięcie na początku pierwszego akapitu
\usepackage{indentfirst}
% - komenda \url 
\usepackage{hyperref}
% - dołączanie obrazków
\usepackage{graphics}
% - szersza strona
\usepackage[nofoot,hdivide={2cm,*,2cm},vdivide={2cm,*,2cm}]{geometry}
\frenchspacing
% - brak numerów stron
\pagestyle{empty}

% dane autora
\author{Maurycy Borkowski, 980640000}
\title{Przykładowy plik w systemie \LaTeX}
\date{\today}

% początek dokumentu
\begin{document}
\maketitle
Zadanie 2

\section{Polecenie id}
Polecenie id służy do wyświetlenia numerów identyfikacyjnych i nazw kolejno: obecnego użytkownika i grup do których on należy.


\section{Czas uniksowy}
Polega na mierzeniu ilości sekund (bez sekund przestępnych), które upłynęły od 1.01.1970.
Problem roku 2038 polega na tym, że liczba sekund, którę upłynęły jest przechowywana jako liczba 32 bitowa. 32 bity przestaną wystarczać to zapisywania ile minęło czasu w dniu 19.01.2038 o godzinie 03:14:07 UTC.

\section{Podwójna kompilacja}
W zadaniu 1 gdy dwa razy kompilowałem plik sprawozdanie.tex wtedy w akapicie Wprowadzenie utworzyły się łącza do literatury zamiast znaków zapytania.

\section{Rodzdział pusty}
\dots

\section{Zwierzątko}
\begin{verbatim}
           FISHKISSFISHKIS               
       SFISHKISSFISHKISSFISH            F
    ISHK   ISSFISHKISSFISHKISS         FI
  SHKISS   FISHKISSFISHKISSFISS       FIS
HKISSFISHKISSFISHKISSFISHKISSFISH    KISS
  FISHKISSFISHKISSFISHKISSFISHKISS  FISHK
      SSFISHKISSFISHKISSFISHKISSFISHKISSF
  ISHKISSFISHKISSFISHKISSFISHKISSF  ISHKI
SSFISHKISSFISHKISSFISHKISSFISHKIS    SFIS
  HKISSFISHKISSFISHKISSFISHKISS       FIS
    HKISSFISHKISSFISHKISSFISHK         IS
       SFISHKISSFISHKISSFISH            K
         ISSFISHKISSFISHK               
\end{verbatim}

\end{document}
