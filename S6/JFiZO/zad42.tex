\documentclass{article}
\usepackage[utf8x]{inputenc}
\usepackage{polski}
\usepackage{pythonhighlight}

\usepackage{amssymb, amsmath, amsfonts, amsthm, cite, mathtools, enumerate, rotating, hyperref,soul, graphicx}
\newcommand \eq[1]{\begin{equation} \begin{split}  #1 \end{split} \end{equation}}

\makeatletter
\newcommand\tab[1][1cm]{\hspace*{#1}}
\def\@seccntformat#1{%
  \expandafter\ifx\csname c@#1\endcsname\c@section\else
  \csname the#1\endcsname\quad
  \fi}
\makeatother

\newtheorem{lemma}{Lemat}
\newtheorem{theorem}{Twierdzenie}

\title{AiSD L6}
\date{19.05.2021}
\author{Maurycy Borkowski}
\begin{document}
% \maketitle
\section{zadanie 42.}
Pokażemy, że język $L = \{w \in \{0, 1\}^* : |w|_0 \leq |w|_1 \leq 2|w|_0\}$ jest bezkontekstowy, tworząc niedeterministyczny automat ze stosem $\mathcal{A}$ rozstrzygający $L$.\\\\
Pomysł:\\
Chcemy stworzyć automat z licznikiem, będzie on liczył przewagę zer/jedynek. Na jedną jedynkę przypada co najmniej jedno i co najwyżej dwa zera i to chcemy zawrzeć w przejściach. Stan $q_2$ traktujemy jako możliwość usunięcia "$-$" drugi raz ($\epsilon$-przejściem), co odpowiada przypisania do $1$ drugiego $0$.\\\\
Rozważmy NPDA $\mathcal{A}$: $\langle \{0,1\}, \{q_1, q_2\}, \{+, -, Z\}, Z, \delta \rangle$, akceptujemy słowo gdy po jego przeczytaniu mamy pusty stos w $q_1$ (występuje tylko $Z$).\\
Zdefinujmy relację $\delta$:
\begin{align}
\delta(q_1,0,Z, q_1, Z+)\\
\delta(q_1,0,Z, q_1, Z++)\\
\delta(q_1,0,+, q_1, ++)\\
\delta(q_1,0,+, q_1, +++)\\
% \delta(q_1,0,-, q_1, +)\\
% \delta(q_1,0,-, q_1, -++)\\
\delta(q_1,0,-, q_1, \epsilon)\\
\delta(q_1,0,-, q_2, \epsilon)\\
\delta(q_1,1,Z, q_1, Z-)\\
\delta(q_1,1,-, q_1, --)\\
\delta(q_1,1,+, q_1, \epsilon)\\ %???? \delta(q_1,1,+, q_1, \epsilon)\\
\delta(q_2,0,Z, q_2, Z+)\\
\delta(q_2,0,Z, q_2, Z++)\\
\delta(q_2,0,+, q_2, ++)\\
\delta(q_2,0,+, q_2, +++)\\
\delta(q_2,\epsilon,-, q_1, \epsilon)\\
% \delta(q_2,0,-, q_1, +++)\\
\delta(q_2,1,Z, q_2, Z-)\\
% \delta(q_2,1,-, q_1, --)\\
\delta(q_2,1,+, q_2, +-) %???? \delta(q_1,1,+, q_1, \epsilon)\\
\end{align}
\begin{theorem}
Automat $\mathcal{A}$ rozstrzyga język L.
\end{theorem}
\begin{proof}
Weźmy dowolne słowo $w \in L$ oraz dwa dowolne słowa $w', w'' \notin L$, \\tż. $|w'|_1 < |w'|_0$ oraz $2|w''|_0 < |w''|_1$.\\
Pokażemy, że dla $w$ istnieje przejście automatu $\mathcal{A}$ kończące ze stosem pustym, a dla $w',w''$ nie.\\
\clearpage
\subsection*{w}
Skoro $|w|_0 \leq |w|_1 \leq 2|w|_0$ to dla każdego zera w słowie jesteśmy w stanie określić czy przypada ono na 1 czy na 2 jedynki. Czytając słowo jedyny niedeterminizm występuje w przypadku czytania zera, wtedy dla zer określonych dla samotnych jedynek używamy przejść 1,3,5,10,12 natomiast dla zer określonych dla par jedynek używamy przejść 2,4,11,13,6+14 (przejście 6 i gdy będzie "$-$" na górze stosu przejście 14). Jedyną wątpliwością może być, czy zawsze będziemy w stanie wrócić z $q_2$ gdy chcieliśmy usunąć dwa "$-$", możemy rozważać wszystkie zera z samotnymi jedynkami na początku a zera dla par jedynek na końcu, wtedy mamy pewność, że w pewnym momencie w $q_2$ na stosie będzie "$-$", odpowiada to tym jedynkom, do których przypisane jest \textit{zero dla par} (zera dla samotnych zostały skasowane pierwszymi jedynkami w $q_1$) i wtedy używamy $\epsilon$-przejścia 14.
\subsection*{w'}
Każde wystąpienie jedynki w słowie dodaje na stos "$-$", natomiast każde wystąpienie zera z "$-$" na górze stosu usuwa co najmniej jeden "$-$" (gdy $Z$ lub "$+$" na górze stosu to go powiększamy kolejnym "$+$"/"$++$"), skoro zachodzi $|w'|_1 < |w'|_0$ to nie ma możliwości usunięcia wszystkich "$-$" i skończenia z pustym stosem, zostaną jakieś "$+$".
\subsection*{w''}
Każda jedynka dodaje "$-$", ale jedno wystapienie zera może usunąć co najwyżej dwa "$-$", raz przejściem do $q_2$ i raz $\epsilon$-przejściem. Skoro $2|w''|_0 < |w''|_1$ to pewne jedynki nie zostaną \textit{usunięte} zerami i skończymy ze niepustym stosem zawierającym "$-$".
\end{proof}

\clearpage

\end{document}