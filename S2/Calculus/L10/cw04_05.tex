\documentclass{article}
\usepackage[utf8]{inputenc}
\usepackage{polski}


\usepackage{amssymb, amsmath, amsfonts, amsthm, cite, mathtools, enumerate, rotating, hyperref}
\newcommand \eq[1]{\begin{equation} \begin{split}  #1 \end{split} \end{equation}}


\makeatletter
\newcommand\tab[1][1cm]{\hspace*{#1}}
\def\@seccntformat#1{%
  \expandafter\ifx\csname c@#1\endcsname\c@section\else
  \csname the#1\endcsname\quad
  \fi}
\makeatother

\title{}
\date{04.05.2020}
\author{Maurycy Borkowski}
\begin{document}
\maketitle

\section{SUMA: 8 punktów}
\section{L10z8 (5 punktów)}

$$
I(r) = \int_0^\pi \ln{(1 - 2r \cos x + r^2)}dx = \int_0^\pi f(r,x)dx
$$
$$
I^\prime(r) = \frac{d}{dr} \left( \int_0^\pi f(r,x)dx \right) = \int_0^\pi \frac{\partial f}{\partial r} (r,x) dx
$$
Obliczmy pochodną cząstkową:
$$
\frac{\partial f}{\partial r} (r,x) = \frac{d}{dr} (\ln{(1 - 2r \cos x + r^2)}) = \frac{2(r-\cos x)}{r^2 -2r \cos x + 1}
$$
Z tego:
$$
I^\prime(r) = \int_0^\pi \frac{2(r-\cos x)}{r^2 -2r \cos x + 1} dx
$$
Łatwo zauważyć ($|r| < 1$), że funkcja pod całkowa po prawej jest zawsze dodatnia, dalek:
$$
\int_0^\pi \frac{2(r-\cos x)}{r^2 -2r \cos x + 1} dx < \int_0^\pi \frac{2(r-\cos x)}{r^2 -2r \cos x + {\cos x}^2} dx = \int_0^\pi \frac{2}{r-\cos x} dx = 0
$$
Ostatnia równość wynika z symetrii $\cos x$ na przedziale $[0,\pi]$\\
Skoro $I(0) = 0$ a pochodna jest zerowa to funkcja jest stała zatem:
$I(r) = 0$ dla $|r| < 1$.
\section{L10z4 (3 punkty)}
$$
\sum_{n=1}^{\infty} \frac{(\ln n)^{2000}}{n^{1,00001}}
$$
Funkcja sumowana jest dodatnia i malejąca, zatem zbieżność szeregu jest równoważna zbieżności:
$$
\int_a^\infty \frac{(\ln x)^{2000}}{x^{1,00001}} dx
$$
Punkt osobliwy: $\infty$
$$
\int_a^b \frac{(\ln x)^{2000}}{x^{1,00001}} dx = 
$$
Korzystamy z:
$$
\int \frac{(\ln x)^n\,dx}{x^m} = -\frac{(\ln x)^n}{(m-1)x^{m-1}} + \frac{n}{m-1}\int\frac{(\ln x)^{n-1} dx}{x^m}
$$
Zbieżność całki jest równoważna zbieżności:
$$
\int_a^\infty \frac{\ln x}{x^{1,00001}} dx = \left [-\frac{\ln x}{0,00001x^{0,00001}}-\frac{1}{(0,00001)^2 x^{0,00001}} \right]_a^\infty \rightarrow 0
$$
Zatem szereg $
\sum_{n=1}^{\infty} \frac{(\ln n)^{2000}}{n^{1,00001}}
$ jest zbieżny.
\end{document}

