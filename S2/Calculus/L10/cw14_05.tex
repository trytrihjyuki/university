\documentclass{article}
\usepackage[utf8]{inputenc}
\usepackage{polski}
\usepackage{graphicx}
\usepackage{tikz}
\usepackage{calrsfs}
\usetikzlibrary{arrows}
\graphicspath{ {.} }
\DeclareMathAlphabet{\pazocal}{OMS}{zplm}{m}{n}
\usepackage{amssymb, amsmath, amsfonts, amsthm, cite, mathtools, enumerate, rotating, hyperref}
\newcommand \eq[1]{\begin{equation} \begin{split}  #1 \end{split} \end{equation}}


\makeatletter
\def\@seccntformat#1{%
  \expandafter\ifx\csname c@#1\endcsname\c@section\else
  \csname the#1\endcsname\quad
  \fi}
\makeatother

\title{Analiza (suma 10 punktów)}
\date{14.05.2020}
\author{Maurycy Borkowski}
\begin{document}
\maketitle
\section{zad. 12* (10 punktów)}
\subsection*{I}
$$
F(a,b) = \int_0^\infty e^{-ax^2}\cos {bx} dx
$$
Funkcja pod całką jest ciągła na całym przedziale całkowania:
$$
\frac{\partial F}{\partial b} (a,b) = \frac{\partial }{\partial b} \int_0^\infty e^{-ax^2}\cos {bx} dx = 
\int_0^\infty \frac{d}{d b} e^{-ax^2}\cos {bx} dx = \int_0^\infty -e^{-ax^2}\sin {bx} \cdot x dx
$$
Skorzystamy z tożsamości:
$$
-2a \int_0^\infty x e^{-ax^2}\sin {bx} dx = e^{-ax^2}\sin{bx} {|}_0^\infty - \int_0^\infty e^{-ax^2}b\cos{bx}dx
$$
$e^{-ax^2}\sin{bx} {|}_0^\infty = 0 - 0 = 0$ dalej otrzymujemy:
$$
\int_0^\infty x e^{-ax^2}\sin {bx} dx =  \frac{b}{2a}\int_0^\infty e^{-ax^2}\cos{bx}dx
$$
Wobec powyższego:
$$
-\frac{b}{2a}\int_0^\infty e^{-ax^2}\cos{bx}dx = -\frac{b}{2a}F(a,b) = \frac{\partial F}{\partial b} (a,b)
$$
Korzystamy z faktu $y^\prime = cxy \implies y = y(0)exp(cx^2/2)$ traktując $F(a,b)$ jako funkcję jednej zmiennej $b$ z parametrem $a$:
$$
F(a,b) = F(a,0)\cdot exp(-\frac{b^2}{4a}) = \frac{\sqrt \pi}{2 \sqrt a}\cdot exp(-\frac{b^2}{4a})
$$
\\
\subsection*{II}
$$
F(a,b) = \int_0^\infty xe^{-ax^2}\sin {bx} dx
$$
Zauważmy, że funkcja do policzenia w tym podpunkcie to pochodna cząstkowa (po b) z poprzedniego zadania więc wystarczy, że nałożymy pochodną na wynik:
$$
F(a,b) = \frac{\partial }{\partial b} F(a,0)\cdot exp(-\frac{b^2}{4a}) = -\frac{\sqrt \pi}{2 \sqrt a} \frac{1}{2}ab\cdot exp({-\frac{ab^2}{4}})
$$

\end{document}

