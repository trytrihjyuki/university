\documentclass{article}
\usepackage[utf8]{inputenc}
\usepackage{polski}


\usepackage{natbib}
\usepackage{graphicx}
\usepackage{amsfonts}
\usepackage{hyperref}
\usepackage{amsmath}
\usepackage{dsfont}


\begin{document}

\makeatletter
\def\@seccntformat#1{%
  \expandafter\ifx\csname c@#1\endcsname\c@section\else
  \csname the#1\endcsname\quad
  \fi}
\makeatother

\section{3/10 (5 punktów)}

Wskazówka wynika z nierówności:
$$
(b-ac)^2 \geq 0
$$
korzystając z niej mamy:
$$
\lambda f(x)^2 + {\lambda}^{-1}g(x)^2 \geq 2|{f(x)g(x)}|
$$
dalej (zależności te były na wykładzie):
$$
\int_a^b \lambda f(x)^2 + \int_a^b {\lambda}^{-1}g(x)^2 \geq \int_a^b 2|{f(x)g(x)}|
$$
podnosząc nierówność do kwadratu otrzymujemy:
$$
\left(\int_a^b \lambda f(x)^2\right)^2 + 2 \left(\int_a^b \lambda f(x)^2 \int_a^b {\lambda}^{-1}g(x)^2\right) + \left(\int_a^b {\lambda}^{-1}g(x)^2\right)^2 \geq \left(\int_a^b 2|{f(x)g(x)}|\right)^2
$$



\end{document}

% Nieużywane Kursory