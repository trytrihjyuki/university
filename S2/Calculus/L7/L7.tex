\documentclass{article}
\usepackage[utf8]{inputenc}
\usepackage{polski}


\usepackage{amssymb, amsmath, amsfonts, amsthm, cite, mathtools, enumerate, rotating, hyperref}
\newcommand \eq[1]{\begin{equation} \begin{split}  #1 \end{split} \end{equation}}


\makeatletter
\newcommand\tab[1][1cm]{\hspace*{#1}}
\def\@seccntformat#1{%
  \expandafter\ifx\csname c@#1\endcsname\c@section\else
  \csname the#1\endcsname\quad
  \fi}
\makeatother

\title{}
\date{9.04.2020}
\author{Maurycy Borkowski}
\begin{document}
\maketitle

\section{5 (3-5 punktów)}
Dane:\\
\tab tempo podnoszenia wiadra: $v_y = \frac{dy}{dt} = 0.5 [\frac{m}{s}] $,\newline
\tab tempo wylewania się wody z wiadra: $v_{H2O} = \frac{dV}{dt} = 0.5 [\frac{l}{s}] = 5 \cdot 10^{-4} [\frac{m^3}{s}] $,\newline
\tab początkowa objętość wody w wiadrze: $V_0 = 10 [l] = 1\cdot 10^{-2} [m^3]$,\newline
\tab masa wiadra: $m_w = 0.5 [kg]$.\newline\newline
Policzmy czas po jakim cała woda z wiadra się wyleje:
$$
0 = V_0 - \int_0^{t_k} v_{H2O} dt = V_0 - v_{H2O}t_k
$$
$$
t_k = \frac{V_0}{v_{H2O}}
$$
Z tego wysokość na jaką wzniesię się wiadro zanim wyleję się z niego woda:
$$
h = \int_0^{t_k} v_y dt = v_y \cdot t_k = \frac{v_yV_0}{v_{H2O}}
$$
Szukamy zależności masy układu od wysokości $m(y) = ?$:
$$
m(y) = m_w + {\rho}_{H2O} V(y) = m_w + {\rho}_{H2O}\int_0^{y} \frac{v_{H2O}}{v_y} dy = m_w + {\rho}_{H2O}\frac{v_{H2O}}{v_y}y
$$
Praca jaka zostanie wykonana:
$$
\int_0^h F_g dy = \int_0^h gm(y)dy  = g \int_0^h \left( m_w + {\rho}_{H2O}\frac{v_{H2O}}{v_y}y\right)dy =
$$
$$
= gm_wh + g{\rho}_{H2O}\frac{v_{H2O}}{2v_y}h^2 \approx 539.55 [J]
$$
\\\\
\section{7 (3-5 punktów)}
Szukamy środka masy płatu.
$$
y_{sm} = \frac{2}{ah} \int_0^h y dm = \frac{2}{ah}\int_0^{h} \frac{y^2a}{2h}dy = \frac{1}{h^2} \int_0^{h} y^2 dy = \frac{1}{3} h
$$
Szukaną pracę policzmy z zasady zachowania energii:
$$
W = \Delta E = E_k - E_0 = mgy_{sm} - mg0 = \frac{2\cdot1}{2} \cdot g \cdot \frac{2}{3} \approx 6.54 [J]
$$

\end{document}

