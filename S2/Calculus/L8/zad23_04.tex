\documentclass{article}
\usepackage[utf8]{inputenc}
\usepackage{polski}


\usepackage{amssymb, amsmath, amsfonts, amsthm, cite, mathtools, enumerate, rotating, hyperref}
\newcommand \eq[1]{\begin{equation} \begin{split}  #1 \end{split} \end{equation}}


\makeatletter
\newcommand\tab[1][1cm]{\hspace*{#1}}
\def\@seccntformat#1{%
  \expandafter\ifx\csname c@#1\endcsname\c@section\else
  \csname the#1\endcsname\quad
  \fi}
\makeatother

\title{}
\date{23.04.2020}
\author{Maurycy Borkowski}
\begin{document}
\maketitle

\section{L8z9 (4 punkty)}
\begin{proof}
Z warunku $\int_0^1 x^n f(x) dx = 0$, dla dowolnego wielomianu $Q(x)$:
$$\int_0^1 Q(x) f(x) dx = 0$$
w szczególności dla wielomianów z Twierdzenia Weierstrassa:
$$
0 \leq \int_0^1 f(x)f(x) dx = \int_0^1 f(x)f(x) dx - \int_0^1 P_n(x) f(x) dx = \\
$$
$$
= \int_0^1 f(x) \left (f(x) - P_n(x)\right)dx \leq \int_0^1 f(x) \frac{1}{n}
$$
Możemy wybierać $n$ dowolnie duże, więc wtedy wartość $\int_0^1 f(x)f(x) dx$ będzie dowolnie blisko zera.\\
Skoro całka $\int_0^1 f(x)f(x) dx = 0$ to $f(x) = 0 \quad dla \quad x \in [0,1]$, ponieważ w innym wypadku w pewnym punkcie (i jego otoczeniu, z ciągłości) funkcja przyjmowała by dodatnie wartości, więc całka nie byłaby zerowa.
\end{proof}
\section{L8z8 (5 punktów)}
\begin{proof}
Rozważmy funkcję ze wskazówki:
$$
g(x) = G(\arccos{x}) \quad x \in [-1,1]
$$
$$
G(\theta) = g(\cos \theta) \quad \theta \in [0,\pi]
$$
$\arccos x$ jest funkcją ciągłą na $x \in [-1,1]$ więc $g(x)$ ciągła\\
Niech $P_n(x)$ oznacz ciąg wielomianów przybliżający funkcję $g(x)$.
$$
|G(\theta) - P_n(\cos\theta)| = |G(\arccos{x}) - P_n(x)| = |g(x) - P_n(x)| < \varepsilon
$$
$P_n(x)$ można przekształcić do szukanej formy wielomianu korzystając z:
$$
{\cos^n x} = \begin{cases} \frac{1}{2^{n-1}} \left [ \sum_{r=0}^{2r<n} {n\choose2r} \cos {((n-2r)\theta)}  \right] + \frac{1}{2^n} {n\choose\frac{n}{2}}  \quad  2|n \\ \frac{1}{2^{n-1}} \left [ \sum_{r=0}^{2r<n} {n\choose2r} \cos {((n-2r)\theta)}  \right]\quad  2\nmid n
\end{cases}
$$
\end{proof}
\end{document}

