\documentclass{article}
\usepackage[utf8]{inputenc}
\usepackage{polski}


\usepackage{amssymb, amsmath, amsfonts, amsthm, cite, mathtools, enumerate, rotating, hyperref}
\newcommand \eq[1]{\begin{equation} \begin{split}  #1 \end{split} \end{equation}}


\makeatletter
\newcommand\tab[1][1cm]{\hspace*{#1}}
\def\@seccntformat#1{%
  \expandafter\ifx\csname c@#1\endcsname\c@section\else
  \csname the#1\endcsname\quad
  \fi}
\makeatother

\title{}
\date{20.04.2020}
\author{Maurycy Borkowski}
\begin{document}
\maketitle

\section{6 (5 punktów)}
\begin{proof}[Dowdód indukcyjny]
\textbf{\newline Podstawa indukcji:}
\newline
$$
T_0(x) = 1 \quad bo \quad
1 = \cos 0 = \cos {0\cdot\theta}
$$
$$
T_1(x) = x \quad bo \quad \cos \theta = \cos {1\cdot\theta} 
$$

\textbf{\newline Krok:}
\newline
Załóżmy, że istnieją wielomiany $T_n, T_{n-1}$ takie, że:
$$
T_n(\cos \theta) = \cos{n \theta}
$$
$$
T_{n-1}(\cos \theta) = \cos{(n-1) \theta} 
$$
Z tożsamości trygonometrycznej (suma cosinusów):
$$
\cos {(n+1) \theta} + \cos {(n-1) \theta} = 2 \cos {\frac{2n}{2}\theta} \cos{\frac{2}{2}\theta} = 2 \cos{n \theta} \cos {\theta} 
$$
Z założenia:
$$
\cos {(n+1) \theta} + T_{n-1} = 2T_n \cos \theta
$$
Mamy więc:
$$
T_{n+1} = \cos{(n+1)\theta} = 2T_n \cos \theta - T_{n-1}
$$
Na mocy zasady indukcji, dla każdego $n$ istnieje $T_n$ taki, że: $\cos{n \theta} = T_n(\cos \theta)$
\end{proof}
$$
T_2(x) = 2T_1(x) x - T_0(x) = 2x^2 - 1
$$
$$
T_3(x) = 2T_2(x) x - T_1(x) = 4x^3 - 3x
$$
\section{7 (5 punktów)}
\begin{proof}[Dowdód indukcyjny]
\textbf{\newline Podstawa indukcji:}
\newline
$$
U_0(x) = 1 \quad bo \quad
1 = \frac{\sin (0 + 1)\theta}{\sin{\theta}}
$$
$$
U_1(x) = 2x \quad bo \quad 2\cos\theta = \frac{\sin (1 + 1)\theta}{\sin{\theta}}
$$

\textbf{\newline Krok:}
\newline
Załóżmy, że istnieją wielomiany $U_n, U_{n-1}$ takie, że:
$$
U_n(\cos n\theta) = \frac{\sin (n + 1)\theta}{\sin{\theta}}
$$
$$
U_{n-1}(\cos (n-1)\theta) = \frac{\sin n \theta}{\sin{\theta}}
$$
Z tożsamości trygonometrycznej (suma sinusów):
$$
\sin{(n-1)\theta} + \sin{(n+1)\theta} = 2 \sin n\theta \cos \theta
$$
Po podzieleniu przez $\sin \theta$
$$
\frac{\sin{(n-1)\theta}}{\sin \theta} + \frac{\sin{(n+1)\theta}}{\sin \theta} = 2 \frac{\sin n\theta}
{\sin \theta} \cos \theta
$$
Z założenia:
$$
U_{n-1} + \frac{\sin{(n+1)\theta}}{\sin \theta}  = 2U_n \cos \theta
$$
Mamy więc:
$$
U_{n+1} = \frac{\sin{(n+1)\theta}}{\sin \theta} = 2U_n \cos \theta - U_{n-1}
$$
Na mocy zasady indukcji, dla każdego $n$ istnieje $U_n$ taki, że: \\$U_n(\cos n\theta) = \frac{\sin (n + 1)\theta}{\sin{\theta}}$
\end{proof}
$$
U_2(x) = 2U_1(x) x - U_0(x) = 4x^2 - 1
$$
$$
U_3(x) = 2U_2(x) x - U_1(x) = 8x^3 - 4x
$$
\end{document}

