\documentclass{article}
\usepackage[utf8]{inputenc}
\usepackage{polski}


\usepackage{amssymb, amsmath, amsfonts, amsthm, cite, mathtools, enumerate, rotating, hyperref}
\newcommand \eq[1]{\begin{equation} \begin{split}  #1 \end{split} \end{equation}}


\makeatletter
\newcommand\tab[1][1cm]{\hspace*{#1}}
\def\@seccntformat#1{%
  \expandafter\ifx\csname c@#1\endcsname\c@section\else
  \csname the#1\endcsname\quad
  \fi}
\makeatother

\title{}
\date{28.05.2020}
\author{Maurycy Borkowski}
\begin{document}
\maketitle

\section{SUMA: 17 punktów}
\section{L9z12* (10 punktów)}
Policzmy wartość całki dla parametru $\alpha = 0$:
$$
\int_0^\infty \frac{dx}{2(1+x^2)} = \frac{1}{2} \cdot arctg{x} |_0^\infty = \frac{\pi}{4}
$$
Policzmy różnicę całek dla dowolnego parametru i zerowego parametru:
$$
\int_0^\infty \frac{dx}{2(1+x^2)} - \int_0^\infty \frac{dx}{(1+x^2)(1+x^\alpha)} = \int_0^\infty \frac{1}{2(1+x^2)} - \frac{1}{(1+x^2)(1+x^\alpha)} dx =
$$
Zróbmy podstawienie:
$$
=\int_0^\infty \frac{x^\alpha - 1}{2(1+x^2)(1+x^\alpha)} = \int_0^{\frac{\pi}{2}} \cos^2{x} \cdot \frac{{\tg x}^\alpha - 1}{{\tg{x}}^\alpha + 1} \leq \int_0^{\frac{\pi}{2}} \cdot \frac{{\tg x}^\alpha - 1}{{\tg{x}}^\alpha + 1}
$$
Przeanalizujmy wykres funkcji $\frac{{\tg x}^\alpha - 1}{{\tg{x}}^\alpha + 1}$ na przedziale $[0,\frac{\pi}{2}]$.\\
Zauważamy $x = \frac{\pi}{4}$ to jest jego miejsce zerowe. Co więcej na tym przedziale:
$$
\frac{{\tg x}^\alpha - 1}{{\tg{x}}^\alpha + 1} = - \frac{\tg{(\frac{\pi}{2} - x)}^\alpha - 1}{{\tg{(\frac{\pi}{2} - x)}^\alpha + 1}}
$$
(Nie udowodniałem tego tu, chyba nie o to chodziło w zadaniu, żeby się nad tym pastwić) Zatem ta całka jest zerowa. (Dodatnie zbijają ujemne)\\\\\\
Wobec tego, całka różnicy więc i różnica całek (dla $\alpha = 0$ i $\alpha$ dowolnego też jest zerowa) więc ostatecznie wartość całki nie zależy od parametru $\alpha$ ponadto:
$$
\int_0^\infty \frac{dx}{(1+x^2)(1+x^\alpha)} = \frac{\pi}{4}
$$\\\\
\section{L12z3 (4 punkty)}
\begin{proof}
Z założeń:
$$
\frac{\partial h}{\partial x} > \left| \frac{\partial h}{\partial y} \right| \geq 0
$$
Dalej z Tw. Lagrange'a, dla pewnych punktów a,b:
$$
\frac{h(\pi,e) - h(0,e)}{\pi} = \frac{\partial}{\partial x} h(a,e)
$$
$$
\frac{h(0,e) - h(0,0)}{e} = \frac{\partial}{\partial y} h(0,b)
$$
Dalej z powyższego:
$$
h(\pi,e) - h(0,0) = \pi \frac{\partial}{\partial x} h(a,e) + e \frac{\partial}{\partial y} h(0,b) = e \frac{\partial}{\partial x} h(a,e) + e \frac{\partial}{\partial y} h(0,b) + (\pi - e)\frac{\partial}{\partial x} h(a,e)
$$
Wszystkie składniki po prawej stronie równości są $> 0$, więc:
$$
h(\pi,e) - h(0,0) > 0
$$
\end{proof}
\section{L12z4 (3 punkty)}
\begin{proof}
Z Tw. Lagrange'a, dla pewnych punktów a, b:
$$
\frac{s(3,4) - s(0,4)}{3} = \frac{\partial}{\partial u} s(a,4)
$$
$$
\frac{s(0,4) - s(0,0)}{4} = \frac{\partial}{\partial v} s(3,b)
$$
Dalej:
$$
s(3,4) - s(0,4) = 3\frac{\partial}{\partial u} s(a,4)
$$
$$
s(0,4) - s(0,0) = 4\frac{\partial}{\partial v} s(3,b)
$$
Zatem dla pewnych a,b zachodzi:
$$
s(3,4) - s(0,0) = 3\frac{\partial}{\partial u} s(a,4) + 4\frac{\partial}{\partial v} s(3,b)
$$
Korzystając z założenia $s(3,4) = s(0,0)$, dla pewnych a,b mamy:
$$
s(3,4) - s(0,0) = 0 = 3\frac{\partial}{\partial u} s(a,4) + 4\frac{\partial}{\partial v} s(3,b)
$$
\end{proof}
\end{document}

