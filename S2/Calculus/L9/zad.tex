\documentclass{article}
\usepackage[utf8]{inputenc}
\usepackage{polski}


\usepackage{amssymb, amsmath, amsfonts, amsthm, cite, mathtools, enumerate, rotating, hyperref}
\newcommand \eq[1]{\begin{equation} \begin{split}  #1 \end{split} \end{equation}}


\makeatletter
\newcommand\tab[1][1cm]{\hspace*{#1}}
\def\@seccntformat#1{%
  \expandafter\ifx\csname c@#1\endcsname\c@section\else
  \csname the#1\endcsname\quad
  \fi}
\makeatother

\title{}
\date{28.05.2020}
\author{Maurycy Borkowski}
\begin{document}
\maketitle

\section{SUMA: 10 punktów}
\section{L9z12* (10 punktów)}
Policzmy wartość całki dla parametru $\alpha = 0$:
$$
\int_0^\infty \frac{dx}{2(1+x^2)} = \frac{1}{2} \cdot arctg{x} |_0^\infty = \frac{\pi}{4}
$$
Policzmy różnicę całek dla dowolnego parametru i zerowego parametru:
$$
\int_0^\infty \frac{dx}{2(1+x^2)} - \int_0^\infty \frac{dx}{(1+x^2)(1+x^\alpha)} = \int_0^\infty \frac{1}{2(1+x^2)} - \frac{1}{(1+x^2)(1+x^\alpha)} dx =
$$
Zróbmy podstawienie:
$$
=\int_0^\infty \frac{x^\alpha - 1}{2(1+x^2)(1+x^\alpha)} = \int_0^{\frac{\pi}{2}} \cos^2{x} \cdot \frac{{\tg x}^\alpha - 1}{{\tg{x}}^\alpha + 1} \leq \int_0^{\frac{\pi}{2}} \cdot \frac{{\tg x}^\alpha - 1}{{\tg{x}}^\alpha + 1}
$$
Przeanalizujmy wykres funkcji $\frac{{\tg x}^\alpha - 1}{{\tg{x}}^\alpha + 1}$ na przedziale $[0,\frac{\pi}{2}]$.\\
Zauważamy $x = \frac{\pi}{4}$ to jest jego miejsce zerowe. Co więcej na tym przedziale:
$$
\frac{{\tg x}^\alpha - 1}{{\tg{x}}^\alpha + 1} = - \frac{\tg{(\frac{\pi}{2} - x)}^\alpha - 1}{{\tg{(\frac{\pi}{2} - x)}^\alpha + 1}}
$$
Zatem ta całka jest zerowa. (Dodatnie zbijają ujemne)\\\\\\
Wobec tego, całka różnicy więc i różnica całek (dla $\alpha = 0$ i $\alpha$ dowolnego też jest zerowa) więc ostatecznie wartość całki nie zależy od parametru $\alpha$ ponadto:
$$
\int_0^\infty \frac{dx}{(1+x^2)(1+x^\alpha)} = \frac{\pi}{4}
$$
\end{document}

