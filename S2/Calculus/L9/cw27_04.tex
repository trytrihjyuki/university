\documentclass{article}
\usepackage[utf8]{inputenc}
\usepackage{polski}


\usepackage{amssymb, amsmath, amsfonts, amsthm, cite, mathtools, enumerate, rotating, hyperref}
\newcommand \eq[1]{\begin{equation} \begin{split}  #1 \end{split} \end{equation}}


\makeatletter
\newcommand\tab[1][1cm]{\hspace*{#1}}
\def\@seccntformat#1{%
  \expandafter\ifx\csname c@#1\endcsname\c@section\else
  \csname the#1\endcsname\quad
  \fi}
\makeatother

\title{}
\date{27.04.2020}
\author{Maurycy Borkowski}
\begin{document}
\maketitle

\section{SUMA: 10 punktów}
\section{L9z8 (5 punktów)}
\subsection*{a}
$\int_0^\infty \frac{\sin x}{\sqrt x} dx$ ma dwie osobliwości: $0,\infty$
\subsection*{$\infty$}
Z twierdzenia o wartości średniej:
$$
\left| \int_{b^\prime}^{b^{\prime\prime}} \frac{\sin x}{\sqrt x} dx \right | = \left | \frac{1}{\sqrt {b^\prime}} \int_{b^\prime}^\xi \sin x dx \right | = \left | \frac{\cos {b^\prime} - \cos {\xi}}{\sqrt {b^\prime}} \right | \leq \frac{2}{\sqrt {b^\prime}}
$$
dla ustalonego $\varepsilon$ bierzemy $b_0 = \frac{4}{\varepsilon^2}$ wtedy dla $b_0 < b^{\prime} < b^{\prime\prime}$ mamy:\\ $\left| \int_{b^\prime}^{b^{\prime\prime}} \frac{\sin x}{\sqrt x} dx \right | < \varepsilon$
\\Z warunku Cauchyego: całka jest zbieżna w osobliwości w $\infty$
\subsection*{$0$}
$$
\int_{0}^{c} \frac{\sin x}{\sqrt x} dx = \int_{0}^{c} \frac{\sin x}{x}{\sqrt x}dx \leq  \int_{0}^{c} {\sqrt x} dx = C
$$
Z kryterium porównawczego całka jest zbieżna (bezwzględnie) w osobliwości $0$.
\\\\
Zatem całka jest zbieżna bezwzględnie.\\\\
\subsection*{b}
$\int_0^\infty \frac{\cos x}{\sqrt x (x+1)} dx$ ma dwie osobliwości: $0,\infty$
\subsection*{$\infty$}
Z twierdzenia o wartości średniej:
$$
\left| \int_{b^\prime}^{b^{\prime\prime}} \frac{\cos x}{\sqrt x (x+1)} dx \right | = \left | \frac{1}{\sqrt {b^\prime} (1 + b^\prime)} \int_{b^\prime}^\xi \sin x dx \right | = \left | \frac{\cos {b^\prime} - \cos {\xi}}{{\sqrt {b^\prime} (1 + b^\prime)}} \right | \leq \frac{2}{\sqrt {b^\prime} (1 + b^\prime)}
$$
Analogiczne rozumowanie jak w \textbf{a}\\
\\Z warunku Cauchyego: całka jest zbieżna w osobliwości w $\infty$
\subsection*{$0$}
$$
\left| \int_{0}^{c} \frac{\cos x}{\sqrt x (x+1)} dx \right | \leq \left| \int_{0}^{c} \frac{1}{\sqrt x (x+1)} dx \right| \leq \left| \int_{0}^{c} \frac{1}{\sqrt x} dx \right| = \sqrt c
$$
Z kryterium porównawczego całka jest zbieżna (bezwzględnie) w osobliwości $0$.
\\\\
Zatem całka jest zbieżna bezwzględnie.\\\\
\section{3 przykłady rachunkowe (6 punktów)}
\subsection*{I}
$$
\int_0^\pi \frac{dx}{\sin x} \quad punkty\quad osobliwe: \pi, 0
$$
Dla $x \in [0,\pi] \quad x > \sin x$ z tego: $\frac{1}{\sin x} > \frac{1}{x}$.\\
Dalej z kryterium porównawczego
$$
\int_0^\pi \frac{1}{\sin x} dx > \int_0^\pi \frac{1}{x} dx
$$
Całka po prawej nierówności nie jest zbieżna bo $\lim_{x \to 0^+} \ln x = \infty$
\subsection*{II}
$$
\int_0^\infty \frac{\cos x}{1 + x^2}  dx\quad punkty\quad osobliwe: 0, \infty
$$
$$
\int_{c}^\infty \frac{\cos x}{1 + x^2}  dx < \int_{c}^\infty \frac{1}{x^2} = -\frac{1}{x} \mid^\infty_c \rightarrow 
\frac{1}{c}
$$
Sprawdźmy w 0:
$$
\int_{0}^{c} \frac{\cos x}{1 + x^2}  dx < \int_{0}^c \frac{1}{1 + x^2} = -\arctan x \mid^c_0 \rightarrow = -\arctan c
$$
Zatem całka jest zbieżna.\\\\
\subsection*{III}
$$
\int_0^1 \frac{\cos x}{\sqrt x} dx \quad punkty\quad osobliwe:  0
$$
$$
\int_0^1 \frac{\cos x}{\sqrt x} dx <\int_0^1 \frac{1}{\sqrt x} dx = 2 \sqrt x \mid^1_0 = 2
$$
\end{document}

