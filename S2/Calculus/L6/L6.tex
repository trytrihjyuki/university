\documentclass{article}
\usepackage[utf8]{inputenc}
\usepackage{polski}


\usepackage{amssymb, amsmath, amsfonts, amsthm, cite, mathtools, enumerate, rotating, hyperref}
\newcommand \eq[1]{\begin{equation} \begin{split}  #1 \end{split} \end{equation}}


\makeatletter
\def\@seccntformat#1{%
  \expandafter\ifx\csname c@#1\endcsname\c@section\else
  \csname the#1\endcsname\quad
  \fi}
\makeatother

\title{}
\date{7.04.2020}
\author{Maurycy Borkowski}
\begin{document}
\maketitle
  5 zadań (po jednym podpunkcie, podobno tyle wystarcza) z L6 po dwa punkty. Suma 10 punktów.\normalsize
\section{Z1}
Sprawdźmy punkty przecięć wykresów:
$$
\begin{cases} y^2 = 6x \\ x^2 = 6y \end{cases}
$$
$$
y^4 = 6^3y
$$
Szukane punkty to: $(0,0) , (6,6)$.\\
Dla $x \in [0,6]$:
$$
\sqrt{6x} > \frac{x^2}{6}
$$
Pole szukanego obszaru:
$$
\int_0^6 \sqrt{6x}dx - \int_0^6 \frac{x^2}{6}dx = \sqrt6\int_0^6 \sqrt{x}dx  - \frac{1}{6}\int_0^6 x^2dx = 
$$
$$
6^{\frac{1}{2}} \frac{2}{3} 6^{\frac{3}{2}} - \frac{1}{6}\frac{1}{3}6^3 = 24 - 12 = 12
$$
\section{Z2}
Długość krzywej opisanej parametrycznie:
\\$ 0 \leq t \leq 2\pi$
$$
L = \int_0^{2\pi} \sqrt{{x^{\prime}(t)}^2 + {y^{\prime}(t)}^2}dt = \int_0^{2\pi} \sqrt{{{({\cos^3 t})}^{\prime}}^2 + {{({\sin^3 t})}^{\prime}}^2}dt = \int_0^{2\pi} \sqrt{{(-3 \cos^2t \sin t)}^2 + {(3 \sin^2t \cos t)}^2}dt =
$$
$$
= \int_0^{2\pi} \sqrt{9 \cos^4t \sin^2 t + 9 \sin^4t \cos^2 t}dt = \int_0^{2\pi} \sqrt{9 \cos^2t \sin^2 t (\sin^2 t+ \cos^2 t)}dt = \int_0^{2\pi} 3 |\cos t \sin t| dt = 
$$
Z własności $\sin$:
$$
=\frac{3}{2} \int_0^{2\pi} |\sin 2t| dt = 3\int_0^{\pi} |\sin 2t| dt = 3\cdot2\int_0^{\pi} |\sin t|dt = 6
$$
\\
\section{Z3}
Długość krzywej opisanej we współrzędnych biegunowych:\\
$ r = \theta^2, 0 \leq \theta \leq 4\sqrt2$
$$
L = \int_0^{4\sqrt2} \sqrt{ {(\theta^2)^\prime}^2 + \theta^4} d\theta = \int_0^{4\sqrt2} \sqrt{4\theta^2 + \theta^4}d\theta = \int_0^{4\sqrt2} \theta\sqrt{4 + \theta^2}d\theta
$$
Całkujemy przez podstawianie:
$$
\int_0^{4\sqrt2} \theta\sqrt{4 + \theta^2}d\theta = \frac{1}{2} \int_4^{36} \sqrt{u}du = \frac{1}{2} \left [\frac{2}{3} 36^{\frac{3}{2}} - \frac{2}{3} 4^{\frac{3}{2}} \right] = \frac{208}{3}
$$
\section{Z4}
Pole powierzchni bryły obrotowej:\\
$f(x) = \frac{1}{3}x^3, x \in [0,\sqrt2]$
$$
S = 2\pi \int_0^{\sqrt2} \frac{1}{3}x^3 \sqrt{1 + {\left(\frac{1}{3}x^3 \right)^\prime}^2}dx = \frac{2\pi}{3} \int_0^{\sqrt2} x^3 \sqrt{1 + x^4}dx = 
$$
Całkujemy przez podstawianie:
$$
= \frac{\pi}{6} \int_1^5 \sqrt u du = \frac{\pi}{6} \left[\frac{2}{3} 5^{\frac{3}{2}} - \frac{2}{3} 1^{\frac{3}{2}}\right] = \frac{\pi}{9} \left (5 \sqrt 5  - 1\right)
$$
\section{Z5}
Objętość bryły obrotowej:\\
$f(x) = \frac{-1}{x}, x \in [-3,-2]$
$$
V = \pi \int_{-3}^{-2}  {\left (\frac{-1}{x} \right)}^2 dx = \pi \int_{-3}^{-2} x^{-2} dx = \pi \left [-x^{-1} \right]_{-3}^{-2} = \frac{\pi}{6}
$$
\end{document}

% Nieużywane Kursory