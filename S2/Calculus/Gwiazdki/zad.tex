\documentclass{article}
\usepackage[utf8]{inputenc}
\usepackage{polski}


\usepackage{amssymb, amsmath, amsfonts, amsthm, cite, mathtools, enumerate, rotating, hyperref}
\newcommand \eq[1]{\begin{equation} \begin{split}  #1 \end{split} \end{equation}}


\makeatletter
\newcommand\tab[1][1cm]{\hspace*{#1}}
\def\@seccntformat#1{%
  \expandafter\ifx\csname c@#1\endcsname\c@section\else
  \csname the#1\endcsname\quad
  \fi}
\makeatother

\title{}
\date{08.06.2020}
\author{Maurycy Borkowski}
\begin{document}
\maketitle

\section{SUMA: 41 punktów}
\section{L5z9* (10 punktów)}
$$
\xi(x) = \int_0^x{f(t)dt}
$$
Szukamy fukcji f tż:
$$
f(x) = e^{-\xi(x)}
$$
Przy oznaczeniu $F$ jako funkcji pierwotnej $f$:
$$
f(x) = e^{-F(x)}
$$
$$
F^\prime(x) = e^{-F(x)}
$$
$$
F^\prime(x)e^{F(x)} = 1
$$
Zauważamy:
$$
\left( e^{F(x)} \right)^\prime = 1
$$
Wnioskujemy $F(x) = \log x$, dalej:
$$
F(x)^\prime = \frac{1}{x} = f(x)
$$
Możemy sprawdzić:
$$
f(x) = \frac{1}{x} = e^{-\log x} = e^{-\int_0^x{\frac{1}{t}dt}} = e^{-\xi(x)}
$$\\\\\\\\\\\\\\\\\\\
\section{L5z10* (10 punktów)}
\begin{proof}
Zastosujmy tw. o wartości średniej $N$ razy:
\begin{align*}
\int_a^b f(x)dx = f(c_0) = 0\\
\int_a^b x f(x)dx = f(c_1)\int_a^b xdx = 0\\
\int_a^b x^2 f(x)dx = f(c_2)\int_a^b x^2dx = 0\\
   &\vdots\\
\int_a^b x^N f(x)dx = f(c_N)\int_a^b x^Ndx = 0\\
\end{align*}
Wystarczy zatem uzasadnić, że: $c_i \neq c_j$ dla $i \neq j$.\\
Załóżmy nie wprost, że $f$ ma tylko jedno miejsce zerowe, oznaczmy je jako $d$, wtedy $f(x) < 0$ dla $x < d$ i $f(x) > 0$ dla $x > d$:
$$
\int_a^d f(x)dx + \int_d^b f(x)dx = 0
$$
$$
\int_a^d x f(x)dx + \int_d^b x f(x)dx = 0
$$
W obu powyższych sumach BSO lewy składnik jest ujemny, prawy dodatni.\\
Widać, że dolna równość nie jest prawdziwa dla niezerowej funkcji (dla zerowej teza jest trywialna), funkcja liniowa jest rosnąca i skaluje oba czynniki (bardziej drugi) w tym przypadku suma będzie większa od zera. Można to rozumowanie przeciągnąć indukcyjnie po n przedziałach, w ten sposób udowadniamy, że są one parami różne.
\end{proof}
\section{L5z13* (10 punktów)}
$$
\int_0^a e^{-x} \left (1 + \frac{x}{1!} + \frac{x^2}{2!} + \cdots + \frac{x^{100}}{100!} \right )dx = 50
$$
Rozpoznajemy rozwinięcie $e^x$ w szereg Taylora:
$$
\int_0^a e^{-x} \left (e^x - R_{100}(x,0) \right )dx = 50
$$
$$
f(a) = a - \int_0^a \frac{R_{100}(x,0)}{e^{x}}dx
$$
Zauważmy, że dla $a = 50$ $f(a) < 50$ bo prawy czynnik jest niezerowy.\\Dalej:
$$
% R_{100}(x,0) = \int_0^x \frac{(x-t)^{100}}{100!} e^t dt
R_{100}(x,0) % = \frac{x^{101}}{101!} e^{(\theta(x))}
 = \int_0^x \frac{(x-t)^{100}}{100!} e^t dt
$$
$f(100) > 50$ ponieważ: (funkcja podcałkowa jest malejąca)

$$
\int_0^{100} \frac{R_{100}(x,0)}{e^{x}}dx < 100 \cdot \frac{R_{100}(x,1)}{e^{1}} \approx 10^{-3} \ll 50
$$
Z skoro $f(50) < 50$ i $f(100) > 50$ to z własności Darboux'a (funkcje ciągłe wszystkie) istnieje rozwiązanie równania z zadania.
\section{L12z9 (3 punkty)}
Korzystając z reguły łańcucha:
$$
\begin{cases}
\frac{\partial z}{\partial r} = \frac{\partial z}{\partial x}\frac{\partial x}{\partial r} + \frac{\partial z}{\partial y}\frac{\partial y}{\partial r}\\
\frac{\partial z}{\partial \theta} = \frac{\partial z}{\partial x}\frac{\partial x}{\partial \theta} + \frac{\partial z}{\partial y}\frac{\partial y}{\partial \theta}\\
\end{cases}
$$
$$
\begin{cases}
\frac{\partial z}{\partial y} = \left(\frac{\partial y}{\partial r}\right)^{-1} \left( \frac{\partial z}{\partial r} - \frac{\partial z}{\partial x}\frac{\partial x}{\partial r} \right)\\
\frac{\partial z}{\partial x} = \left(\frac{\partial x}{\partial \theta}\right)^{-1} \left( \frac{\partial z}{\partial \theta} - \frac{\partial z}{\partial y}\frac{\partial y}{\partial \theta} \right)\\
\end{cases}
$$
$$
\begin{cases}
\frac{\partial z}{\partial y} = \left(\sin \theta\right)^{-1} \left( \frac{\partial z}{\partial r} - \frac{\partial z}{\partial x}\cos \theta\right)\\
\frac{\partial z}{\partial x} = \left(-r\sin\theta\right)^{-1} \left( \frac{\partial z}{\partial \theta} - \frac{\partial z}{\partial y}r\cos\theta \right)\\
\end{cases}
$$
$$
\frac{\partial z}{\partial x} = \left(-r\sin\theta\right)^{-1} \left( \frac{\partial z}{\partial \theta} - \left(\sin \theta\right)^{-1} \left( \frac{\partial z}{\partial r} - \frac{\partial z}{\partial x}\cos \theta\right)r\cos\theta \right)\\
$$
Po uporządkowaniu mamy:
$$
\frac{\partial z}{\partial x} = \frac{\partial z}{\partial r}\cos \theta - \frac{\partial z}{\partial \theta}\frac{\sin \theta}{r}
$$
Analogicznie z powyższych równań otrzymujemy:
$$
\frac{\partial z}{\partial y} = \frac{\partial z}{\partial r}\sin\theta + \frac{\partial z}{\partial \theta}\frac{\cos \theta}{r}
$$\\\\\\\\\\\
\section{L12z12 (3 punkty)}
Dane: $V_x = 30 [\frac{km}{h}],V_y = 160 [\frac{km}{h}]$ $x_0 = 1 [km]$ $y_0 = 2 [km]$\\\\Rozwiązanie:\\\\
$$S(x,y) = \sqrt{x^2 + y^2}$$
Korzystając z reguły łańcucha:
$$\frac{dS}{dt} = \frac{\partial S}{\partial x} \frac{dx}{dt} + \frac{\partial S}{\partial y} \frac{dy}{dt}$$
$$\frac{dS}{dt} = 30 \cdot 2x \cdot \frac{1}{2\sqrt{x^2 + y^2}} + 160 \cdot 2y \cdot \frac{1}{2\sqrt{x^2 + y^2}}$$
$$\frac{dS}{dt} = \frac{1}{\sqrt{x^2 + y^2}} (30x + 160y)$$
W szukanym punkcie:
$$
\frac{1}{\sqrt{5}} (350) = 70 \sqrt5
$$\\\\\\\\\\\\\\\\\\\\\\\\\\\\\\\\\\\\\\\\\\
\section{L13z4* (5 punktów)}
$$
\left ( \frac{\partial y}{\partial x} \right )\left ( \frac{\partial z}{\partial y} \right )\left ( \frac{\partial x}{\partial z} \right ) = -1
$$
Powyższe równanie (prawdopodobnie) pochodzi od równania opisującej stan gazu doskonałego (Równanie Clapeyrona) $pV = RT$ w bardziej matematycznej formie $\frac{xy}{z} = const$.\\Możemy wtedy określić:
$$
x = f(y,z) = \frac{C\cdot z}{y}
$$
$$
y = g(x,z) = \frac{C\cdot z}{x}
$$
$$
z = h(x,y) = \frac{xy}{C}
$$
Wtedy:
$$
\frac{\partial y}{\partial x} = \frac{\partial g}{\partial x} = -\frac{C \cdot z}{x^2}
$$
$$
\frac{\partial z}{\partial y} = \frac{\partial h}{\partial y} = \frac{x}{C}
$$
$$
\frac{\partial x}{\partial z} = \frac{\partial f}{\partial z} = \frac{C}{y}
$$
mnożąc powyższe pochodne cząstkowe:
$$
\left ( \frac{\partial y}{\partial x} \right )\left ( \frac{\partial z}{\partial y} \right )\left ( \frac{\partial x}{\partial z} \right )  = \left ( -\frac{C \cdot z}{x^2} \right) \left( \frac{x}{C}\right) \left( \frac{C}{y}\right) = -C\frac{z}{xy}
$$
Ale zdefiniowaliśmy $C = \frac{xy}{z}$, zatem:
$$
\left ( \frac{\partial y}{\partial x} \right )\left ( \frac{\partial z}{\partial y} \right )\left ( \frac{\partial x}{\partial z} \right ) = -C\frac{z}{xy} = -C\frac{1}{C} = -1
$$
\end{document}

