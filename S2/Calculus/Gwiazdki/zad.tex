\documentclass{article}
\usepackage[utf8]{inputenc}
\usepackage{polski}


\usepackage{amssymb, amsmath, amsfonts, amsthm, cite, mathtools, enumerate, rotating, hyperref}
\newcommand \eq[1]{\begin{equation} \begin{split}  #1 \end{split} \end{equation}}


\makeatletter
\newcommand\tab[1][1cm]{\hspace*{#1}}
\def\@seccntformat#1{%
  \expandafter\ifx\csname c@#1\endcsname\c@section\else
  \csname the#1\endcsname\quad
  \fi}
\makeatother

\title{}
\date{08.06.2020}
\author{Maurycy Borkowski}
\begin{document}
\maketitle

\section{SUMA: 40 punktów}
\section{L5z9* (10 punktów)}
$$
\xi(x) = \int_0^x{f(t)dt}
$$
Szukamy fukcji f tż:
$$
f(x) = e^{-\xi(x)}
$$
Przy oznaczeniu $F$ jako funkcji pierwotnej $f$:
$$
f(x) = e^{-F(x)}
$$
$$
F^\prime(x) = e^{-F(x)}
$$
$$
F^\prime(x)e^{F(x)} = 1
$$
Zauważamy:
$$
\left( e^{F(x)} \right)^\prime = 1
$$
Wnioskujemy $F(x) = \log x$, dalej:
$$
F(x)^\prime = \frac{1}{x} = f(x)
$$
Możemy sprawdzić:
$$
f(x) = \frac{1}{x} = e^{-\log x} = e^{-\int_0^x{\frac{1}{t}dt}} = e^{-\xi(x)}
$$\\\\\\\\\\\\\\\\\\\
\section{L5z10* (10 punktów)}
\begin{proof}
$$
\int_a^b x^n f(x)dx = 0
$$
\end{proof}
\section{L5z13* (10 punktów)}
\section{L11z13* (10 punktów)}
%Trzeba e^x oszacować ze szeregu taylora
%https://math.stackexchange.com/questions/2569710/a-function-whose-partial-derivatives-exist-at-a-point-but-is-not-continuous
\end{document}

