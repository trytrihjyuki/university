\documentclass{article}
\usepackage[utf8]{inputenc}
\usepackage{polski}


\usepackage{amssymb, amsmath, amsfonts, amsthm, cite, mathtools, enumerate, rotating, hyperref}
\newcommand \eq[1]{\begin{equation} \begin{split}  #1 \end{split} \end{equation}}


\makeatletter
\newcommand\tab[1][1cm]{\hspace*{#1}}
\def\@seccntformat#1{%
  \expandafter\ifx\csname c@#1\endcsname\c@section\else
  \csname the#1\endcsname\quad
  \fi}
\makeatother

\title{Suma 10 punktów}
\date{15.06.2020}
\author{Maurycy Borkowski}
\begin{document}
\maketitle
\section{zad. 11* (10 punktów)}
Zdefiniujmy $f(x,y)$:
$$
f(x,y) =
\begin{cases}
\frac{xy}{x^2+y^2} \quad gdy \quad (x,y) \neq (0,0) \\
0 \quad w \quad pozosatałych \quad przypadkach\\
\end{cases}
$$
Funkcja nie jest ciągła w $(0,0)$ ponieważ idąc po osi $y = x$
funkcja jest stała równa:
$$
f(t,t) = \frac{t^2}{t^2 + t^2} = \frac{1}{2}
$$
Idąc natomiast po osiach x i y do zera funkcja jest zerowa $f(x,0) = f(y,0) = 0$
Funkcja f ma pochodne cząstkowe w każdym punkcie:
$$
\frac{\partial f}{\partial x} = \frac{y(x^2+y^2) -xy\cdot2x}{(x^2+y^2)^2}
$$
Mianownik jest zerowy wtw gdy $(x,y) = (0,0)$  a wtedy pochodne są zerowe bo f jest zerowa, więc wszystkie pochodne cząstkowe istnieją. Analogicznie $\frac{\partial f}{\partial y}$
\end{document}

