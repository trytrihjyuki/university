\documentclass{article}
\usepackage[utf8]{inputenc}
\usepackage{polski}


\usepackage{amssymb, amsmath, amsfonts, amsthm, cite, mathtools, enumerate, rotating, hyperref}
\newcommand \eq[1]{\begin{equation} \begin{split}  #1 \end{split} \end{equation}}


\makeatletter
\def\@seccntformat#1{%
  \expandafter\ifx\csname c@#1\endcsname\c@section\else
  \csname the#1\endcsname\quad
  \fi}
\makeatother

\title{}
\date{24.03.2020}
\author{Maurycy Borkowski}
\begin{document}
\maketitle

\section{4/11 (10 punktów)}
$Lemat.$
$$
\int_0^b f(x) = \int_0^b f(b-x)
$$
Podstawiając $u = b - x$ oraz $du = -dx$:
$$
\int_0^b f(x) = \int_b^0 f(u)du = \int_b^0 f(b-x)(-dx) = \int_0^b f(b-x)dx 
$$
\begin{proof}
$$
\int_0^\pi xf(\sin x)dx = \int_0^\pi (\pi-x)f(\sin{(\pi-x)})dx = \int_0^\pi \pi f(\sin x)dx - \int_0^\pi xf(\sin x)dx
$$
$$
\int_0^\pi xf(\sin x)dx = \frac{\pi}{2}\int_0^\pi f(\sin x)dx
$$
\end{proof}
\end{document}

% Nieużywane Kursory