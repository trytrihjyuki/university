\documentclass{article}
\usepackage[utf8]{inputenc}
\usepackage{polski}
\usepackage{graphicx}
\usepackage{tikz}
\usepackage{calrsfs}
\usetikzlibrary{arrows}
\graphicspath{ {.} }
\DeclareMathAlphabet{\pazocal}{OMS}{zplm}{m}{n}
\usepackage{amssymb, amsmath, amsfonts, amsthm, cite, mathtools, enumerate, rotating, hyperref}
\newcommand \eq[1]{\begin{equation} \begin{split}  #1 \end{split} \end{equation}}


\makeatletter
\def\@seccntformat#1{%
  \expandafter\ifx\csname c@#1\endcsname\c@section\else
  \csname the#1\endcsname\quad
  \fi}
\makeatother

\title{Algebra L9 (suma 14 punktów)}
\date{13.05.2020}
\author{Maurycy Borkowski}
\begin{document}
\maketitle
\section{zad. 4 (1 punkt)}
\subsection*{I}
$$
A=
\begin{pmatrix}
2 & 0 \\
0 & 2 \\
\end{pmatrix}
$$
$$
\mu_{\pazocal{A}}(t) = t-2
$$
\subsection*{II}
$$
B=
\begin{pmatrix}
2 & 1 \\
1 & 3 \\
\end{pmatrix}
$$
$$
\mu_{\pazocal{B}}(t) = t^3-4\cdot t^2 + 5
$$
\\Rozpisałem na kartce potęgi macierzy i zgadłem wielomiany.
\section{zad. 6 (1 punkt)}
$$
[[X, Y], Z] + [[Y, Z],X] + [[X, Z], Y] = 0
$$
$$
(XY-YX)Z - Z(XY-YX) + (YZ-ZY)X - X(YZ-ZY) + (XZ-ZX)Y - Y(XZ-ZX) =
$$
Z łączności dodawania:
$$
XYZ -YXZ - ZXY + ZYX + YZX -ZYX - XYZ + XZY + XZY - ZXY -YXZ + YZX = 0
$$\\\\
\section{zad. 7 (2 punkty)}
$$
[M_u,M_v] = 
\begin{pmatrix}
0 & -u_3 & u_2 \\
u_3 & 0 & -u_1 \\
-u_2 & u_1 & 0 \\
\end{pmatrix} \cdot
\begin{pmatrix}
0 & -v_3 & v_2 \\
v_3 & 0 & -v_1 \\
-v_2 & v_1 & 0 \\
\end{pmatrix} -
\begin{pmatrix}
0 & -v_3 & v_2 \\
v_3 & 0 & -v_1 \\
-v_2 & v_1 & 0 \\
\end{pmatrix} \cdot
\begin{pmatrix}
0 & -u_3 & u_2 \\
u_3 & 0 & -u_1 \\
-u_2 & u_1 & 0 \\
\end{pmatrix} =
$$
$$
\begin{pmatrix}
0 & u_2v_1 - u_1v_2 & u_3v_1 - u_1v_3 \\
u_1v_2 - u_2v_1 & 0 & u_3v_2 - u_2v_3 \\
u_1v_3 - u_3v_1 & u_2v_3-u_3v_2 & 0 \\
\end{pmatrix} =
M_{u\times v}
$$
Skoro dla dowolnych wektorów możemy zapisać ich iloczyn wektorowy w postaci komutatora macierzowego a z \textbf{zad. 6} wiemy, że on spełnia tożsamość Jacobiego to iloczyn wektorowy też ją spełnia.
\section{zad. 8 (2 punkty)}
\subsection*{I}
$$
m_D^B(F\circ G) = m_D^C(F)m_C^B(G)
$$
Uzasadnienie:\\
\begin{itemize}
\item Prawa: wektor zapisany w bazie B przenosimy przekształceniem G jest zapisany w bazie C, mamy wektor w bazie C następnie przenosimy go przekształceniem F jest zapisany w bazie D.
\item Lewa: wektor zapisany w bazie B przenosimy przekształceniem G jest zapisany w pewnej bazie teraz na niego nakładamy przekształcenie F i jest zapisany w bazie B.
\end{itemize}
\subsection*{II}

$
m_{B^\prime}^B(Id_U)
$
zgodnie z przyjętym (na wykładzie) oznaczeniem jest macierzą przejścia z bazy $B$ do $B^\prime$
\section{zad. 9 (2 punkty)}
\subsection*{$\Rightarrow$}
Skoro $F$ jest izomorfizmem to istnieje $F^{-1}:W\rightarrow V$:
$$
E = m_C^C(F\circ F^{-1}) = m_D^C(F)m_C^D(F^{-1})
$$
Znaleźliśmy macierz odwrotną, zatem szukana macierz jest odwracalna.
\subsection*{$\Leftarrow$}
Oznaczmy macierz odwrotną jako G.
$$
E = m_D^C(F)G
$$
Wnioskujemy że $G$ jest macierzą pewnego przekształcenia liniowego, które bierze wektory z W w bazie D i \textit{wyrzuca} wektory z V z bazy C. Dalej:
$$
E = m_D^C(F)m_C^D(H) = m_C^C(F\circ H) = m_C^C(Id)
$$
Więc: $F\circ H = Id$ z tego F jest izomorfizmem.
\section{zad. 10 (2 punkty)}
\subsection*{I}
$$
\mu_{\pazocal{A}}(t) = t^2 - 6t + 5
$$
Rozwiązuję układ równań:
$$
\begin{cases}
a^2 + bc - 6a + 5 = 0\\
d^2 + bc - 6d + 5 = 0\\
ab+bd-6b=0\\
ac+cd-6c=0\\
\end{cases}
$$
Wnioskuje $a = d = 3$, wtedy $b,c$ prawie dowolne:
$$
A =
\begin{pmatrix}
3 & 5 \\
1 & 3 \\
\end{pmatrix}
$$
\subsection*{II}
$$
\mu_{\pazocal{B}}(t) = t^2 - 4t + 4
$$
Rozwiązuję układ równań:
$$
\begin{cases}
a^2 + bc - 4a + 4 = 0\\
d^2 + bc - 4d + 4 = 0\\
ab+bd-4b=0\\
ac+cd-4c=0\\
\end{cases}
$$
Wnioskuje $a = d = 2$, wtedy $b,c$ prawie dowolne:
$$
B =
\begin{pmatrix}
2 & 4 \\
1 & 2 \\
\end{pmatrix}
$$
\section{zad. 12 (2 punkty)}
Najogólnieszą postacią elementu z $K[A]$ jest:
$$
a_0 A^m + a_1 A^{m-1} +a_2 A^{m-2} + \dots + a_{m-1}A + a_mE
$$
Jak widać, żeby stworzyć go wystarczy baza: ${A,E}$. Oczywiste, że $\{A,E\} \leq 2$

\section{zad. 13 (2 punkty)}
Skoro $V = ker \pazocal{P} \oplus Im \pazocal{P}$ to każdy wektor z $V$ zapisujemy jednoznacznie:\\
$v = u + w$ gdzie $u \in ker \pazocal{P}$ oraz $w \in Im \pazocal{P}$\\
Z liniowości $\pazocal{P}$:\\
$$
\pazocal{P}v = \pazocal{P}u + \pazocal{P}w = \pazocal{P}w
$$
A to już jest definicja rzutu (z szarego z Kostrikina).
\end{document}

