\documentclass{article}
\usepackage[utf8]{inputenc}
\usepackage{polski}


\usepackage{amssymb, amsmath, amsfonts, amsthm, cite, mathtools, enumerate, rotating, hyperref}
\newcommand \eq[1]{\begin{equation} \begin{split}  #1 \end{split} \end{equation}}


\makeatletter
\renewcommand*\env@matrix[1][*\c@MaxMatrixCols c]{%
  \hskip -\arraycolsep
  \let\@ifnextchar\new@ifnextchar
  \array{#1}}
\def\@seccntformat#1{%
  \expandafter\ifx\csname c@#1\endcsname\c@section\else
  \csname the#1\endcsname\quad
  \fi}
\makeatother

\title{Lista 3}
\date{25.03.2020}
\author{Maurycy Borkowski}
\begin{document}
\maketitle

\section{zad. 1 (1 punkt)}
a)$$ F\left(\begin{pmatrix}
            x\\
            y\\
            z\\
           \end{pmatrix}\right)
        =
            \begin{pmatrix}
            x\\
            y\\
            z\\
            \frac{x+2y+z}{-3}\\
           \end{pmatrix}
$$
b)$$P(1) = a + b + c + d + e = 0$$
$$P(0)^{\prime\prime} = 12a0^2 + 6b0 + c = c = 0$$
$$F\left(\begin{pmatrix}
            x\\
            y\\
            z\\
           \end{pmatrix}\right)
        = P(x^{\prime}) = x{x^{\prime}}^4 + y{x^{\prime}}^3 + 0{x^{\prime}}^2 + z{x^{\prime}} {-(x+y+z)}$$
$$F^{-1}(X-X^3) = \begin{pmatrix}
            0\\
            -1\\
            1\\
           \end{pmatrix}$$
\section{zad. 3 (1 punkt)}
$$P(-2) =-8a+4b-2c+d=1$$
$$P(-1) =-a+b-c+d=3$$
$$P(1) =a+b+c+d=13 $$
$$P(2) =8a+4b+2c+d=33$$
Rozwiążmy układ równań:
$$\begin{cases} -8a+4b-2c+d=1 \\ -a+b-c+d=3 \\ a+b+c+d=13 \\8a+4b+2c+d=33 \end{cases}$$
Macierz rozszerzona $(A|B)$:
$$
\begin{pmatrix}[cccc|c]
-8 & 4 & -2 & 1 & 1 \\
-1 & 1 & -1 & 1 & 3 \\
1 & 1 & 1 & 1 & 13 \\
8 & 4 & 2 & 1 & 33 \\
\end{pmatrix}
$$
W postaci schodkowej:
$$
\begin{pmatrix}[cccc|c]
1 & 1 & 1 & 1 & 13 \\
0 & 2 & 0 & 2 & 16 \\
0 & 0 & -6 & -3 & -39 \\
0 & 0 & 0 & -6 & -30 \\
\end{pmatrix}
$$
Z niej uzyskujemy rozwiązanie, szukany wielomian:
$$P(x) = x^3 + 3x^2 + 4x + 5$$
\section{zad. 4 (1 punkt)}
a)
\\
Po sprowadzeniu do postaci schodkowej:
$$
\begin{pmatrix}[cccc|c]
1 & 7 & 9 & 4 & 2 \\
0 & 4 & 5 & 1 & 0 \\
0 & 0 & 0 & 0 & 0 \\
\end{pmatrix}
$$
Odpowiada ona:
$$\begin{cases} x_1+7x_2+9x_3+4x_4=2 \\ 4x_2+5x_3+x_4 = 0\end{cases}$$
Mamy:
$$x_2 = \frac{5x_3 + x_4}{-4}$$
$$x_1 = \frac{7}{4}(5x_3+x_4) - 9x_3 -4x_4 +2 = -\frac{1}{4}x_3 - \frac{9}{4}x_4 + 2$$
Ostatecznie:
$$
X = \begin{pmatrix}
    x_1 \\
    x_2 \\
    x_3 \\
    x_4 \\
    \end{pmatrix}
    =
    \begin{pmatrix}
    -\frac{1}{4}x_3 - \frac{9}{4}x_4 + 2 \\
    \frac{5x_3 + x_4}{-4} \\
    x_3 \\
    x_4 \\
    \end{pmatrix}
    =
    \begin{pmatrix}
    2 \\
    0 \\
    0 \\
    0 \\
    \end{pmatrix}
    +
    x_3
    \begin{pmatrix}
    -1/4 \\
    -5/4 \\
    1 \\
    0 \\
    \end{pmatrix}
    +
    x_4
    \begin{pmatrix}
    -9/4 \\
    -1/4 \\
    0 \\
    1 \\
    \end{pmatrix}
$$
b)
Macierz rozszerzona:
$$
\begin{pmatrix}[cccc|c]
-9 & 10 & 3 & 7 & 7 \\
-4 & 7 & 1 & 3 & 5 \\
7 & 5 & -4 & -6 & 3 \\
\end{pmatrix}
$$
$$
\begin{pmatrix}[cccc|c]
-9 & 10 & 3 & 7 & 7 \\
7 & 5 & -4 & -6 & 3 \\
0 & 23 & -3 & -1 & 16 \\
\end{pmatrix}
$$
$$
\begin{pmatrix}[cccc|c]
-63 & 70 & 21 & 49 & 49 \\
63 & 45 & -36 & -54 & 27 \\
0 & 23 & -3 & -1 & 16 \\
\end{pmatrix}
$$
$$
\begin{pmatrix}[cccc|c]
-63 & 70 & 21 & 49 & 49 \\
0 & 115 & -15 & -5 & 76 \\
0 & 23 & -3 & -1 & 16 \\
\end{pmatrix}
$$
$$
\begin{pmatrix}[cccc|c]
-63 & 70 & 21 & 49 & 49 \\
0 & 23 & -3 & -1 & 16 \\
0 & 0 & 0 & 0 & -4 \\
\end{pmatrix}
$$
Wobec powyższego układ jest sprzeczny.

\section{zad. 7 (2 punkty)}
\begin{proof}
Z tw. Bezouta wielomian z zadania dzieli się bez reszty przez:
$(x-1), (x-2) \dots (x-14)$ wobec tego jest postaci $(x-1)(x-2)\dots(x-14)Q(x)$ albo jest wielomianem zerowym. Wiemy, że stopień wielomianu z zadania $\leq$ 13, wnioskujemy, że wielomian, o którym mowa jest wielomianem zerowym.
\\\\
Zdefiniujmy $F: \textbf{R}^{14} \rightarrow \textbf{R}_{13}[X] $ następująco:
$$
F\left(\begin{pmatrix}
            x_1\\
            x_2\\
            \vdots\\
            x_{14}\\
           \end{pmatrix}\right)
           =
           P(x)
$$
taki  że,
$$
P(1) = x_1,
P(2) = x_2 \dots P(14) = x_{14}
$$
Z poprzedniej części wiemy, że:
$$
F\left(\begin{pmatrix}
            0\\
            0\\
            \vdots\\
            0\\
           \end{pmatrix}\right)
           =
           0
$$
Zauważmy, że jeżeli $x_i \neq 0$ dla pewnego $i$ to na pewno:
$$
F\left(\begin{pmatrix}
            x_1\\
            x_2\\
            \vdots\\
            x_{14}\\
           \end{pmatrix}\right)
           \neq
           0
$$
Możemy zatem wywnioskować: $ker F = \{0\}$
\\
Z faktu (był na wykładzie) lub Tw. o indeksie:
$$
kerF = \{0\} \Leftrightarrow F \hspace{0.2cm} jest \hspace{0.2cm} 1-1
$$
Dalej, skoro $dim \textbf{R}^{14} = dim \textbf{R}_{13}[X] = 14 < \infty$ to $F$ jest izomorfizmem.
\\\\
Więc dla dowolnych $x_1,x_2 \dots x_{14}$ istnieje jedyny $P(x)$ spełniający warunki zadania.
\end{proof}
\section{zad. 9 (2 punkty)}
Sprawdźmy czy dany zestaw wektorów jest bazą tej p.l. to znaczy czy jest LNZ (bo $dim R^3 = 3$)
\\
Rozwiążemy układ równań odpowiadający:
$$
\alpha\begin{pmatrix}
     1 \\
     4 \\
     2 \\
    \end{pmatrix}
    + \beta
    \begin{pmatrix}
     5 \\
     -1 \\
     3 \\
    \end{pmatrix}
    + \gamma
    \begin{pmatrix}
     3 \\
     7 \\
     7 \\
    \end{pmatrix}
    = 0
$$
$$
\begin{pmatrix}
1 & 5 & 3 \\
4 & -1 & 7 \\
2 & 3 & 7 \\
\end{pmatrix}
$$
$$
\begin{pmatrix}
1 & 5 & 3 \\
0 & -21 & -5 \\
0 & -7 & 1 \\
\end{pmatrix}
$$
$$
\begin{pmatrix}
1 & 5 & 3 \\
0 & -7 & 1 \\
0 & 0 & -8 \\
\end{pmatrix}
$$
Zatem jedynym rozwiązaniem układu jest 
$\begin{pmatrix}
0 \\
0 \\
0 \\
\end{pmatrix}$
\\Mamy więc:
$$
\alpha\begin{pmatrix}
     1 \\
     4 \\
     2 \\
    \end{pmatrix}
    + \beta
    \begin{pmatrix}
     5 \\
     -1 \\
     3 \\
    \end{pmatrix}
    + \gamma
    \begin{pmatrix}
     3 \\
     7 \\
     7 \\
    \end{pmatrix}
    = 0
    \Rightarrow
    \alpha = 0 \wedge \beta = 0 \wedge \gamma = 0
$$
Więc zestaw wektorów z zadania jest LNZ i z tym jest bazą $R^3$
\\
Zatem dowolny wektor z $R^3$ powinien być rozwiązaniem układu równań.
$$\begin{cases} 0=0 \\ 0=0 \\ 0=0  \end{cases}$$

\section{zad. 11 (2 punkty)}
\begin{proof}
Skoro układ ma dwa rozwiązania to jeżeli traktujemy kolumny jako wektory to da się pewien wektor $AX = B$
przedstawić na dwa sposoby. Zatem układ kolumn nie jest LNZ bo przedstawia się niejednoznacznie w układzie kolumn macierzy. Oznaczmy wektory oznaczające kolumny: $v_1, \dots v_n$, oraz bazę p.l. rozpinanej przez te wektory $B$.\\
$dim(B) < n$ bo baza jest maksymalnym zbiorem LNZ, a zbiór $v_1, \dots v_n$ LZ.\\
Z definicji rzędu macierzy:
$$
rank(A) = dim(Lin(v_1, \dots v_n)) = dim(B) < n
$$
\end{proof}


\section{zad. 12 (2 punkty)}
\begin{proof}
%Skoro $rank(A) = m$ to wiersze traktowane jako wektory są układem LNZ wektorów.\\
Wiemy, że $dim$ p.l. rozpinanych przez kolumny i wiersze są sobie równe i są równe $rank(A) = m$.\\
Wnioskujemy, że kolumny traktowane jako wektory należą do p.l. $K^m$.\\
Skoro $dim$ p.l. rozpinanej przez kolumny jest równy $m$ a i z powyższego to kolumny są bazą $K^m$.\\
W takim razie dowolny wektor $AX = B \in K^m$ można zapisać kombinacją liniową kolumn. Układ jest niesprzeczny.
\end{proof}
Przykład:\\
$$
\begin{cases}
2x + y = 0\\
x + 2y = 0\\
3x + 3y = 0\\
\end{cases}
$$
$Rank(A) = 2$

\end{document}