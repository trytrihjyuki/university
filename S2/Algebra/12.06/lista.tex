\documentclass{article}
\usepackage[utf8]{inputenc}
\usepackage{polski}


\usepackage{amssymb, amsmath, amsfonts, amsthm, cite, mathtools, enumerate, rotating, hyperref}
\newcommand \eq[1]{\begin{equation} \begin{split}  #1 \end{split} \end{equation}}


\makeatletter
\newcommand\tab[1][1cm]{\hspace*{#1}}
\def\@seccntformat#1{%
  \expandafter\ifx\csname c@#1\endcsname\c@section\else
  \csname the#1\endcsname\quad
  \fi}
\makeatother

\title{}
\date{12.06.2020}
\author{Maurycy Borkowski}
\begin{document}
\maketitle

\section{SUMA: 11 punktów}
\section{zad. 1 (1 punkt)}
Posłuchajcie, bracia miła,\\
rzecz wam może i banalna\\
usłyszycie mój zamętek,\\
macierz zwie się ortogonalna\\

Pożałuj mię, stary, młody,\\
boć mi przyszły krwawe gody.\\
nazwa mogła być dość konwencjonalna\\
po prostu logicznie \dots ortonormalna\\
\section{zad. 4 (1 punkt)}
\begin{proof}
Dla dowolnego wektora $(x,y,z)$ z płaszczyzny i dowolnego $(x',y',z')$ z $Lin([a,b,c])$:
$$
<(x,y,z),(x',y',z')> = xx' + yy' + zz'
$$
Z def. przestrzenii liniowej i  założeń:
$$
<(x,y,z),(x',y',z')> = xx' + yy' + zz' = x\cdot k a + y\cdot k b + z\cdot k c = k\cdot (ax + by +cz) = k \cdot 0 = 0
$$
Z dowolności wyboru $(x,y,z)$ i $(x',y',z')$:\\
$Lin([a,b,c]) = W^\perp$
\end{proof}
\section{zad. 5 (1 punkt)}
\begin{proof}
Z założeń:
$$
||x|| = ||y|| = x_1^2 + x_2^2 + x_3^2 = y_1^2 + y_2^2 +y_3^2
$$
$$
<x+y,x-y> = x_1^2 - y_1^2 + x_2^2 - y_2^2 + x_3^2 - y_3^2
$$
Z pierwszego równania:
$$
||x|| - ||y|| = x_1^2 - y_1^2 + x_2^2 - y_2^2 + x_3^2 - y_3^2 = 0
$$
Zauważamy w powyżej sumie szukaną tożsamość:
$$
<x+y,x-y> = ||x|| - ||y|| = 0
$$
\end{proof}
% \section{zad. 6 (1 punkt)}
% $$
% \begin{cases}
% 2x_1 - 3x_2 + 4x_3 - 3x_4 = 0\\
% 3x_1 - x_2 + 11x_3 - 13x_4 = 0\\
% 4x_1 + x_2 + 18x_3 - 23x_4 = 0\\
% \end{cases}
% $$

\section{\\\\\\\\zad. 7 (2 punkty)}
$U,W < V$:
\subsection*{II}
Korzystamy z własności: $X \subset Y \implies Y^\perp \subset X^\perp$:
$$(U + W)^\perp \subset U^\perp$$
Analogicznie pokazujemy:
$$(U + W)^\perp \subset W^\perp$$
Z powyższych inkluzji
$$
(U + W)^\perp  \subset  U^\perp \cap W^\perp
$$
Inkluzja w drugą stronę jest w trywialna: wektory z przekroju są prostopadłymi do pewnego wektora z $U$ i $W$, ten wektor też znajduję się w $U+W$.
\subsection*{I}
Z \textbf{II}
$$
{U^\perp}^\perp \cap {W^\perp}^\perp = ({U}^\perp + {W}^\perp)^\perp
$$
Dalej:
$$
({U^\perp}^\perp \cap {W^\perp}^\perp)^\perp = (({U}^\perp + {W}^\perp)^\perp)^\perp 
$$
Korzystając z tw. ${V^\perp}^\perp = V$:
$$
(U \cap W)^\perp  =  U^\perp + W^\perp
$$
\section{zad. 8 (2 punkty)}
\subsection*{I}
Dla dowolnego $v$:
$$P_w^2(v) = P_w((v|e_1)e_1 + \dots +(v|e_k)e_k)$$
Z rodzielności iloczynu skalarnego względem dodawania:
$$P_w^2(v) = P_w((v|e_1)e_1) + \dots + P_w((v|e_k)e_k))$$
Baza jest ortonormalna więc:
$$P_w^2(v) = (v|e_1)e_1 + \dots + (v|e_k)e_k$$
Zgodnie z definicją:
$$P_w^2(v) = (v|e_1)e_1 + \dots + (v|e_k)e_k = P_w(v)$$
\subsection*{II}
$P_w$ zgodnie z definicją to jest pewna kombinacja liniowa wektorów $e_j$ czyli wektory $x = P_w(y)$ są kombinacją liniową wektorów bazy $W$ czyli należą do przestrzeni liniowej rozpinanej przez tą bazę czyli do $W$.
\section{zad. 8 (2 punkty)}
\begin{proof}
Oznaczmy jako $U$ przestrzeń dopełniającą $U \oplus W = V$.\\
Niech $v = v_u + v_w$ z odpowiednich podprzestrzeni liniowych.\\
Z definicji rzutu: $P_w(v) = v_w \neq w$:
$$
||v-w|| = ||v_u + v_w - w|| = ||v_u + X||
$$
Gdzie $X$ jest pewnym niezerowym wektorym. Dalej:
$$
||v-w|| = ||v_u + X|| > ||v_u|| = ||v_u + v_w - v_w|| = ||v_u + v_w - P_w(v)|| = ||v - P_w(v)||
$$
\end{proof}
23
\section{zad. 16 (2 punkty)}
\begin{proof}
Korzystamy z: macierz jest ortogonalna wtw. gdy $AA^T = I$.\\
A,B dowolne macierze ortogonalne:
$$
(AB)(AB)^T = (AB)B^TA^T = ABB^TA^T = AIA^T = I
$$
\end{proof}
\end{document}

