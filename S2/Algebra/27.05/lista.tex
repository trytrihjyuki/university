\documentclass{article}
\usepackage[utf8]{inputenc}
\usepackage{polski}


\usepackage{amssymb, amsmath, amsfonts, amsthm, cite, mathtools, enumerate, rotating, hyperref}
\newcommand \eq[1]{\begin{equation} \begin{split}  #1 \end{split} \end{equation}}
\DeclareMathAlphabet{\pazocal}{OMS}{zplm}{m}{n}

\makeatletter
\renewcommand*\env@matrix[1][*\c@MaxMatrixCols c]{%
  \hskip -\arraycolsep
  \let\@ifnextchar\new@ifnextchar
  \array{#1}}
\def\@seccntformat#1{%
  \expandafter\ifx\csname c@#1\endcsname\c@section\else
  \csname the#1\endcsname\quad
  \fi}
\makeatother

\title{Lista 10 (suma: 15 punktów)}
\date{27.05.2020}
\author{Maurycy Borkowski}
\begin{document}
\maketitle
\section{zad. 2 (1 punkt)}
$$
\begin{cases}
A
\begin{pmatrix}
1\\
0\\
\end{pmatrix}
= 
\begin{pmatrix}
-\frac{1}{2}\\
0\\
\end{pmatrix}\\
A
\begin{pmatrix}
1\\
1\\
\end{pmatrix}
= 
\begin{pmatrix}
1\\
1\\
\end{pmatrix}\\
\end{cases}
$$
Mamy więc:
$$
\begin{cases}
a = -\frac{1}{2}\\
c \cdot 1 = 0\\
a + b = 1\\
c + d = 1\\
\end{cases}
$$
zatem:
$$
A =
\begin{pmatrix}
-\frac{1}{2} && \frac{3}{2}\\
0 && 1\\
\end{pmatrix}
$$
\section{zad. 3 (1 punkt)}
\subsection*{I}
$$\pazocal{X}_A(t) = 
\begin{vmatrix}
t-2 && 0\\
0 && t-3\\
\end{vmatrix}
= (t-2)(t-3)
$$
Podprzestrzenie własne dla wartości własnych 2,3:
$$
\begin{pmatrix}
2 && 0\\
0 && 3\\
\end{pmatrix}
\begin{pmatrix}
x\\
y\\
\end{pmatrix} =
\begin{pmatrix}
2x\\
2y\\
\end{pmatrix}
$$
$$
V^2 = \{\begin{pmatrix}
k\\
0\\
\end{pmatrix} : k \in \mathbb{R}\}
$$
$$
\begin{pmatrix}
2 && 0\\
0 && 3\\
\end{pmatrix}
\begin{pmatrix}
x\\
y\\
\end{pmatrix} =
\begin{pmatrix}
3x\\
3y\\
\end{pmatrix}
$$
$$
V^3 = \{\begin{pmatrix}
0\\
k\\
\end{pmatrix} : k \in \mathbb{R}\}
$$
\subsection*{II}
$$\pazocal{X}_A(t) = 
\begin{vmatrix}
t-1 && -2\\
0 && t-1\\
\end{vmatrix}
= (t-1)^2
$$
Podprzestrzeń własna dla wartości własnej 1:
$$
\begin{pmatrix}
1 && 2\\
0 && 1\\
\end{pmatrix}
\begin{pmatrix}
x\\
y\\
\end{pmatrix} =
\begin{pmatrix}
x + 2y\\
y\\
\end{pmatrix} = 
\begin{pmatrix}
x\\
y\\
\end{pmatrix}
$$
$$
V^1 = \{\begin{pmatrix}
k\\
0\\
\end{pmatrix} : k \in \mathbb{R}\}
$$
\section{zad. 4 (1 punkt)}
$$
\begin{vmatrix}
t-5 && -1\\
0 && t-5\\
\end{vmatrix}=
\begin{vmatrix}
t-5 && 0\\
0 && t-5\\
\end{vmatrix}
= (t-5)^2
$$
Sprawdźmy dla krotność geometryczną wartości własnej (wymiar podprzestrzeni własnej) obu macierzy :
$$
\begin{pmatrix}
5 && 0\\
0 && 5\\
\end{pmatrix}
\begin{pmatrix}
x\\
y\\
\end{pmatrix} =
\begin{pmatrix}
5x\\
5y\\
\end{pmatrix} = 
\begin{pmatrix}
5x\\
5y\\
\end{pmatrix}
$$
$$
V^5 = \{\begin{pmatrix}
k\\
p\\
\end{pmatrix} : k,p \in \mathbb{R}\}
$$
Macierz $\begin{pmatrix}
5 && 0\\
0 && 5\\
\end{pmatrix}$ diagonalizuje się spełnia wszystkie warunki twierdzenia (wymiar pp. wł. jest równy krotności pierwiastka wielomianu charakterystycznego).
$$
\begin{pmatrix}
5 && 1\\
0 && 5\\
\end{pmatrix}
\begin{pmatrix}
x\\
y\\
\end{pmatrix} =
\begin{pmatrix}
5x + y\\
5y\\
\end{pmatrix} = 
\begin{pmatrix}
5x\\
5y\\
\end{pmatrix}
$$
$$
V^5 = \{\begin{pmatrix}
k\\
0\\
\end{pmatrix} : k \in \mathbb{R}\}
$$
Macierz $\begin{pmatrix}
5 && 1\\
0 && 5\\
\end{pmatrix}$ nie diagonalizuje się wymiar pp. wł.  nie jest równy krotności pierwiastka wielomianu charakterystycznego (wymiar 1, krotność 2).
\\\\\\\\
\section{zad. 5 (1 punkt)}
\subsection*{I}
$$\pazocal{X}_A(t) =
\begin{vmatrix}
t-1 && 0 && 0\\
0 && t-1 && 0\\
0 && 0 && t-1\\
\end{vmatrix} =
(t-1)^3
$$
Jedna wartość własna 1 o krotności algebraicznej 3. Sprawdźmy krotność geometryczną:
$$
\begin{pmatrix}
1 && 0 && 0\\
0 && 1 && 0\\
0 && 0 && 1\\
\end{pmatrix}
\begin{pmatrix}
x\\
y\\
z\\
\end{pmatrix} =
\begin{pmatrix}
x\\
y\\
z\\
\end{pmatrix}
$$
$$
V^1 = \mathbb{R}^3
$$
Krotność geometryczna równa krotności algebraicznej (równe 3).
\subsection*{II}
$$\pazocal{X}_A(t) =
\begin{vmatrix}
t-2 && 0 && 0\\
0 && t+1 && 0\\
0 && 0 && t+1\\
\end{vmatrix} =
(t-2)(t+1)^2
$$
Dwie wartości własne: -1 o krotności algebraicznej 2 i 2 o krotności alg. 1. Sprawdźmy krotności geometryczne:
$$
\begin{pmatrix}
2 && 0 && 0\\
0 && -1 && 0\\
0 && 0 && -1\\
\end{pmatrix}
\begin{pmatrix}
x\\
y\\
z\\
\end{pmatrix} =
\begin{pmatrix}
-x\\
-y\\
-z\\
\end{pmatrix}
$$
$$
V^{-1} = \{\begin{pmatrix}
0\\
k\\
p\\
\end{pmatrix} : k,p \in \mathbb{R}\}
$$
Krotność geometryczna równa krotności algebraicznej (równe 2).
$$
\begin{pmatrix}
2 && 0 && 0\\
0 && -1 && 0\\
0 && 0 && -1\\
\end{pmatrix}
\begin{pmatrix}
x\\
y\\
z\\
\end{pmatrix} =
\begin{pmatrix}
2x\\
2y\\
2z\\
\end{pmatrix}
$$
$$
V^{2} = \{\begin{pmatrix}
k\\
0\\
0\\
\end{pmatrix} : k \in \mathbb{R}\}
$$
Krotność geometryczna równa krotności algebraicznej (równe 1).
\section{zad. 6 (1 punkt)}
\begin{proof}
Wielomianem charakterystycznym macierzy górnotrójkątnej jest:
$$
\pazocal{X}_A(t) =(t - a_{11})(t - a_{22})\dots(t - a_{nn})
$$
Z założeń $a_{ii} \neq a_{jj}$ dla $i \neq j$, zatem wielomian charakterystyczny ma n różnych pierwiastków, dokładnie: $a_{11},a_{22},\dots,a_{nn}$.\\
Wszystkie wartości własne są proste, więc całe widmo też jest proste. Z twierdzenia jeżeli widmo jest proste to macierz operatora jest diagonalizowalna.
\end{proof}
\section{zad. 8 (2 punkty)}
$$M =
\begin{pmatrix}
3 && 0\\
4 && 2\\
\end{pmatrix}
$$
$$
\pazocal{X}_M(t) =(t-3)(t-2)
$$
Dwie wartości własne $\lambda = 3, \mu = 2$ szukamy wektorów własnych:
\subsection*{$\lambda = 3$}
$$
\begin{pmatrix}
3 && 0\\
4 && 2\\
\end{pmatrix}
\begin{pmatrix}
x\\
y\\
\end{pmatrix}
=
\begin{pmatrix}
3x\\
3y\\
\end{pmatrix}
$$
Wektor własny: $\begin{pmatrix}
1\\
4\\
\end{pmatrix}$
\subsection*{$\mu = 2$}
$$
\begin{pmatrix}
3 && 0\\
4 && 2\\
\end{pmatrix}
\begin{pmatrix}
x\\
y\\
\end{pmatrix}
=
\begin{pmatrix}
2x\\
2y\\
\end{pmatrix}
$$
Wektor własny: $\begin{pmatrix}
0\\
1\\
\end{pmatrix}$
Zatem M = $\begin{pmatrix}
1 && 0\\
4 && 1\\
\end{pmatrix}
\begin{pmatrix}
3 && 0\\
0 && 2\\
\end{pmatrix}
\begin{pmatrix}
1 && 0\\
-4 && 1\\
\end{pmatrix}$
$$
M^6 = \begin{pmatrix}
1 && 0\\
4 && 1\\
\end{pmatrix}
\begin{pmatrix}
3^6 && 0\\
0 && 2^6\\
\end{pmatrix}
\begin{pmatrix}
1 && 0\\
-4 && 1\\
\end{pmatrix} =
\begin{pmatrix}
3^6 && 0\\
4\cdot3^6 && 2^6\\
\end{pmatrix}
\begin{pmatrix}
1 && 0\\
-4 && 1\\
\end{pmatrix}
= \begin{pmatrix}
3^6 && 0\\
2660 && 2^6\\
\end{pmatrix}
$$
% \section{zad. 9 (2 punkty)}
% $$A=
% \begin{pmatrix}
% 3\sqrt3 && -7\\
% 1 && -\sqrt3\\
% \end{pmatrix}
% $$
\section{zad. 10 (2 punkty)}
\subsection*{I}
Wszystkie wektory należące do płaszczyzny po rzucie na nią nie zmieniają się. Wartość własna 1, pp. własna płaszczyzna, na którą jest rzut.\\
Wektory prostopadłe do niej po rzucie się zerują. Wartość własna 0, pp. własna wektory prostopadłe do płaszczyzny.
\subsection*{II}
Wszystkie wektory należące do płaszczyzny po odbiciu symetrycznym względem niej nie zmieniają się. Wartość własna 1, pp. własna płaszczyzna, na którą jest rzut.\\
Wektory prostopadłe do niej po wykonaniu symetrii \textit{obracają się}. Wartość własna -1, pp. własna wektory prostopadłe do płaszczyzny.
\section{zad. 11 (2 punkty)}
\subsection*{I}
$$M =
\begin{pmatrix}
1 && 4 && -1\\
0 && 3 && -5\\
0 && 0 && -2\\
\end{pmatrix}
$$
$$
\pazocal{X}_M(t) =(t-1)(t-3)(t+2)
$$
Trzy wartości własne $\lambda = 1, \mu = 3, \gamma = -2$ szukamy wektorów własnych (małe rachunki robiłem na kartce do obliczania pp. wł.):
\subsection*{$\lambda = 1$}
$$
\begin{pmatrix}
1 && 4 && -1\\
0 && 3 && -5\\
0 && 0 && -2\\
\end{pmatrix}
\begin{pmatrix}
x\\
y\\
z\\
\end{pmatrix} =
\begin{pmatrix}
x\\
y\\
z\\
\end{pmatrix}
$$
$$
V^{1} = \{\begin{pmatrix}
k\\
0\\
0\\
\end{pmatrix} : k \in \mathbb{R}\}
$$ 
\subsection*{$\mu = 3$}
$$
\begin{pmatrix}
1 && 4 && -1\\
0 && 3 && -5\\
0 && 0 && -2\\
\end{pmatrix}
\begin{pmatrix}
x\\
y\\
z\\
\end{pmatrix} =
\begin{pmatrix}
3x\\
3y\\
3z\\
\end{pmatrix}
$$
$$
V^{3} = \{\begin{pmatrix}
2k\\
k\\
0\\
\end{pmatrix} : k \in \mathbb{R}\}
$$
\subsection*{$\gamma = -2$}
$$
\begin{pmatrix}
1 && 4 && -1\\
0 && 3 && -5\\
0 && 0 && -2\\
\end{pmatrix}
\begin{pmatrix}
x\\
y\\
z\\
\end{pmatrix} =
\begin{pmatrix}
-2x\\
-2y\\
-2z\\
\end{pmatrix}
$$
$$
V^{-1} = \{\begin{pmatrix}
-k\\
k\\
k\\
\end{pmatrix} : k \in \mathbb{R}\}
$$
Ostatecznie:
$$
M =
\begin{pmatrix}
1 && 2 && -1\\
0 && 1 && 1\\
0 && 0 && 1\\
\end{pmatrix}
\begin{pmatrix}
1 && 0 && 0\\
0 && 3 && 0\\
0 && 0 && -2\\
\end{pmatrix}
\begin{pmatrix}
1 && 2 && -1\\
0 && 1 && 1\\
0 && 0 && 1\\
\end{pmatrix}^{-1}
$$
\subsection*{II}
$$M =
\begin{pmatrix}
2 && -1 && 2\\
5 && -3 && 3\\
-1 && 0 && -2\\
\end{pmatrix}
$$
$$
\pazocal{X}_M(t) = (t+1)^3
$$
Jedna wartość własna $\lambda = -1$ szukamy wektorów własnych:
\subsection*{$\lambda = -1$}
$$
\begin{pmatrix}
2 && -1 && 2\\
5 && -3 && 3\\
-1 && 0 && -2\\
\end{pmatrix}
\begin{pmatrix}
x\\
y\\
z\\
\end{pmatrix} =
\begin{pmatrix}
-x\\
-y\\
-z\\
\end{pmatrix}
$$
$$
\begin{cases}
2x - y + 2z = -x\\
5x -3y + 3z = -y\\
-x -2z = -z
\end{cases}
$$
Z powyższego:
$$
V^{-1} = \{\begin{pmatrix}
k\\
k\\
-k\\
\end{pmatrix} : k \in \mathbb{R}\}
$$
Krotność algebraiczna $\neq$ krotność geometryczna ($1 < 3$). Zatem macierz nie jest diagonalizowalna.\\\\\\\\\\\\\\\\\\\\\
\section{zad. 12 (2 punkty)}
\subsection*{I}
$$
A=
\begin{pmatrix}
1 && 0 && -1\\
2 && 0 && 2\\
2 && -1 && 4\\
\end{pmatrix}
$$
$$
\pazocal{X}_A(t) = (t-2)^2(t-1)
$$
\subsection*{$\lambda = 2$}
$$
\begin{pmatrix}
1 && 0 && -1\\
2 && 0 && 2\\
2 && -1 && 4\\
\end{pmatrix}
\begin{pmatrix}
x\\
y\\
z\\
\end{pmatrix}=
\begin{pmatrix}
2x\\
2y\\
2z\\
\end{pmatrix}
$$
$$
\begin{cases}
x - z = 2x\\
2x + 2y = 2y\\
2x - y + 4z = 2z\\
\end{cases}
$$
Ten układ nie ma rozwiązania nietrywialnego. Zatem wymiar pp. wł. jest równy 0. Więc $A$ nie jest diagonalizowalna.
\subsection*{II}
$$
A=
\begin{pmatrix}
4 && -5 && 2\\
5 && -7 && 3\\
6 && -9 && 4\\
\end{pmatrix}
$$
$$
\pazocal{X}_A(t) =
\begin{vmatrix}
t-4 && 5 && -2\\
-5 && t+7 && -3\\
-6 && 9 && t-4\\
\end{vmatrix}
=
(t-4)
\begin{vmatrix}
t+7 && -3\\
9 && t-4\\
\end{vmatrix}
-5
\begin{vmatrix}
-5 && -3\\
-6 && t-4\\
\end{vmatrix}
+2
\begin{vmatrix}
-5 && t+7\\
-6 && 9\
\end{vmatrix} =
$$
$$
t^3  - t^2
$$
Wielomian charakterystyczny nie rozkłada się na czynniki liniowe w $\mathbb{R}$ zatem $A$ nie jest diagonalizowalna.
\\\\\\\\\\\\\\\\\\\\\\\\\\\\\\
\section{zad. 14 (2 punkty)}
Metodą prób i błędów wyznaczyłem wzór następnie udowodniłem jego poprawność.
$$
\begin{pmatrix}
\lambda && 1 && 0\\
0 && \lambda && 1\\
0 && 0 && \lambda\\
\end{pmatrix}^n
=
\begin{pmatrix}
\lambda^n && n\lambda^{n-1} && \frac{n(n-1)}{2}\lambda^{n-2}\\
0 && \lambda^n && n\lambda^{n-1}\\
0 && 0 && \lambda^n\\
\end{pmatrix}
$$
\begin{proof}
\subsection*{\\podstawa indukcji:}
Dla 1 mamy ten sam wzór, działą.
\subsection*{krok:}
Załóżmy, że dla n wzór jest poprawny. Teraz:
$$
\begin{pmatrix}
\lambda && 1 && 0\\
0 && \lambda && 1\\
0 && 0 && \lambda\\
\end{pmatrix}^{n+1}
=
\begin{pmatrix}
\lambda^n && n\lambda^{n-1} && \frac{n(n-1)}{2}\lambda^{n-2}\\
0 && \lambda^n && n\lambda^{n-1}\\
0 && 0 && \lambda^n\\
\end{pmatrix}
\begin{pmatrix}
\lambda && 1 && 0\\
0 && \lambda && 1\\
0 && 0 && \lambda\\
\end{pmatrix}
=
$$
$$
=\begin{pmatrix}
\lambda^{n+1} && (n+1)\lambda^{n} && \frac{n(n+1)}{2}\lambda^{n-1}\\
0 && \lambda^{n+1} && (n+1)\lambda^{n}\\
0 && 0 && \lambda^{n+1}\\
\end{pmatrix}
$$
Na mocy zasady indukcji ten wzór jest poprawny dla dowolnego $n \in \mathbb{N}$
\end{proof}
\end{document}