\documentclass{article}
\usepackage[utf8]{inputenc}
\usepackage{polski}


\usepackage{amssymb, amsmath, amsfonts, amsthm, cite, mathtools, enumerate, rotating, hyperref}
\newcommand \eq[1]{\begin{equation} \begin{split}  #1 \end{split} \end{equation}}


\makeatletter
\newcommand\tab[1][1cm]{\hspace*{#1}}
\def\@seccntformat#1{%
  \expandafter\ifx\csname c@#1\endcsname\c@section\else
  \csname the#1\endcsname\quad
  \fi}
\makeatother

\title{}
\date{03.06.2020}
\author{Maurycy Borkowski}
\begin{document}
\maketitle

\section{SUMA: 14 punktów}

\section{zad. 3 (1 punkt)}
Możliwe krotności: (4,1,1), (3,2,1), (2,2,2). Razy 3! (jakie wartości własne).\\
Ostatecznie możliwych postaci jest: $3*6 = 18$.
\section{zad. 4 (1 punkt)}
\subsection*{I}
$$
\begin{cases}
a^2 + bc = 4\\
ab + bd = 1\\
ac + cd = 0\\
cb + d^2 = 4\\
\end{cases}
$$
Rozwiązanie:
$$
\begin{pmatrix}
2 & \frac{1}{4}\\
0 & 2\\
\end{pmatrix}
$$
\subsection*{II}
$$
\begin{pmatrix}
3 & 1\\
-1 & 5\\
\end{pmatrix}
$$
Ma jedną wartość własną 4, krotność 2. Jedna klatka:
$$
J =
\begin{pmatrix}
4 & 1\\
0 & 4\\
\end{pmatrix}
$$
Szukamy bazy jordanowskiej:\\
Znajdujemy wektor własny 
$\begin{pmatrix}
1\\
1\\
\end{pmatrix}$ więc dalej:
$$
X =
\begin{pmatrix}
1 & 0\\
1 & 1\\
\end{pmatrix}
$$
Szukana macierz:
$$
M = X
\begin{pmatrix}
2 & \frac{1}{4}\\
0 & 2\\
\end{pmatrix}
X^{-1}
$$
\section{zad. 5 (1 punkt)}
\subsection*{I}
$$A =
\begin{pmatrix}
1 & -3 & 4\\
4 & -7 & 8\\
6 & -7 & 7\\
\end{pmatrix}
$$
Wartości własne: 3, -1 (krotność 2).\\
$rank A = 2$ więc będzie jedna klatka rozmiaru 2 odpowiadająca -1. Zatem:
$$J =
\begin{pmatrix}
3 & 0 & 0\\
0 & -1 & 1\\
0 & 0 & -1\\
\end{pmatrix}
$$
\subsection*{II}
$$B =
\begin{pmatrix}
4 & -5 & 7\\
1 & -4 & 9\\
-4 & 0 & 5\\
\end{pmatrix}
$$
Wartości własne:1, 2+i, 2-i.\\
Ta macierz diagonalizuję się nad $\mathbb{C}$
$$J =
\begin{pmatrix}
1 & 0 & 0\\
0 & 2-i & 1\\
0 & 0 & 2+i\\
\end{pmatrix}
$$
\section{zad. 6 (1 punkt)}
$$
A=
\begin{pmatrix}
4 & -5\\
1 & 2\\
\end{pmatrix}
$$
Wartości własne: 3+2i, 3-2i.\\
Szukamy wektorów własnych:
$$
\begin{pmatrix}
1 \pm 2i\\
1\\
\end{pmatrix}
$$
Zatem:
$$
A =
\begin{pmatrix}
1 + 2i & 1 - 2i\\
1 & 1\\
\end{pmatrix}
\begin{pmatrix}
3 + 2i & 0\\
0 & 3 - 2i\\
\end{pmatrix}
\begin{pmatrix}
1 + 2i & 1 - 2i\\
1 & 1\\
\end{pmatrix}^{-1}
$$
% \section{zad. 7 (2 punkty)}
% Zauważmy, że niezależnie od turnieju macierz ma same zera na diagonali (gracze nie wygrywają z samymi sobą bo nie ma takich meczy). Zatem oznacza to, że dla
% Po przemnożeniu dwóch macierzy turniejowych
\section{zad. 8 (2 punkty)}
\subsection*{I}
$$A=
\begin{pmatrix}
2 & -1 & 2\\
5 & -3 & 3\\
-1 & 0 & -2\\
\end{pmatrix}
$$
Jest jedna wartość własna: -1\\
Przestrzeń liniowa:
$$
\begin{cases}
2x - y + 2z = -x\\
5x - 3y + 3z = -y\\
-x - 2z = -z\\
\end{cases}
$$
$$
V^{-1} = lin(\left \{\begin{pmatrix}
 1\\
 1\\
-1\\
\end{pmatrix}\right\})
$$
Przestrzenie pierwiastkowe:
$$
ker(A+1)^2 = lin(\left \{\begin{pmatrix}
 1\\
 1\\
-1\\
\end{pmatrix}
\begin{pmatrix}
 0\\
 1\\
 1\\
\end{pmatrix}\right\})
$$
$$
ker(A+1)^3 = \mathbb{R}^3
$$
Baza jordanowska (losowy, przemnożony przez A+E, $(A+E)^2$:
$$
\begin{pmatrix}
 1\\
 0\\
 0\\
\end{pmatrix}
\begin{pmatrix}
 2\\
 5\\
 -1\\
\end{pmatrix}
\begin{pmatrix}
 2\\
 2\\
 -2\\
\end{pmatrix}
$$
Klatka będzie jedna rozmiaru 3:
Ostatecznie:
$$
A =
\begin{pmatrix}
 1 & 2 & 2\\
 0 & 5 & 2\\
 0 & -1 &-2\\
\end{pmatrix}
\begin{pmatrix}
 -1 & 1 & 0\\
 0 & -1 & 1\\
 0 & 0 &-1\\
\end{pmatrix}
\begin{pmatrix}
 1 & 2 & 2\\
 0 & 5 & 2\\
 0 & -1 &-2\\
\end{pmatrix}^{-1}
$$
\subsection*{II}
$$
B=
\begin{pmatrix}
0 & -2 & 3 & 2\\
1 & 1 & -1 & -1\\
0 & 0 & 2 & 0\\
1 & -1 & 0 & 1\\
\end{pmatrix}
$$
wartości własne: 0 i 2 z krotnościami 2
\\Przestrzenie liniowe:
$$
\begin{cases}
-2y + 3z + 2h = 2x\\
x + y - z - h = 2y\\
2z = 2z\\
x - y + h = 2h\\
\end{cases}
$$
$$
V^2 = lin(\left \{\begin{pmatrix}
 1\\
 0\\
0\\
1\\
\end{pmatrix}\right\})
$$
$$
\begin{cases}
-2y + 3z + 2h = 0\\
x + y - z - h = 0\\
2z = 0\\
x - y + h = 0\\
\end{cases}
$$
$$
V^0 = lin(\left \{\begin{pmatrix}
 0\\
 1\\
0\\
1\\
\end{pmatrix}
\right\})
$$
Przestrzenie pierwiastkowe dla 2:
$$
ker(B-2)^2 = lin \{\begin{pmatrix}
 2\\
 0\\
 1\\
 1\\
\end{pmatrix}
\begin{pmatrix}
 0\\
 0\\
 -1\\
 1\\
\end{pmatrix}\})
$$
Przestrzenie pierwiastkowe dla 0:
$$
ker(B)^3 = ker(B)^2 = lin \{\begin{pmatrix}
 1\\
 0\\
 0\\
 0\\
\end{pmatrix}
\begin{pmatrix}
 0\\
 1\\
 0\\
 1\\
\end{pmatrix}\})
$$
Baza jordanowska (losowy, przemnożony przez B, losowy , przemnożony B):
$$
\begin{pmatrix}
 1\\
 0\\
 0\\
 0\\
\end{pmatrix}
\begin{pmatrix}
 0\\
 1\\
 0\\
 1\\
\end{pmatrix}
\begin{pmatrix}
 2\\
 0\\
 1\\
 1\\
\end{pmatrix}
\begin{pmatrix}
 1\\
 0\\
 0\\
 1\\
\end{pmatrix}
$$
Klatki  dwie rozmiaru 2:
Ostatecznie:
$$
B =
\begin{pmatrix}
 0 & 1 & 1 & 2\\
 1 & 0 & 0 & 0\\
 0 & 0 & 0 & 1\\
 1 & 0 & 0 & 1\\
\end{pmatrix}
\begin{pmatrix}
 0 & 1 & 0 & 0\\
 0 & 0 & 0 & 0\\
 0 & 0 & 2 & 1\\
 0 & 0 & 0 & 2\\
\end{pmatrix}
\begin{pmatrix}
 0 & 1 & 1 & 2\\
 1 & 0 & 0 & 0\\
 0 & 0 & 0 & 1\\
 1 & 0 & 0 & 1\\
\end{pmatrix}^{-1}
$$\\\\\\\\\\\\\\\\
\section{zad.9 (2 punkty)}
$$
\begin{pmatrix}
 2 & 1\\
 3 & 0\\
\end{pmatrix}
\begin{pmatrix}
 6 & 1\\
 0 & 6\\
\end{pmatrix}
\begin{pmatrix}
 2 & 1\\
 3 & 0\\
\end{pmatrix}^{-1}
=
\begin{pmatrix}
 8 & \frac{-4}{3}\\
 3 & 4\\
\end{pmatrix}
$$
\section{zad. 11 (2 punkty)}
\subsection*{I}
$$
A^2 = I
$$
A jest pierwiastkiem wielomianu $t^2 - 1 = (t-1)(t+1)$ który jest podzielny przez wielomian minimalny. Więc wielomian minimalny składa się tylko z czynników liniowych, zatem jest diagonalizowalna.
\subsection*{II}
$$
A^2 = A
$$
Jest to warunek rzutu. Skoro to jest rzut to macierz jest diagonalizowalna, więc postać Jordana to będzie macierz diagonalna.
\section{zad. 13 (2 punkty)}
Jeżeli $n \geq 10000$ teza jest oczywista.\\\\
Pokażemy, że stopień nimpotentności $(= 10000)$ jest $\leq n$.\\
Jeżeli A jest macierzą nipotntną to jej wielomianem minimalnym jest $x^k$ dla pewnego $k$. Dalej, wielomian minimalny jest dzielnikiem wielomianu charakterystycznego, którego stopień jest równy $n$. Z tego $k < n$ czyli $10000 < n$.
\section{zad. 14 (2 punkty)}
\begin{proof}
Załóżmy niewprost, że istnieją takie $a$ ($BSO \quad a_0 \neq 0$) \\(notacja $v_k = N^kv$):
$$
a_0v_0 + \dots a_k v_k = 0
$$
Mnożąc obustronnie przez  $N^{k}$ korzystamy z $N^j = 0$ dla $j > k$:
$$
a_0 N^{k} v_0 = a_0v_k = 0
$$
Z założeń $v_k = n^kv \neq 0$ otrzymujemy sprzeczność $a_0 = 0$
\end{proof}

\end{document}

