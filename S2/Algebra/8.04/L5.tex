\documentclass{article}
\usepackage[utf8]{inputenc}
\usepackage{polski}


\usepackage{amssymb, amsmath, amsfonts, amsthm, cite, mathtools, enumerate, rotating, hyperref}
\newcommand \eq[1]{\begin{equation} \begin{split}  #1 \end{split} \end{equation}}


\makeatletter
\renewcommand*\env@matrix[1][*\c@MaxMatrixCols c]{%
  \hskip -\arraycolsep
  \let\@ifnextchar\new@ifnextchar
  \array{#1}}
\def\@seccntformat#1{%
  \expandafter\ifx\csname c@#1\endcsname\c@section\else
  \csname the#1\endcsname\quad
  \fi}
\makeatother

\title{Lista 5}
\date{8.04.2020}
\author{Maurycy Borkowski}
\begin{document}
\maketitle

\section{zad. 2 (1 punkt)}
\subsection*{I}
Łatwo zauważyć, że z tej postaci można dojść do macierzy diagonalnej (też górnotrójkątnej) stosując operacje elementarne. Dodajemy kolejne wiersze przemnożone przez odpowieni skalar od góry do dołu, idąc od wiersza $n$ do wiersza $1$.\\
Z wniosku z wykładu, wiemy że wyznacznik macierzy górnotrójkątnej to iloczyn wyrazów na jej głównej przekątnej:
$$
\begin{vmatrix}
a_{11} & 0 & 0 & \cdots & 0 \\
a_{21} & a_{22} & 0 & \cdots & 0 \\
a_{31} & a_{32} & a_{33} & \cdots & 0 \\
\vdots & \vdots & \vdots & \quad & \vdots \\
a_{n1} & a_{n2} & a_{n3} & \cdots & a_{nn} \\
\end{vmatrix} = 
\begin{vmatrix}
a_{11} & 0 & 0 & \cdots & 0 \\
0 & a_{22} & 0 & \cdots & 0 \\
0 & 0 & a_{33} & \cdots & 0 \\
\vdots & \vdots & \vdots & \quad & \vdots \\
0 & 0 & 0 & \cdots & a_{nn} \\
\end{vmatrix} = a_{11} \cdot a_{22} \dots a_{nn}
$$
\subsection*{II}
Najpierw używamy $\lfloor{\frac{n}{2}} \rfloor$ operacji zmian wierszy aż dojdziemy do macierzy górnotrójkątnej. Później z wniosku z wykładu liczymy wyznacznik:
$$
\begin{vmatrix}
0 & \cdots & 0 & 0 & a_{1n} \\
0 & \cdots & 0 & a_{2,n-1} & a_{2n} \\
0 & \cdots & a_{3,n-2} & a_{3,n-1} & a_{3n} \\
\vdots & \quad & \vdots & \vdots & \vdots\\
a_{n1} & \cdots & a_{n,n-2} & a_{n,n-1} & a_{nn} \\
\end{vmatrix} = {(-1)}^{\lfloor{\frac{n}{2}} \rfloor}
\begin{vmatrix}
a_{n1} & \cdots & a_{n,n-2} & a_{n,n-1} & a_{nn} \\
\vdots & \vdots & \vdots & \quad & \vdots \\
0 & \cdots & a_{3,n-2} & a_{3,n-1} & a_{3n} \\
0 & \cdots & 0 & a_{2,n-1} & a_{2n} \\
0 & \cdots & 0 & 0 & a_{1n} \\
\end{vmatrix} = 
$$
$$
= {(-1)}^{\lfloor{\frac{n}{2}} \rfloor} \cdot a_{n1} \dots a_{2,n-1} \cdot a_{1n}
$$
\\\\
\section{zad. 3 (1 punkt)}
Korzystając z rozwinięcia Laplace'a:
\subsection*{I}
$$
\begin{vmatrix}
a & 3 & 0 & 5 \\
0 & b & 0 & 2 \\
1 & 2 & c & 3 \\
0 & 0 & 0 & d \\
\end{vmatrix} =
d
\begin{vmatrix}
a & 3 & 0 \\
0 & b & 0 \\
1 & 2 & c \\
\end{vmatrix} =
dc
\begin{vmatrix}
a & 3\\
0 & b\\
\end{vmatrix} =dc \cdot (ab - 3\cdot0) = abcd
$$
\subsection*{II}
$$
\begin{vmatrix}
1 & 0 & 2 & a \\
2 & 0 & b & 0 \\
3 & c & 4 & 5 \\
d & 0 & 0 & 0 \\
\end{vmatrix} = 
d
\begin{vmatrix}
0 & 2 & a \\
0 & b & 0 \\
c & 4 & 5 \\
\end{vmatrix} = 
dc
\begin{vmatrix}
2 & a \\
b & 0 \\
\end{vmatrix} = =dc \cdot (2\cdot0 - ab) = -abcd
$$
\section{zad. 4 (1 punkt)}
Najpierw stosujemy operacje elementarne dodawania wierszy pomnożonych przez odpowiedni skalar, a później stosujemy rozwinięcie Laplace'a.
\subsection*{I}
$$
\begin{vmatrix}
1 & 2 & 3 & 4 \\
-3 & 2 & -5 & 13 \\
1 & -2 & 10 & 4 \\
-2 & 9 & -8 & 25 \\
\end{vmatrix} = 
\begin{vmatrix}
0 & 4 & -7 & 0 \\
-3 & 2 & -5 & 13 \\
1 & -2 & 10 & 4 \\
-2 & 9 & -8 & 25 \\
\end{vmatrix} = 
(-1)\cdot4
\begin{vmatrix}
-3 & -5 & 13 \\
1 & 10 & 4 \\
-2 & -8 & 25 \\
\end{vmatrix}
-7
\begin{vmatrix}
-3 & 2  & 13 \\
1 & -2  & 4 \\
-2 & 9  & 25 \\
\end{vmatrix} = 301
$$
\subsection*{II}
$$
\begin{vmatrix}
7 & 6 & 9 & 4 & -4 \\
1 & 0 & -2 & 6 & 6 \\
7 & 8 & 9 & -1 & -6 \\
1 & -1 & -2 & 4 & 5 \\
-7 & 0 & -9 & 2 & -2 \\
\end{vmatrix} = 
\begin{vmatrix}
0 & 6 & 0 & 6 & -6 \\
1 & 0 & -2 & 6 & 6 \\
7 & 8 & 9 & -1 & -6 \\
1 & -1 & -2 & 4 & 5 \\
-7 & 0 & -9 & 2 & -2 \\
\end{vmatrix} = 
\begin{vmatrix}
0 & 6 & 0 & 6 & -6 \\
1 & -6 & -2 & 0 & 12 \\
7 & 8 & 9 & -1 & -6 \\
1 & -1 & -2 & 4 & 5 \\
-7 & 0 & -9 & 2 & -2 \\
\end{vmatrix} = 
$$
$$
=
\begin{vmatrix}
0 & 6 & 0 & 6 & -6 \\
1 & -6 & -2 & 0 & 12 \\
7 & 8 & 9 & -1 & -6 \\
1 & -1 & -2 & 4 & 5 \\
0 & 8 & 0 & 1 & -8 \\
\end{vmatrix} = 
\begin{vmatrix}
0 & 6 & 0 & 6 & -6 \\
1 & -6 & -2 & 0 & 12 \\
7 & 8 & 9 & -1 & -6 \\
1 & -1 & -2 & 4 & 5 \\
0 & 0 & 0 & -7 & 0 \\
\end{vmatrix} = 
(-1)(-7)
\begin{vmatrix}
0 & 6 & 0 & -6 \\
1 & -6 & -2 &  12 \\
7 & 8 & 9 & -6 \\
1 & -1 & -2 &  5 \\
\end{vmatrix} = 
$$
$$
=7
\begin{vmatrix}
0 & 6 & 0 & -6 \\
1 & -6 & -2 &  12 \\
7 & 8 & 9 & -6 \\
0 & 5 & 0 &  -7 \\
\end{vmatrix}
=7(-1)6
\begin{vmatrix}
1  & -2 &  12 \\
7  & 9 & -6 \\
0  & 0 &  -7 \\
\end{vmatrix} +
7(-1)(-6)
\begin{vmatrix}
1 & -6 & -2 \\
7 & 8 & 9 \\
0 & 5 & 0 \\
\end{vmatrix} = 42(161 - 115) = 1932
$$
\section{zad. 6 (1 punkt)}
a)\\
Wiemy: $det(\lambda A) = {\lambda}^n det(A)$ więc wyznacznik zmieni się na przeciwny gdy $n$ będzie nieparzyste.
\\\\b)\\
\\\\c)\\
Możemy zamieniać kolejnie kolumny, zmian będzie potrzeba $n -1$ więc wyznacznik będzie trzeba pomnożyć przez ${(-1)}^{n-1}$.
\\\\d)\\
Podobnie jak w zad. 2 więc wyznacznik będzie trzeba pomnożyć przez ${(-1)}^{\lfloor{\frac{n}{2}} \rfloor}$.
\section{zad. 7 (2 punkty)}
\subsection*{I}
Dodajmy pierwszy wiersz do każdego wiersza:
$$
\begin{vmatrix}
1 & 2 & 3 & \dots & n\\
-1 & 0 & 3 & \dots & n\\
-1 & -2 & 0 & \dots & n\\
\vdots & \vdots & \vdots & \quad &\vdots\\
-1 & -2 & -3 & \dots & 0\\
\end{vmatrix} = 
\begin{vmatrix}
1 & 2 & 3 & \dots & n\\
0 & 2 & 6 & \dots & 2n\\
0 & 0 & 3 & \dots & 3n\\
\vdots & \vdots & \vdots & \quad &\vdots\\
0 & 0 & 0 & \dots & n\\
\end{vmatrix} = 1 \cdot 2 \dots \cdot n = n!
$$
\subsection*{II}
Odejmijmy od wszystkich wierszy pierwszy wiersz:
$$
\begin{vmatrix}
1 & 1 & \dots & 1 & -n\\
1 & 1 & \dots & -n & 1\\
\vdots & \vdots & \quad & \vdots &\vdots\\
1 & -n & \dots & 1 & 1\\
-n & 1 & \dots & 1 & 1\\
\end{vmatrix} = 
\begin{vmatrix}
1 & 1 & \dots & 1 & -n\\
0 & 0 & \dots & -(n+1) & n+1\\
\vdots & \vdots & \quad & \vdots &\vdots\\
0 & -(n+1) & \dots & 0 & n + 1\\
-(n+1) & 0 & \dots & 0 & n + 1\\
\end{vmatrix} = 
$$
Dodajemy do ostatniego pierwszy wiersz pomnożony przez $n+1$:
$$
=
\begin{vmatrix}
1 & 1 & \dots & 1 & -n\\
0 & 0 & \dots & -(n+1) & n+1\\
\vdots & \vdots & \quad & \vdots &\vdots\\
0 & -(n+1) & \dots & 0 & n + 1\\
0 & n+1 & \dots & n+1 &  -(n + 1)(n-1)\\
\end{vmatrix} = 
$$
Zamieniamy wiersze w środku kolejnością:
$$
={(-1)}^{\lfloor{\frac{n-2}{2}} \rfloor}
\begin{vmatrix}
1 & 1 & \dots & 1 & -n\\
0 & -(n+1) & \dots & 0 & n + 1\\
\vdots & \vdots & \quad & \vdots &\vdots\\
0 & 0 & \dots & -(n+1) & n+1\\
0 & n+1 & \dots & n+1 &  -(n + 1)(n-1)\\
\end{vmatrix}=
$$
Odejmujemy od ostatniego wiersza każdy oprócz pierwszego wiersz i ostatecznie otrzymujemy:
$$
={(-1)}^{\lfloor{\frac{n-2}{2}} \rfloor}
\begin{vmatrix}
1 & 1 & \dots & 1 & -n\\
0 & -(n+1) & \dots & 0 & n + 1\\
\vdots & \vdots & \quad & \vdots &\vdots\\
0 & 0 & \dots & -(n+1) & n+1\\
0 & 0 & \dots & 0 &  -(n + 1)\\
\end{vmatrix}={(-1)}^{\lfloor{\frac{n-2}{2}} \rfloor}{(-1)}^{n-1}(n+1)^{n-1}
$$
\section{zad. 8 (2 punkty)}
\subsection*{I}
Z rozwinięcia Laplace'a pierwszej kolumny:
$$
\begin{vmatrix}
x & y & 0 & \dots & 0 & 0 \\
0 & x & y & \dots & 0 & 0 \\
0 & 0 & x & \dots & 0 & 0 \\
\vdots & \vdots & \vdots & \quad &\vdots &\vdots \\
0 & 0 & 0 & \dots & x & y \\
x & 0 & 0 & \dots & 0 & y \\
\end{vmatrix}
=x
\begin{vmatrix}
 x & y & \dots & 0 & 0 \\
 0 & x & \dots & 0 & 0 \\
 \vdots & \vdots & \quad &\vdots &\vdots \\
 0 & 0 & \dots & x & y \\
 0 & 0 & \dots & 0 & y \\
\end{vmatrix}
+ {(-1)}^n x
\begin{vmatrix}
 y & 0 & \dots & 0 & 0 \\
 x & y & \dots & 0 & 0 \\
 0 & x & \dots & 0 & 0 \\
 \vdots & \vdots & \quad &\vdots &\vdots \\
0 & 0 & \dots & x & y \\
\end{vmatrix}=
$$
$$
= x^{n-2} + {(-1)}^{n+1}+ x\cdot y^{n-1}
$$

\subsection*{II}
Odejmujemy od każdego wiersza pierwszy:
$$
\begin{vmatrix}
a_1 + x & a_2 & \dots & a_n \\
a_1  & a_2 +x & \dots & a_n \\
\vdots  & \vdots & \quad & \vdots \\
a_1  & a_2  & \dots & a_n + x \\
\end{vmatrix}=
\begin{vmatrix}
a_1 + x & a_2 & \dots & a_n \\
-x & x & \dots & 0 \\
\vdots  & \vdots & \quad & \vdots \\
-x & 0  & \dots &  x \\
\end{vmatrix}=
$$
Skorzystamy z rozwinięcia Laplace'a względem pierwszej kolumny:
$$
(a_1 + x)x^{n-1} + (-x)a_2x^{n-2} + a_2\cdot 0 \dots = a_1x^{n-1} + x^n - a_2x^{n-2}
$$

\section{zad. 13 (2 punkty)}
Pamiętając wzór:
$$
\sum_{\sigma \in S_n} {\varepsilon}_{\sigma} a_{1,{\sigma}_1} a_{2,{\sigma}_2} \dots a_{n,{\sigma}_n}
$$
Skoro wszystkie wyrazy $A$ są całkowite to dodowolna suma sum ich też jest całkowita.\\
Mamy: $det(A) \in \mathbb{Z}$\\
podobnie: $det(A^{-1}) \in \mathbb{Z}$
Z multyplikatywności wyznacznika:
$$
1 = det(E) = det(AA^{-1}) = det(A)det(A^{-1})
$$
iloczyn dwóch liczb całkowitych jest równy jeden wtw. gdy te liczby to $\pm 1$.\\
$$
det(A) \in \{-1,1\}
$$
\\\\
\section{zad. 13 (2 punkty)}
\begin{proof}
Pamiętając wzór:
$$
\sum_{\sigma \in S_n} {\varepsilon}_{\sigma} a_{1,{\sigma}_1} a_{2,{\sigma}_2} \dots a_{n,{\sigma}_n}
$$
Sumujemy po wszystkich permutacjach więc napewno wystąpi czynnik:
$$
{\varepsilon}_{\sigma} a_{1,1} a_{2,2} \dots a_{n,n}
$$
Wyznacznik konkretnie macierzy $A-xE$ zawiera więc składnik:
$$
\pm (a_{1,1}-x)(a_{2,2}-x) \dots (a_{n,n}-x)
$$
zatem:
$$
det(A-xE) = \pm x^n + Q(x)
$$
\end{proof}
\end{document}