\documentclass{article}
\usepackage[utf8x]{inputenc}
\usepackage{polski}
\usepackage{pythonhighlight}

\usepackage{amssymb, amsmath, amsfonts, amsthm, cite, mathtools, enumerate, rotating, hyperref,soul}
\newcommand \eq[1]{\begin{equation} \begin{split}  #1 \end{split} \end{equation}}


\makeatletter
\newcommand\tab[1][1cm]{\hspace*{#1}}
\def\@seccntformat#1{%
  \expandafter\ifx\csname c@#1\endcsname\c@section\else
  \csname the#1\endcsname\quad
  \fi}
\makeatother

\newtheorem{lemma}{Lemat}
\newtheorem{theorem}{Twierdzenie}

\begin{document}
\subsection*{Rozdzielone zmienne}
$$
y^\prime(t) = f(t)g(y)
$$
dzielimy przez $g$ i całkujemy (rozdzielamy $y,t$):
$$
\int_{y(t_0)}^{y(t)}\frac{1}{g(z)}dz = \int_{t_0}^t f(s)ds
$$
\subsection*{Liniowe I-szego rzędu}
$$
y^\prime(t) + p(t)y(t) = q(t)
$$
$q(t) = 0$ to jednorodne\\
mnożymy przez czynnik całkujący $e^{P(t)}$ i zwija się do pochodnej:
$$
\left( y(t)e^{P(t)}\right)^\prime = q(t)e^{P(t)} \quad \text{(lub zero)}
$$
$$
y(t) = e^{-P(t)} \left (\int_{t_0}^t q(s)e^{P(s)}ds + y_0e^{P(t_0)}\right)
$$
\subsection*{Równanie zupełne}
$$
M(t,y) + N(t,y)y^\prime = 0
$$
tż. $\frac{\partial}{\partial y}M(t,y) = \frac{\partial}{\partial t} N(t,y)$ można zapisać alternatywnie z $y^\prime = \frac{dy}{dt}$\\
Szukamy różniczki zupełnej $\varphi$:
$$M(t,y) = \frac{\partial}{\partial t} \varphi(t,y)$$
$$N(t,y) = \frac{\partial}{\partial y} \varphi(t,y)$$
Z jednego całkujemy ($\varphi(t,y) = \int M(t,y)\textbf{dt}$) zamiast stałej ($h(y)$) i liczymy $\varphi$ do drugiego podstawiamy, różniczkujemy i mamy stałą  ($h(y)$) z $N$
$$\frac{\frac{\partial M}{\partial y} - \frac{\partial N}{\partial t}}{N} = \phi(t)$$
Jeżeli powyższe gówno jest zależne tylko od $t$ to $e^{\int \phi(t)}$ jest czynnikiem całkującym (uzupełnia do zupełnego). Jak poniższe gówno od tylko $y$ to $e^{\int \psi(y)}$:
$$\frac{\frac{\partial N}{\partial t} - \frac{\partial M}{\partial y}}{M} = \psi(y)$$
Inne dziwne gówna:
\begin{enumerate}
\item Bernoulliego $\frac{dy}{dt} + p(t)y(t) = q(t)+y^m(t)$. 
Mnożymy przez $(1-m)y^{-m}$ podstawiamy $z(t) = y^{1-m}(t)$ i mamy liniowe niejednorodne
\item Riccatiego $\frac{dy}{dt} + p(t)y(t) + q(t)y^2(t) = f(t)$
\end{enumerate}
\clearpage
\begin{enumerate}
    \item Malthus: $P^\prime(t) = aP(t) \rightarrow P(t)=P_0e^{a(t-t_0)}$ w chuj 
    \item Von Foerstera: $P^\prime(t) = aP^2(t) \rightarrow P(t)=P_0\frac{a}{1-aP_0t}$ w chuj ale w $T_{ks}$
    \item Benjamina Gompertza $P^\prime(t) = \lambda e^{\alpha t}P(t) \rightarrow P(t)=P_0e^{-\frac{\lambda}{\alpha}(1-e^{\alpha(t-t_0)})}$
    \item Verhalust $P(t)^\prime = bP(t)\left(\frac{a}{b} - P(t)\right)$ pochodna zmienia znak jak duzo $P$
    \item Ciało: $T^\prime(t) = k(T(t) - T_o) \land T(0) = T_0 \rightarrow T(t) = T_o + (T_0-T_o)e^{-kt}$ liniowe niejednordone, $o \neq 0$
\end{enumerate}
\subsection*{Peano - istnienie}
$y^\prime = f(t,y)$ ciągła na prostokącie: $R = \{(t,y); t_0 \leq t \leq t_0 +a, |y-y_0| \leq b\}$\\
$M = \max_{(t,y) \in R}|f(t,y)| \quad \alpha = \min \left\{a,\frac{b}{M}\right\}$ \\
istnieje \textbf{conajmniej} jedno rozwiąznie na $[t_0,t_0+\alpha]$
\subsection*{Picard-Lindelof - jedyność}
$y^\prime = f(t,y)$ i $\frac{\partial}{\partial y} f(t,y)$ ciągła na prostokącie:\\
$R = \{(t,y); t_0 \leq t \leq t_0 +a, |y-y_0| \leq b\}$ tak samo $M,\alpha$\\
istnieje \textbf{dokładnie} jedno rozwiąznie na $[t_0,t_0+\alpha]$ chcemy mieć, że $\alpha$ to dowolna zmienna, wtedy mamy $[t_0,\infty]$
\subsection*{Iteracja Picarda}
\begin{align*}
y_0(t) = y_0\\
y_1(t) = y_0 + \int_{t_0}^t f(s,y_0(s))ds\\
\vdots\\
y_{n+1}(t) = y_0 + \int_{t_0}^t f(s,y_n(s))ds\\
\end{align*}
$y_n$ w granicy to rozwiązanie
\subsection*{Lemat Gronwalla}
$U(t)$ nieujemna funkcja:
$$
u(t) \leq a+ b\int_{t_0}^t u(s)ds
$$
dla $a,b \geq 0$, wtedy:
$$
u(t) \leq ae^{b(t-t_0)}
$$
\end{document}