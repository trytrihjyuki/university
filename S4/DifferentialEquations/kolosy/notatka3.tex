\documentclass{article}
\usepackage[utf8x]{inputenc}
\usepackage{polski}
\usepackage{pythonhighlight}

\usepackage{amssymb, amsmath, amsfonts, amsthm, cite, mathtools, enumerate, rotating, hyperref,soul}
\newcommand \eq[1]{\begin{equation} \begin{split}  #1 \end{split} \end{equation}}
\usepackage[top=1.1cm, bottom=1.1cm]{geometry}
\usepackage[normalem]{ulem}
\makeatletter
\newcommand\tab[1][1cm]{\hspace*{#1}}
\def\@seccntformat#1{%
  \expandafter\ifx\csname c@#1\endcsname\c@section\else
  \csname the#1\endcsname\quad
  \fi}
\makeatother

\newtheorem{lemma}{Lemat}
\newtheorem{theorem}{Twierdzenie}
\begin{document}
\textbf{Równanie transportu}\\
$a\frac{\partial}{\partial x}u(x,y) + b \frac{\partial}{\partial y} u(x,y) = 0$, $ay-bx=C$ charakterystki, stałe rozwy\\
$u(x,y) = f(C) = f(ay-bx)$ dla pewnej $f$ się liczy z war. pocz.\\
Istnienie, czy na wszystkich charakterystykach jest stałe?\\\\
$a(x,y)\frac{\partial}{\partial x}u(x,y) + b(x,y) \frac{\partial}{\partial y} u(x,y) = f(u)$,\\
charakterystyki rozwy: $\frac{dy}{dx} = \frac{b(x,y)}{a(x,y)}$ szukamy $C$ podstawiamy $f(C) = f(guwno)$ z war. pocz. $f$\\\\
\textbf{Równanie struny}\\
$u_{tt}(x,t) = c^2 \Delta u(x,t)$\quad$\Delta=$suma drugich pochodnych\\
Wzór d'Alemberta: $$u(x,t)=\frac{1}{2}(\phi(x+ct)+\phi(x-ct)) +\frac{1}{2c}\int_{x-ct}^{x+ct}\psi(y)dy$$
gdzie: $u(x,0) = \phi(x)$ oraz $u_t(x,0) = \psi(x)$
\end{document}