\documentclass{article}
\usepackage[utf8x]{inputenc}
\usepackage{polski}


\usepackage{amssymb, amsmath, amsfonts, amsthm, cite, mathtools, enumerate, rotating, hyperref}
\newcommand \eq[1]{\begin{equation} \begin{split}  #1 \end{split} \end{equation}}


\makeatletter
\newcommand\tab[1][1cm]{\hspace*{#1}}
\def\@seccntformat#1{%
  \expandafter\ifx\csname c@#1\endcsname\c@section\else
  \csname the#1\endcsname\quad
  \fi}
\makeatother

\title{Bazy Danych L1}
\date{7.03.2021}
\author{Maurycy Borkowski}
\begin{document}
\maketitle

\section{zad. 1}
Zauważamy, że tylko operator $\sigma$ zmniejsza liczbę rekordów. Więc jeżeli dałoby się zapisać $\setminus$ to musimy użyć $\sigma$. Załóżmy, że istniałaby formuła na operator różnicy i używa ona porównań do $n$ elementów, wtedy dla zbiorów $A, B$ tż. $|A \setminus B| = n+1$ jednego elementu nie porównamy.
\section{zad. 2}
Nie jest to poprawne zapytanie, w przypadku $X=Z \neq \emptyset$ oraz $Z =\emptyset$ zapytanie zwróci pustą relację a powinno zwrócić $X = Z$, nie można go poprawić nie używając symbolu $\cup$
\section{zad. 3}
\subsection*{a)}
Bary, w których ktoś bywa i podają conajmniej jeden sok.
\subsection*{b)}
$$\pi_{osoba, bar} \left (\sigma_{n \geq 5} G_{count_{sok}(osoba,bar)}(\Pi_{osoba,sok,bar}(P \bowtie L))\right)$$
\subsection*{c)}
$$
G_{min_{cena}(osoba,sok)}(\pi_{osoba,sok,cena}B \bowtie P)
$$
\subsection*{d)}
$$
\pi_{osoba,sok,bar} \left(P \bowtie G_{min_{cena}(osoba,sok)}(\pi_{osoba,sok,cena} B \bowtie P) \right)
$$
\section{zad. 4}
W innym pliku.
\end{document}

