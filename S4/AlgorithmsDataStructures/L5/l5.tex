\documentclass{article}
\usepackage[utf8x]{inputenc}
\usepackage{polski}
\usepackage{pythonhighlight}

\usepackage{amssymb, amsmath, amsfonts, amsthm, cite, mathtools, enumerate, rotating, hyperref,soul}
\newcommand \eq[1]{\begin{equation} \begin{split}  #1 \end{split} \end{equation}}


\makeatletter
\newcommand\tab[1][1cm]{\hspace*{#1}}
\def\@seccntformat#1{%
  \expandafter\ifx\csname c@#1\endcsname\c@section\else
  \csname the#1\endcsname\quad
  \fi}
\makeatother

\newtheorem{lemma}{Lemat}
\newtheorem{theorem}{Twierdzenie}

\title{AiSD L5}
\date{12.05.2021}
\author{Maurycy Borkowski}
\begin{document}
\maketitle
\section{zadanie 1.}
Spisane na kartce.
\section{zadanie 2.}
Spisane na kartce.
\clearpage
\section{zadanie 3.}
Ile mamy hiperpłaszczyzn z odpowiedzią "TAK"?\\
Wszędzie indziej jest odpowiedź "NIE".
$$
\binom{n}{3} = \frac{n!}{(n-3)!\cdot3!} = \Theta(n^3)
$$
Na ile obszarów dzieli przestrzeń $n$-wymiarową, $m$-hiperpłaszczyzn (\href{https://math.stackexchange.com/questions/409518/how-many-resulting-regions-if-we-partition-mathbbrm-with-n-hyperplanes}{dowód)}?
$$
\sum_{i=0}^{n} \binom{m}{i}
$$
Oszacujemy:
$$
\binom{m}{i} = \frac{m!}{(m-i)!\cdot i!} \leq \frac{m^i}{i!}\leq \frac{m^i}{i}
$$
Stąd:
$$
\sum_{i=0}^{n} \binom{m}{i} \leq \sum_{i=0}^{n} \frac{m^i}{i} \leq n\cdot \frac{m^n}{n} = m^n
$$
Mamy więc co najwyżej $m^n$ obszarów.\\\\
Stąd, nie możemy podzielić przestrzeni wejść na więcej niż\\ $\Theta(n^3)^n = 2^{\Theta(n\log{n})}$ maxymalnych obszarów (liści) z tą samą odpowiedzią "NIE".


\clearpage
\section{zadanie 4.}
Adwersarz przygotowuje sobie $2n$ ciągów w następujący sposób (w każdym ciągu zmienia dokładnie jedną parę sąsiednich elementów):
\begin{align*}
A_0 = a_1,b_1,a_2,b_2,a_3,b_3,\ldots\\
A_1 = b_1,a_1,a_2,b_2,a_3,b_3,\ldots\\
A_2 = a_1,a_2,b_1,b_2,a_3,b_3,\ldots\\
A_3 = a_1,b_1,b_2,a_2,a_3,b_3,\ldots\\
A_4 = a_1,b_1,a_2,a_3,b_2,b_3,\ldots\\
\ldots
\end{align*}
Teraz pokażemy, jak adwersarz może odpowiadać tak, by za każdym razem odrzucał co najwyżej jeden z ciągów. Oczywiście nie będziemy zadawać \textit{głupich} pytań tj. o relacje wewnątrz $a_i$ lub $b_j$ tylko pomiędzy nimi. Rozpatrzmy przypadki:
\begin{enumerate}
    \item $i > j+1 $\\
    w każdym z $2n$ ciągów możemy odpowiedzieć, że $a_i > b_j$ nie odrzucając żadnego z nich
    \item $i < j $\\
    w każdym z $2n$ ciągów możemy odpowiedzieć, że $a_i < b_j$ nie odrzucając żadnego z nich
    \item $i = j $\\
    w każdym oprócz jednego ciągu $a_i < b_i$, stąd adwersarz musi odrzucić jeden ciąg
    \item $i = j+1 $\\
    w każdym oprócz jednego ciągu $a_i > b_i$, stąd adwersarz musi odrzucić jeden ciąg
\end{enumerate}
Widzimy, więc, że każde zapytanie do adwersarza odrzuca co najwyżej jeden ciąg. Stąd musimy wykonać $2n -1$ zapytań (porównań).
\clearpage
\section{zadanie 6.}
\subsection*{Wystarcza}
Możemy zbudować sobie drzewo \textit{turniejowe}, aby wyłonić zwycięzce musimy wykonać $n-1$ porównań (liczba wierzchołków w drzewie binarnym bez liści). Aby wyłonić srebrnego medaliste, trzeba sprawdzić z wszystkich, którzy przegrali ze zwycięzcą. Ich jest: $\lceil{\log{n}}\rceil$ więc żeby wyłonić drugie miejsce musimy wykonać jeszcze $\lceil{\log{n}}\rceil - 1$ porównań.\\Ostatecznie wykonamy: $n + \lceil{\log{n}}\rceil - 2$ porównań.
\subsection*{Potrzeba}
Algorytm wyłania:
\begin{itemize}
    \item element największy, $a$
    \item element drugi największy, $b$
    \item zbiór pozostałych elementów, $R$ (oczywiście $|R|=n-2$)
\end{itemize}
Żeby stwierdzić czy dany element ma należeć do $R$ musimy wiedzieć, że jest on mniejszy od $b$ lub od innego elementu z $R$, stąd trzeba wykonać conajmniej $n-2$ porównania z elementami ze zbioru $\{b\} \cup R$.\\\\
Oznaczmy przez $L(x)$ liczbę elementów o których wiemy, że \textit{przegrała} z $x$.\\
Adwersarz, gdy porównujemy elementy $x,y$ wybierze ten gdzie daje nam mniej informacji:
$$
\begin{cases}
x<y \text{   ,gdy   } L(x) < L(y)\\
x>y \text{    w p.p.}
\end{cases}
$$
Wtedy zbiory elementów mniejszych będą się zmieniały następująco:
$$
\begin{cases}
L(y)=L(x)+L(y) \land L(x) = 0 \text{   ,gdy   } L(x) < L(y)\\
L(x)=L(x)+L(y) \land L(y) = 0 \text{    w p.p.}
\end{cases}
$$
W najgorszym przypadku (dla adwersarza) po udzieleniu odpowiedzi $L(x)$ zwiększy się dwukrotnie, stąd po $k$ porównaniach, $L(x) \leq 2^k$, pod koniec algorytmu $L(a) = n$ stąd:
$$
n = L(a) \leq 2^k \implies \lceil \log{n} \rceil \leq k
$$
czyli potrzeba jeszcze conajmniej $\lceil \log{n} \rceil$ (kroków=porównań z $a$).\\\\
Porównania potrzebne do wyłonienia $a$ ($\lceil{\log{n}}\rceil$) nie dają informacji nic o tym jak postąpić z resztą elementów. Są rozłączne z ($n-2$) porównaniami, tworzącymi $R$ te porównania stwierdzają mniejszość względem $b$ lub innego elementu z $R$.  Stąd te dwa zbiory porównań są rozłączne.\\\\
Ostatecznie potrzeba: $n + \lceil{\log{n}}\rceil - 2$ porównań.


\end{document}