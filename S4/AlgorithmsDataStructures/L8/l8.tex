\documentclass{article}
\usepackage[utf8x]{inputenc}
\usepackage{polski}
\usepackage{pythonhighlight}

\usepackage{amssymb, amsmath, amsfonts, amsthm, cite, mathtools, enumerate, rotating, hyperref,soul, graphicx}
\newcommand \eq[1]{\begin{equation} \begin{split}  #1 \end{split} \end{equation}}

\makeatletter
\newcommand\tab[1][1cm]{\hspace*{#1}}
\def\@seccntformat#1{%
  \expandafter\ifx\csname c@#1\endcsname\c@section\else
  \csname the#1\endcsname\quad
  \fi}
\makeatother

\newtheorem{lemma}{Lemat}
\newtheorem{theorem}{Twierdzenie}

\title{AiSD L8}
\date{16.06.2021}
\author{Maurycy Borkowski}
\begin{document}
\maketitle
\section{zadanie 1.}
Wszystkie operacje \textit{Union} wykonują się w czasie stałym. Każda operacja \textit{Union} tworzy jedną krawędź w strukturze drzewiastej.\\\\
Gdy wykonujemy \textit{Find} dokonujemy kompresji, każda z krawędzi, której przechodzimy zostaje bezpośrednio dołączona do korzenia. Każdą z krawędzi przejdziemy dokładnie raz (w nietrywialnym przejściu).\\\\
Stąd $\sigma$ instrukcji \textit{Union} będzie wymagało proporcjonalnie dużo operacji do stworzenia $\sigma$ krawędzi i przejścia co najwyżej $\sigma$ krawędzi (w nietrywialnych przejściach, każdą raz).
\section{zadanie 8.}
Wystarczy zastosować algorytm KMP na danych:\\
Wzorzec: $T$
Tekst: $T' + T'$ gdzie '$+$' oznacz konkatenacje napisów.\\\\
Złożoność czasowa, jak w KMP $\mathcal{O}(T+T')$
\end{document}