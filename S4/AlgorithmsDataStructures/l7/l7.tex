\documentclass{article}
\usepackage[utf8x]{inputenc}
\usepackage{polski}
\usepackage{pythonhighlight}

\usepackage{amssymb, amsmath, amsfonts, amsthm, cite, mathtools, enumerate, rotating, hyperref,soul, graphicx}
\newcommand \eq[1]{\begin{equation} \begin{split}  #1 \end{split} \end{equation}}

\makeatletter
\newcommand\tab[1][1cm]{\hspace*{#1}}
\def\@seccntformat#1{%
  \expandafter\ifx\csname c@#1\endcsname\c@section\else
  \csname the#1\endcsname\quad
  \fi}
\makeatother

\newtheorem{lemma}{Lemat}
\newtheorem{theorem}{Twierdzenie}

\title{AiSD L7}
\date{9.06.2021}
\author{Maurycy Borkowski}
\begin{document}
\maketitle
\section{zadanie 2.}
Dzielimy płaszczyznę na siatkę z kwadracików o boku $\frac d 2$. W każdym kwadraciku jest co najwyżej jeden punkt (założenie, że $d$ to najmniejsza odległość między sprocesowanymi już punktami).\\\\
Rozpatrując punkt $p_i$ musimy rozpatrzeć tylko $25 = 5^2$ kwadracików (duży kwadrat o boku $2d$) w nim znajduje się okrąg bez brzegu o promieniu $d$ = wszystkie miejsca gdzie może wystąpić bliższy punkt z procesowanym $p_i$.\\\\
Gdy żaden nie znajdziemy w podejrzanych $25$ kwadracikach zbyt bliskiego punktu, dodajemy do tablicy haszującej nowy zapełniony kwadracik z $p_i$. $\mathcal{O}(1)$\\\\
Gdy znajdziemy punkt bliższy w podejrzanych musimy stworzyć nową siatkę z mniejszymy kwadracikami o boku $d'$. Każdy punkt na nowo trzeba wstawić do pola siatki. $\mathcal{O}(i)$\\\\
Jeżeli punkty są przemieszane losowo (prawdopobieństwo bycia w najbliższej parze, każdy punkt ma takie samo) to prawdopodobieństwo tego, że $p_i$ jest w najbliższej parze wynosi: $\frac{i-1}{i(i-1)} = \frac{1}{i}$\\\\
Wystarczy więc spermutować punkty na początku żeby otrzymać oczekiwany czas działania:
$$
\mathbb{E}Czas = \sum_{i=0}^{n} \frac{i-1}{i}\mathcal{O}(1) + \frac{1}{i}\mathcal{O}(i) = \mathcal{O}(n)
$$

\clearpage
\section{zadanie 3.}
Oznaczmy:
$$
X_i=
\begin{cases}
1 \text{ gdy $i$-ta lista jest pusta}\\
0 \text{ w p.p.}
\end{cases}
$$
Wtedy liczba pustch list wynosi:
$$
Y = \sum_{i=1}^{n}X_i
$$
Wartość oczekiwana $X_i$ wynosi:
$$
\mathbb{E}X_i = \left(1-\frac{1}{n} \right)^n
$$
Prawdopodobieństwo, że za każdym z $n$ razy, nie wrzucimy do $i$-tej listy.\\
Z liniowości wartości oczekwianej:
$$
\mathbb{E}Y = \mathbb{E}\left(\sum_{i=1}^{n}X_i \right) = \sum_{i=1}^{n}\mathbb{E}X_i = \sum_{i=1}^{n} \left(1-\frac{1}{n} \right)^n= n \cdot \left(1-\frac{1}{n} \right)^n \leq \frac{n}{e} < \frac{n}{2}
$$
% Wniosek: Oczekwiana wartość kolizji dla każdej listy jest więc mniejsza niż $2$.
\end{document}