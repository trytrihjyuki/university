\documentclass{article}
\usepackage[utf8x]{inputenc}
\usepackage{polski}
\usepackage{pythonhighlight}

\usepackage{amssymb, amsmath, amsfonts, amsthm, cite, mathtools, enumerate, rotating, hyperref}
\newcommand \eq[1]{\begin{equation} \begin{split}  #1 \end{split} \end{equation}}


\makeatletter
\newcommand\tab[1][1cm]{\hspace*{#1}}
\def\@seccntformat#1{%
  \expandafter\ifx\csname c@#1\endcsname\c@section\else
  \csname the#1\endcsname\quad
  \fi}
\makeatother

\title{AiSD L1}
\date{10.03.2021}
\author{Maurycy Borkowski}
\begin{document}
\maketitle

\section{Zadanie 1}
\subsection*{Liczba wierzchołków T}
\begin{python}
G[] <- drzewo
def count(v):
    subtree_cnt = 1
    if G[v].L:
        subtree_cnt += count(G[v].L)
    if G[v].R:
        subtree_cnt += count(G[v].R)
    return subtree_cnt

count(root)
\end{python}
\subsection*{Maksymalna odległość T}
\begin{python}
ans = 0

def getHeight(v):
    if v.L is None and v.R is None:
        return 0
    height_left, height_right = 0, 0
    sons = 0
    if v.L:
        height_left = getHeight(v.L)
        sons += 1
    if v.R:
        height_right = getHeight(v.R)
        sons += 1
    ans = max(ans, height_left + height_right + sons)
    return max(height_left + 1, height_right + 1)

getHeight(root)
\end{python}
W obu algorytmach przechodzimy całe drzewo raz, $\mathcal{O}(n)$.
\section{Zadanie 3}
\begin{python}
def topoSort(edges, n):
    order = []
    inDegree = [0]*n
    G = []*n

    for (u,v) in edges:
        inDegree[v]+=1
        G[u].append(v)

    heap = [v if inDegree[v] == 0 for v in range(n)]

    while queue:
        u = heap.getMin()
        order.append(u)
        heap.delMin()
        for v in G[u]:
            inDegree[v] -= 1
            if inDegree[v] == 0:
                heap.add(v)
    return order
\end{python}
Dla dowolnej krawędzi $(u,v)$ mamy $inDegree[v] = 0 \implies inDegree[u] = 0$\\
ponieważ, $inDegree[v] = 0$ tylko jak przeprocesujemy wszystkie wierzchołki z krawędziami wchodzącymi do $v$, w tym $(u,v)$ więc $u \prec v$ w zadanym porządku.\\\\
Mając do wyboru kilka wierzchołków w kolejce wybieramy najmniejszy by otrzymać porządek pierwszy leksykograficznie.\\\\
$\mathcal{O}(n\log{n})$
\clearpage
\section{Zadanie 4}
\begin{python}
countWays = [0]*n
visited = [0]*n
distances = [0]*n
G = [[]]*n


def prep(u, v):
    countWays[v] = 1
    visited[v] = 1

    dijsktra(distances, v)
    dfs(u)

    return countWays[u]


def dfs(w):
    visited[w] = True
    for son in G[w]:
        if dist[son] < dist[w]:
            if visited[son] != true:
                dfs(son)
            countWays[w] += countWays[son]
    return
\end{python}
W preprocessingu znajdujemy najkrótsze odległości dla wszystkich wierzchołków od $v$ algorytmem Dijkstr'y.\\\\
$countWays[w]$ oznacza liczbę sensownych dróg wychodzących z $w$ do $v$\\
Ścieżka będzie zliczana do senswonych w $countWays[]$ tylko wtedy, gdy głębiej w poddrzewie DFS dojdzie do $v$.\\\\
Liczba sensownych dróg z każdego wierzchołka to suma liczby senswonych dróg jego \textbf{sensownych synów}, to znaczy takich, którzy mogą być następnikami w sensownej ścieżce, $dist[son] < dist[father]$.\\\\
$\mathcal{O}(m\log{n})$
\clearpage
\section{Zadanie 5}
\begin{python}
def findPath(n):
    orderedVertices = topoSort(G)
    distances = [0]*(n+1)
    parent = [0]*(n+1)

    for v in orderedVertices:
       for u in G[v]:
          if distances[u] < distances[v] + 1:
             distances[u] = distances[v] + 1
             parent[u] = v

    def generatePath(v):
       result = [v]
       while parent[v] != 0:
         result.append(parent[v])
         v = parent[v]

       return reversed(result)

    bestVertex = argmax(distances)

    return generatePath(bestVertex)
\end{python}
Zauważmy, że dla dowolnej drogi w $G$ wierzchołki na tej drodze będą leżały zgodnie z porządkiem topologicznym (jest to DAG). \\\\Dla dowolnej drogi $v_0 \ldots v_k$ najpierw sprocesujemy wszystkich rodziców każdego wierzchołka $v_i$ i będziemy znali najdłuższe drogi kończące się w nich, a biorąc wartość największą z tego (i dodając $1$) otrzymamy najdłuższą ścieżkę zakończoną w $v_i$\\\\
By odwtorzyć ścieżkę wystarczy pamiętać przodka w najdłuższej ścieżce do danego wierzchołka.\\\\
$\mathcal{O}(n + m)$
\clearpage
\section{Zadanie 6}
Oznaczenie:\\
$L$ - mniejsze elementy usuwane\\
$R$ - większe elementy usuwane
\subsection*{Lemat}
Istnieje rozwiązanie optymalne takie, że:
$$L = \{a_1, \ldots, a_k\}$$
$$R = \{a_{n-k+1}, \ldots, a_n\}$$
oraz elementy ciągu usuwane są parami $(a_i, a_{n-k+i})$
\begin{proof}.
\subsection*{I \quad $\max(L) \leq \min(R)$}
Jeżeli istnieją $a_i \in L$ oraz $a_j \in R$ tż. $a_i > a_j$ pokażemy, że możemy je zamienić. Oznaczmy przez $a_n, a_m$ elementy usuwane odpowiednio z $a_i$ i $a_j$. Z treści:\\
$2a_i \leq a_n$ więc $2a_j < 2a_i \leq a_n$ z tego $(a_j, a_n)$ też jest poprawną parą do usunięcia.\\
Podobnie, $2a_m \leq a_j < a_i$ więc $(a_m, a_i)$ też jest poprawną parą do usunięcia.\\
Możemy tak zamieniać elementy (nienaruszając rozwiązania), aż spełnimy \textbf{I}.
\subsection*{II \quad $L = \{a_1 \ldots a_k\}$ i $R = \{a_{n-k+1}, \ldots, a_n\}$}
Załóżmy, że istnieje $a_i \notin L$ oraz $a_j \in L$ tż. $i < j$.
Zauważmy, że wtedy możemy usunąć element $a_i$ z elementem usuwanym z $a_j$ bo $a_i < a_j$.\\
Analogicznie pokazujemy dla $R$.
\subsection*{III \quad $(a_i, a_{n-k+i})$}
Jeżeli $a_1$ nie możemy usunąć z $a_{n-k+1}$ to żadnego innego elementu nie możemy z nim usunąć a wiemy, że z kimś jest usuwany. Indukcyjnie powtarzamy.
\end{proof}
Z lematu wiemy, że istnieje pewne rozwiązanie optymalne postaci:\\
$$
\underbrace{***********}_k\_\_\_\_\_\_\_\_\_\_\_\_\_\_\_\underbrace{***********}_k
$$
Jeśli znajdziemy pierwsze $a_1,\ldots,a_k$ elementów, które możemy usunąć z jakaś parą i stwierdzimy w 100\%, że dla $a_{k+1}$ nie znajdziemy elementu do usunięcia, to jest to rozwiązanie optymalne.
\begin{python}
ans = 0
j = n/2 + 1
while j <= n:
    if 2*a[i] <= a[j]:
        ans += 1
        i += 1
    j += 1
\end{python}
$\mathcal{O}(n)$
\section{Zadanie 7}
\begin{python}
def solve(edges, n, V):
    D = []
    dist = [[INF]*n]*n
    for (u,v,c) in edges:
        dist[u][v] = c
        dist[v][u] = c
    for i in range(n):
        dist[i][i] = 0

    def sumuj(u):
        suma = 0
        for i in range(u+1, n):
            for j in range(i+1,n):
                suma += dist[i][j]
        return suma

    for u in range(n,1,-1):
        if u <= k:
            D.append(sumuj(u))
        for i in range(n):
            for j in range(n):
                if dist[i][j] > dist[i][V[u]] + dist[V[u]][j]:
                    dist[i][j] = dist[i][V[u]] + dist[V[u]][j]

    D.append(sumuj(0))
    return reversed(D)
\end{python}
Wykonujemy tak na prawdę Algorytm Floyda-Warshalla z tym, że w zewnętrznej pętli najpierw procesujemy wierzchołki $v_{k+1} \ldots v_n$ (w dowolnej kolejności), a później dokładamy kolejno wierzchołki $v_k, v_{k-1} \ldots v_1$ z ich krawędziami i patrzymy czy możemy uzsykać lepsze najkrótsze ściezki przechodząc przez nowo dodany wierzchołek.\\
Na bieżąco liczymy wynikowe $D_i$\\\\
$\mathcal{O}(n^3)$
\end{document}

