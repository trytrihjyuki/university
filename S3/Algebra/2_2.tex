\documentclass{article}
\usepackage[utf8]{inputenc}
\usepackage{polski}


\usepackage{amssymb, amsmath, amsfonts, amsthm, cite, mathtools, enumerate, rotating, hyperref}
\newcommand \eq[1]{\begin{equation} \begin{split}  #1 \end{split} \end{equation}}


\makeatletter
\newcommand\tab[1][1cm]{\hspace*{#1}}
\def\@seccntformat#1{%
  \expandafter\ifx\csname c@#1\endcsname\c@section\else
  \csname the#1\endcsname\quad
  \fi}
\makeatother

\title{}
\date{20.10.2020}
\author{Maurycy Borkowski}
\begin{document}
\maketitle

\section{L2Z2}
Założenie: $p \nmid n$\\
\begin{proof}
Weźmy dowolny element grupy: $n \in \mathbb{Z}_p^* = (\{1,2,\dots,p-1\},*)$\\\\
$n$ spełnia założenie $p \nmid n$ bo $n < p$\\
Oznaczmy rząd tego elementu: $k = ord(n)$\\
z definicji rządu:
$$
n^{k} = e_{\mathbb{Z}_p^*} = 1
$$
Z Tw. Lagrange'a: w grupie skończonej rząd dowolnej jej podgrupy jest dzielnikiem rzędu grupy, mamy więc:
$$
k \mid p-1
$$
więc dla pewnej $m \in \mathbb{Z}$:
$$
p-1 = km
$$
dalej:
$$
n^{p-1}  \equiv n^{km}  \equiv(n^k)^m  \equiv(1)^m \equiv 1
$$
$$
n^{p-1} \equiv 1
$$
$$
n^{p-1} - 1 \equiv 0
$$
$$
p \mid n^{p-1}-1
$$
\end{proof}

















a\\\\\\\\\\\\\\\\\\\\\\\\\\\\\\\
Wzór: $\frac{n^3+5n+6}{6}$\\\\
\textbf{*} wiemy, że $N$ prostych może podzielić płaszczyznę na maksymalnie $\frac{n^2 + n + 2}{2}$ obszarów.\\
\begin{proof}
\subsection*{\\Podstawa}
Jedną płaszczyzną możemy podzielić przestrzeń na maksymalnie:
$$
\frac{1+5+6}{2} = 2
$$
\subsection*{Krok}
Załóżmy, że przestrzeń możemy podzielić $n$ płaszczyznami na $\frac{n^3+5n+6}{6}$ części. Jednym przecięciem płaszczyzną możemy maksymalnie dodać tyle kawałków przestrzeni ile możemy podzielić jedną płaszczyznę prostymi:
$$
\frac{n^3+5n+6}{6} + \frac{n^2 + n + 2}{2} = \frac{n^3 + 3n^2 + 8n + 12}{6} = \frac{(n+1)^3 + 5(n+1) + 6}{6}
$$
\end{proof}
\end{document}

