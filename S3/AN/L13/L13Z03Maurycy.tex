\documentclass{article}
\usepackage[utf8]{inputenc}
\usepackage{polski}


\usepackage{amssymb, amsmath, amsfonts, amsthm, cite, mathtools, enumerate, rotating, hyperref}
\newcommand \eq[1]{\begin{equation} \begin{split}  #1 \end{split} \end{equation}}

\DeclarePairedDelimiter{\norm}{\lVert}{\rVert} 
\makeatletter
\newcommand\tab[1][1cm]{\hspace*{#1}}
\def\@seccntformat#1{%
  \expandafter\ifx\csname c@#1\endcsname\c@section\else
  \csname the#1\endcsname\quad
  \fi}
\makeatother

\title{Zadanie 3}
\date{20.01.2021}
\begin{document}
\maketitle
Niech $x \in \mathbb{R}^n$ wtedy:
\subsection*{a)}
Oznaczmy $x_k = \max_{1\leq i \leq n}{x_i}$
$$
\norm{x}_\infty = x_k \leq x_k + \sum\limits_{\substack{i = 1 \\ i\neq k}}^n |x_i| = \sum_{i=1}^{n} |x_i| = \norm{x}_1 \leq  \sum_{i=1}^{n} |x_k| = n\norm{x}_\infty
$$
\subsection*{b)}
$$
\norm{x}_\infty = x_k = \sqrt{|x_k|^2} \leq \sqrt{|x_k|^2} = \sqrt{\sum_{i=1}^{n} |x_i|^2} = \norm{x}_2 \leq \sqrt{\sum_{i=1}^{n} |x_k|^2} = \sqrt{n\cdot|x_k|^2} =\sqrt{n}|x_k| = \sqrt{n}\norm{x}_\infty
$$
\subsection*{c)}
Z nierówności Cauchy'ego-Schwarz'a:
$$
\Vert x\Vert_1=
\sum\limits_{i=1}^n|x_i|=
\sum\limits_{i=1}^n|x_i|\cdot 1\leq
\sqrt{\sum\limits_{i=1}^n|x_i|^2}\sqrt{\sum\limits_{i=1}^n 1^2}=
\sqrt{n}\Vert x\Vert_2
$$
Zauważmy, że:
$$
\norm{x}_2^2 = \sum_{i=1}^{n} |x_k|^2 \leq \sum_{i=1}^{n} |x_k|^2 + 2 \cdot \sum\limits_{\substack{i,j= 1 \\ i\neq j}}^n |x_i| = \norm{x}_1^2
$$
to implikuje: $\norm{x}_2 \leq \norm{x}_1$.\\\\
Wobec powyższego:
$$
\frac{1}{\sqrt{n}} \norm{x}_1  \leq\norm{x}_2 \leq \norm{x}_1
$$
\end{document}
