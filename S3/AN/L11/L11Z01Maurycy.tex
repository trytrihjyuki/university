\documentclass{article}
\usepackage[utf8]{inputenc}
\usepackage{polski}


\usepackage{amssymb, amsmath, amsfonts, amsthm, cite, mathtools, enumerate, rotating, hyperref}
\newcommand \eq[1]{\begin{equation} \begin{split}  #1 \end{split} \end{equation}}


\makeatletter
\newcommand\tab[1][1cm]{\hspace*{#1}}
\def\@seccntformat#1{%
  \expandafter\ifx\csname c@#1\endcsname\c@section\else
  \csname the#1\endcsname\quad
  \fi}
\makeatother

\title{Zadanie 1}
\date{6.01.2021}
\begin{document}
\begin{proof}
Z własności wielomianów Czebyszewa punkty ekstremalne $u_k$ są dane wzorem: 

\begin{equation}
u_k = \cos{\frac{k\pi}{n}} \quad k \in [0,\ldots,n]
\end{equation}

Korzystamy z postaci trygonometrycznej $T_n(x)$:

\begin{equation}
T_n = \cos{(n \cos{x}^{-1})}
\end{equation}

W punktach $u_k$ mamy więc:

\begin{equation}
T_{n+j}(u_k) =  \cos{((n+j) \cos{\cos^{-1}{\frac{k\pi}{n}}})} = \cos{(n+j)\frac{k\pi}{n}}
\end{equation}

Korzystając ze wzoru na $\cos$ sumy:
$$
T_{n+j}(u_k) = \cos{\frac{nk\pi}{n}}\cos{\frac{jk\pi}{n}} -  \sin{\frac{nk\pi}{n}}\sin{\frac{jk\pi}{n}} 
$$
Zauważamy $\sin{\frac{nk\pi}{n}} = 0$ oraz rozpisujemy argumenty $\cos$:
$$
T_{n+j}(u_k) = \cos{\frac{nk\pi}{n}}\cos{\frac{jk\pi}{n}} = \cos{n \cos^{-1}{\frac{k\pi}{n}}} \cdot  \cos{j \cos^{-1}{\frac{k\pi}{n}}} = 
$$
$$
= T_n(u_k) \cdot T_j(u_k)
$$
\end{proof}
\end{document}
