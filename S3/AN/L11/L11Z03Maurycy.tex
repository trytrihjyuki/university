\documentclass{article}
\usepackage[utf8]{inputenc}
\usepackage{polski}


\usepackage{amssymb, amsmath, amsfonts, amsthm, cite, mathtools, enumerate, rotating, hyperref}
\newcommand \eq[1]{\begin{equation} \begin{split}  #1 \end{split} \end{equation}}


\makeatletter
\newcommand\tab[1][1cm]{\hspace*{#1}}
\def\@seccntformat#1{%
  \expandafter\ifx\csname c@#1\endcsname\c@section\else
  \csname the#1\endcsname\quad
  \fi}
\makeatother

\title{}
\date{6.01.2021}

\begin{document}
\subsection*{a)}
Korzystamy ze wzoru na błąd w metodzie Simpsona:
\begin{equation}
E_S = \frac{h^4}{180}(b-a)f^{(4)}(\xi_S)
\end{equation}
W naszym przypadku:
$$
\frac{\pi^5}{180N^4} \leq 2\cdot 10^{-5}
$$
Znajdujemy komputerowo rozwiązanie nierówności $N = 18$\\
Korzystamy ze wzoru Simpsona:
$$
\int\limits_{0}^{\pi}\sin{x}dx \approx 
\sum_{j=1}^{n/2}\bigg[\sin(x_{2j-2})+4\sin(x_{2j-1})+\sin(x_{2j})\bigg]\\
{} =
$$
$$= \frac{\pi}{3\cdot18} 
\bigg[2\sum_{j=1}^{n/2-1}\sin(x_{2j}) + 4\sum_{j=1}^{n/2}\sin(x_{2j-1})\bigg] =
$$
$$
= \frac{\pi}{3\cdot18} 
\bigg[2\sum_{j=1}^{8}\sin(2j \cdot \frac{\pi}{18}) + 4\sum_{j=1}^{9}\sin((2j-1) \cdot \frac{\pi}{18})\bigg] \approx 2.00001
$$
\subsection*{b)}
Korzystamy ze wzoru na błąd w metodzie trapezów :
\begin{equation}
E_T = \frac{(b-a)^3}{12N^3}f^{(2)}(\xi_T)
\end{equation}
W naszym przypadku:
$$
\frac{\pi^3}{12N^2} \leq 2\cdot 10^{-5}
$$
Znajdujemy komputerowo rozwiązanie nierówności $N = 360$ 
\end{document}
