\documentclass{article}
\usepackage[utf8]{inputenc}
\usepackage{polski}


\usepackage{amssymb, amsmath, amsfonts, amsthm, cite, mathtools, enumerate, rotating, hyperref}
\newcommand \eq[1]{\begin{equation} \begin{split}  #1 \end{split} \end{equation}}


\makeatletter
\newcommand\tab[1][1cm]{\hspace*{#1}}
\def\@seccntformat#1{%
  \expandafter\ifx\csname c@#1\endcsname\c@section\else
  \csname the#1\endcsname\quad
  \fi}
\makeatother

\title{}
\date{18.11.2020}
\author{}
\begin{document}
\maketitle

\section{L6Z7}
$s$ - naturalna funkcja sklejana trzeciego stopnia\\\\
$f(x_i) = s(x_i)$ dla $a = x_0 < x_1 < \ldots < x_n = b$
\begin{proof}
Całkując przez części
\begin{align*}
&\int_a^b[s^{\prime\prime}(x)]^2 dx = \int_a^b s^{\prime\prime}(x)\cdot s^{\prime\prime}(x) dx = s^{\prime\prime} (x) \cdot s^{\prime}(x) \mid_{[a,b]} - \int_a^b s^{\prime}(x)\cdot s^{\prime\prime\prime}(x) dx  \\
 \intertext{Dzielimy przedział całkowania: $[a,b] = \bigcup_{i=0}^{n-1} [x_i,x_{i+1}]$} 
&= \sum_{i=0}^{n-1} \left ( s^{\prime\prime} (x) \cdot s^{\prime}(x) \mid_{[x_i,x_{i+1}]} - \int_{x_i}^{x_{i+1}} s^{\prime}(x_k)\cdot s^{\prime\prime\prime}(x) dx\right) \\
&= \sum_{i=0}^{n-1} \left ( s^{\prime\prime} (a) \cdot s^{\prime}(a) - s^{\prime\prime} (b) \cdot s^{\prime}(b)  - \int_{x_i}^{x_{i+1}} s^{\prime}(x_k)\cdot s^{\prime\prime\prime}(x) dx\right)\\
 \intertext{Z naturalności $s$ oraz z tego, że $s^{\prime\prime\prime}(x) = \frac{M_{i+1} - M_i}{x_{i+1} -x_i}$ dla $x \in {x_i, x_{i+1}}$ (wzór ten można uzyskać różniczkując wzór (16) z wykładu:}
&= \sum_{i=0}^{n-1} \left (- \int_{x_i}^{x_{i+1}} s^{\prime}(x_k)\cdot s^{\prime\prime\prime}(x) dx\right)\\
&= \sum_{i=0}^{n-1}\frac{M_{i+1} - M_i}{x_{i+1} - x_i} \left(- \int_{x_i}^{x_{i+1}} s^{\prime}(x_k)\cdot dx\right)\\
&= \sum_{i=0}^{n-1} -\frac{M_{i+1} - M_i}{x_{i+1} - x_i}(s(x_{i+1}) - s(x_i))\\
 \intertext{$s$ interpoluje f w punktach $x_i$:}
&= \sum_{i=0}^{n-1} \frac{M_{i+1} - M_i}{x_{i+1} - x_i}(f(x_i) - f(x_{i+1}))\\
&= \sum_{i=0}^{n-1}  M_{i+1}\frac{f(x_i) - f(x_{i+1})}{{x_{i+1} - x_i}} - M_i\frac{f(x_i) - f(x_{i+1})}{{x_{i+1} - x_i}}\\
&= \sum_{i=1}^{n}  M_i\frac{f(x_{i-1}) - f(x_{i})}{{x_{i} - x_{i-1}}} - \sum_{i=0}^{n-1}  M_i\frac{f(x_{i}) - f(x_{i+1})}{{x_{i+1} - x_{i}}}\\
 \intertext{Z naturalności, oraz upraszczając wyrażenie mamy:}
&= \sum_{i=1}^{n-1}  M_i\frac{f(x_{i-1}) - f(x_{i})}{{x_{i} - x_{i-1}}} - M_i\frac{f(x_{i}) - f(x_{i+1})}{{x_{i+1} - x_{i}}}\\
\intertext{Przekształcamy (rozwijamy i wyciągamy minusa z mianownika):}
&= \sum_{i=1}^{n-1}\left(\frac{f(x_i)}{(x_i - x_{i+1})} + \frac{f(x_{i+1})}{(x_{i+1} - x_i)} - \frac{f(x_i)}{(x_i - x_{i-1})} - \frac{f(x_{i-1})}{(x_{i-1} - x_i)}\right)M_i
 \intertext{Zuważamy iloraz różnicowy, otrzymujemy ostatecznie:}
 &= \sum_{i=1}^{n-1}(f[x_i,x_{i+1}] - f[x_{i-1},x_i])M_i
\end{align*}
\end{proof}

\end{document}

