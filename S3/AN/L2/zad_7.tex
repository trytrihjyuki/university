\documentclass{article}
\usepackage[utf8]{inputenc}
\usepackage{polski}


\usepackage{amssymb, amsmath, amsfonts, amsthm, cite, mathtools, enumerate, rotating, hyperref}
\newcommand \eq[1]{\begin{equation} \begin{split}  #1 \end{split} \end{equation}}


\makeatletter
\newcommand\tab[1][1cm]{\hspace*{#1}}
\def\@seccntformat#1{%
  \expandafter\ifx\csname c@#1\endcsname\c@section\else
  \csname the#1\endcsname\quad
  \fi}
\makeatother

\title{}
\date{21.10.2020}
\author{Maurycy Borkowski}
\begin{document}
\maketitle

\section{L2Z7}
\subsection*{a)}
$$
f(x) = \frac{1}{x^2 + c}
$$
$$
\frac{df}{dx} = - \frac{2}{(c+x^2)^2}
$$
Liczymy ze wzoru współczynnik uwarunkowania:
$$
C_f(x) = \left| \frac{2x^2}{(c+x^2)^2}\right| \cdot |x^2 +c| =  \left |\frac{2x^2}{c+x^2} \right|
$$
Zauważamy, gdy:
$$
x^2 \rightarrow -c
$$
to:
$$
C_f(x) \rightarrow \infty
$$
\newline
\subsection*{b)}
$$
f(x) = \frac{1 - \cos{x}}{x^2}
$$
$$
\frac{df}{dx} = \frac{x\sin{x}+2\cos{x}-2}{x^3}
$$
Liczymy ze wzoru współczynnik uwarunkowania:
$$
C_f(x) = \left| \frac{x\sin{x}+2\cos{x}-2}{x^2}\right| \cdot \left| \frac{x^2}{1 - \cos{x}} \right| =  \left |\frac{x\sin{x}}{1-\cos{x}} - 2 \right|
$$
Zauważamy, gdy:
$$
x \rightarrow k \cdot 2\pi
$$
to:
$$
C_f(x) \rightarrow \infty
$$
\\ale:
$$
x \rightarrow 0
$$
to
$$
C_f(x) \not\to \infty
$$
ponieważ:\\\\
$\lim_{x \to 2\pi} \frac{\sin{x}}{1-\cos{x}} = \infty$ \\
\\\\\
$$
\lim_{x \to 0}\frac{x\sin{x}}{1-\cos{x}} = \lim_{x \to 0} \frac{x 2\sin{\frac{x}{2}}\cos{\frac{x}{2}}}{2{\sin{\frac{x}{2}}^2}} = \left(\lim_{x \to 0} \frac{x}{\sin{\frac{x}{2}}}\right) \cdot \left(\lim_{x \to 0} \cos{\frac{x}{2}}\right) = 2 \cdot 1 \cdot 1 = 2
$$


\end{document}

