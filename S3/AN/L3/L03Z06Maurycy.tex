\documentclass{article}
\usepackage[utf8]{inputenc}
\usepackage{polski}


\usepackage{amssymb, amsmath, amsfonts, amsthm, cite, mathtools, enumerate, rotating, hyperref}
\newcommand \eq[1]{\begin{equation} \begin{split}  #1 \end{split} \end{equation}}


\makeatletter
\newcommand\tab[1][1cm]{\hspace*{#1}}
\def\@seccntformat#1{%
  \expandafter\ifx\csname c@#1\endcsname\c@section\else
  \csname the#1\endcsname\quad
  \fi}
\makeatother

\title{}
\date{28.10.2020}

\begin{document}
Zapiszmy problem szukania miejsca zerowego $f(x)$ jako szukanie punktu stałego:
\begin{equation}
\Phi(\alpha) = \alpha \iff f(\alpha) = 0
\end{equation}
\begin{equation}
x_{n+1} = \Phi(x_n) = x_n - rf(x_n)
\end{equation}
Sprawdzamy warunki zbieżności liniowej:
\subsection*{Warunek 1}
Sprawdzamy czy $\Phi(\alpha) = \alpha$
\begin{equation}
\Phi(\alpha) = \alpha - rf(\alpha) = \alpha - 0 = \alpha
\end{equation}
\subsection*{Warunek 2}
Chcemy pokazać, że (jest zwężenie w otoczeniu $\alpha$), $0 < |\Phi^\prime(\alpha)| < 1$:
\begin{equation}
|\Phi^\prime(\alpha)| = 1 - rf^\prime(\alpha)
\end{equation}
więc musi być spełnione
\begin{equation}
0 < rf^\prime(\alpha) < 1
\end{equation}

\end{document}