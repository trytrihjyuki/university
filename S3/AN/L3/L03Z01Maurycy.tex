\documentclass{article}
\usepackage[utf8]{inputenc}
\usepackage{polski}


\usepackage{amssymb, amsmath, amsfonts, amsthm, cite, mathtools, enumerate, rotating, hyperref}
\newcommand \eq[1]{\begin{equation} \begin{split}  #1 \end{split} \end{equation}}


\makeatletter
\newcommand\tab[1][1cm]{\hspace*{#1}}
\def\@seccntformat#1{%
  \expandafter\ifx\csname c@#1\endcsname\c@section\else
  \csname the#1\endcsname\quad
  \fi}
\makeatother

\title{}
\date{27.10.2020}

\begin{document}
Błąd kolejnego przybliżenia:
\begin{multline*}
\varepsilon_{n+1} = \left| x_{n+1} - \frac{1}{c} \right| = \left |x_n(2-cx_n) - \frac{1}{c} \right| = \left |2x_n-cx_n^2 - \frac{1}{c} \right| = \\ =  \left |\frac{2x_n}{c}-x_n^2 - \frac{1}{c^2} \right| \cdot c = \left | x_n - \frac{1}{c} \right | \cdot \left | x_n - \frac{1}{c} \right | \cdot c = \varepsilon_{n}^2\cdot c
\end{multline*}
Z powyższego wynika:
$$
\lim_{n \to \infty} \varepsilon_n  =
\left\{
    \begin{array}{ll}
        0 & \mbox{if } \varepsilon_0 < \frac{1}{c} \\
        \frac{1}{c} & \mbox{if } \varepsilon_0 = \frac{1}{c} \\
        \infty & \mbox{if } \varepsilon_0 > \frac{1}{c} \\
    \end{array}
\right.
$$
Ponadto zbieżność w pierwszym przypadku, jest kwadratowa.\\
Z powyższych rachunków wynika, że metoda ta jest zbieżna tylko gdy:
$$\varepsilon_0 = \left| x_{0} - \frac{1}{c} \right| < \frac{1}{c} \implies x_0 \in \left(0,\frac{2}{c}\right)$$
$$ \begin{align} p(y=y^{(i)}|x^{(i)};\Theta) &= \cases{p(y=1|x;\Theta) &if $y^{(i)}=1$ \\ 1-p(y=1|x^{(i)};\Theta) &if $y^{(i)}=0$} \\ &= p(y=1|x^{(i)};\Theta)^{y^{(i)}}(1-p(y=1|x^{(i)};\Theta))^{(1-y^{(i)})} \\ &= \sigma(x^{(i)}\Theta)^{y^{(i)}}(1-\sigma(x^{(i)}\Theta))^{(1-y^{(i)})} \end{align} $$
\end{document}