\documentclass{article}
\usepackage[utf8]{inputenc}
\usepackage{polski}


\usepackage{amssymb, amsmath, amsfonts, amsthm, cite, mathtools, enumerate, rotating, hyperref}
\newcommand \eq[1]{\begin{equation} \begin{split}  #1 \end{split} \end{equation}}


\makeatletter
\newcommand\tab[1][1cm]{\hspace*{#1}}
\def\@seccntformat#1{%
  \expandafter\ifx\csname c@#1\endcsname\c@section\else
  \csname the#1\endcsname\quad
  \fi}
\makeatother

\title{}
\date{6.10.2020}
\author{Maurycy Borkowski}
\begin{document}
\maketitle

\section{L1Z2}
Weźmy dowolny $x \in U = \{ x \in \mathbb{R}^n : x_i > 0, i = 1,\dots,n\} $. Ustalmy $r = \min_{x_1\dots x_n}$. (Oczywiście $r > 0$ z definicji $U$). \\\\Weźmy dowolny $y \in B_r(x)$ okazuje się,  $y \in U$. \\\\Gdyby tak nie było to: $y_j \leq 0$ dla pewnego $j$ \\ $r = x_{min} \leq x_k \leq x_k - y_j < ||x_k - y_j|| \leq ||x - y|| $
\\Otrzymaliśmy sprzeczność def. $B_r(x)$.

$\mathbb{R}^n \backslash U$ nie jest owtarty, ponieważ dla np. $(0,\dots,0)$ nie znajdziemy $r$ (zawsze będzie np. $(\frac{\sqrt r}{\sqrt {2n}}, \dots, \frac{\sqrt r}{\sqrt {2n}})$). Więc $U$ nie jest domknięty.

\end{document}

