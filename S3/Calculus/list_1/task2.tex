\documentclass{article}
\usepackage[utf8]{inputenc}
\usepackage{polski}


\usepackage{amssymb, amsmath, amsfonts, amsthm, cite, mathtools, enumerate, rotating, hyperref}
\newcommand \eq[1]{\begin{equation} \begin{split}  #1 \end{split} \end{equation}}


\makeatletter
\newcommand\tab[1][1cm]{\hspace*{#1}}
\def\@seccntformat#1{%
  \expandafter\ifx\csname c@#1\endcsname\c@section\else
  \csname the#1\endcsname\quad
  \fi}
\makeatother

\title{zad. 24* i zad. 28}
\date{12.10.2020}
\author{Maurycy Borkowski}
\begin{document}
\maketitle

\section{L1Z24*}
\subsection*{Koło}
Powierzchnia wpisanego prostokątu to:
$$S(x) = 2x \cdot 2y = 2x \cdot (2\sqrt{r^2-x^2}) = 4x\sqrt{r^2-x^2}$$
Liczymy pochodną:
$$\frac{dS}{dx} = 4\sqrt{r^2-x^2} + \frac{4x(-2x)}{2\sqrt{r^2-x^2}}$$
Przekształcając mamy:
$$\frac{dS}{dx} = \frac{4r^2-8x^2}{\sqrt{r^2-x^2}}$$
Szukamy maksimum:
$$
\frac{4r^2-8x^2}{\sqrt{r^2-x^2}} = 0
$$
$$
4r^2-8x^2 = 0
$$
$$
x^2 = \frac{r^2}{2}
$$
Z warunków zadania:
$$
x = \frac{r}{\sqrt{2}}
$$
Dalej liczymy $y$:
$$
y = \sqrt{r^2 - x^2} = \sqrt{r^2 - \frac{r^2}{2}} = \frac{r}{\sqrt{2}}
$$\\\\\\
\subsection*{Kula}
Możemy wyznaczyć jedną współrzędną z dwóch pozostałych (i $r$):
$$
z = \sqrt{4r^2 - x^2 - y^2}
$$
Funkcja,w której szukamy maximum:
$$
V(x,y) = xyz = xy\sqrt{4r^2 - x^2 - y^2}
$$

Chcemy znaleźć punkty, gdzie $\nabla V(x,y) = (0,0)$:
$$
\frac{\partial V}{\partial x} = y\sqrt{4r^2 - x^2 - y^2} - \frac{x^2y}{\sqrt{4r^2 - x^2 - y^2}} = \frac{4r^2y - 2x^2y - y^3}{\sqrt{4r^2 - x^2 - y^2}}
$$
$$
\frac{\partial V}{\partial y} = x\sqrt{4r^2 - x^2 - y^2} - \frac{xy^2}{\sqrt{4r^2 - x^2 - y^2}} = \frac{4r^2x - 2xy^2 - x^3}{\sqrt{4r^2 - x^2 - y^2}}
$$
Przyrównując do $0$ (wyciągamy z mianownika zmienną):
$$
4r^2 - 2x^2 - y^2 = 0
$$
$$
4r^2 - 2y^2 - x^2 = 0
$$
Podstawiając pierwsze do drugiego:
$$
4r^2 - 2(4r^2 -2x^2) - x^2 = 0
$$
$$
- 4r^2 + 3x^2 = 0
$$
$$
x = \frac{2r}{\sqrt{3}}
$$
Wyznaczamy pozostałe współrzędne:
$$
y = 4r^2 - 2(\frac{2r}{\sqrt{3}})^2 =\frac{2r}{\sqrt{3}}
$$
$$
z = \sqrt{4r^2 - 2(\frac{2r}{\sqrt{3}})^2} = \frac{2r}{\sqrt{3}}
$$\\\\\\\\\\\\\\\\\\\\\\
\section{L1Z28}
Oznaczmy:
$g_1(x,y,z) = x^2 + y^2 + z^2, g_2(x,y,z) = x+y+z$, warunki zadania: $g_1(X) = 1, g_2(X) = 0$
Korzystając z twierdzenia (mnożniki Lagrange'a):
$$
\nabla f = \lambda_1 \nabla g_1 + \lambda_2 \nabla g_2
$$
Daje nam to równania:
$$
\begin{cases}
yz = 2x\lambda_1 + \lambda_2 \\
xz = 2y\lambda_1 + \lambda_2\\
xy = 2z\lambda_1 + \lambda_2 \\
\end{cases}
$$
Zauważmy, że $\lambda_1 \neq 0$ inaczej nie byłyby spełnione warunki.\\
Podstawiając 1. do 2. i 3.:
$$
\begin{cases}
\frac{yz - \lambda_2}{2\lambda_1}z = 2y\lambda_1 + \lambda_2\\
\frac{yz - \lambda_2}{2\lambda_1}y = 2z\lambda_1 + \lambda_2 \\
\end{cases}
$$
$$
\begin{cases}
yz^2 -\lambda_2z = 4y\lambda_1^2 + 2\lambda_1\lambda_2\\
y^2z -\lambda_2y = 4z\lambda_1^2 + 2\lambda_1\lambda_2\\
\end{cases}
$$
$$
\begin{cases}
yz^2 -4y\lambda_1^2 = \lambda_2z + 2\lambda_1\lambda_2\\
y^2z -4z\lambda_1^2 = \lambda_2y + 2\lambda_1\lambda_2\\
\end{cases}
$$
$$
\begin{cases}
y(z^2 -4\lambda_1^2) = \lambda_2(z + 2\lambda_1)\\
z(y^2 -4\lambda_1^2) = \lambda_2(y + 2\lambda_1)\\
\end{cases}
$$
$$
\begin{cases}
(z + 2\lambda_1) \left(y(z-2\lambda_1) - \lambda_2 \right) = 0 \\
(y + 2\lambda_1) \left(z(y-2\lambda_1) - \lambda_2 \right) = 0 \\
\end{cases}
$$
$$
\begin{cases}
(z + 2\lambda_1) \left(yz - 2y\lambda_1 - \lambda_2 \right) = 0 \\
(y + 2\lambda_1) \left(yz - 2z\lambda_1 - \lambda_2 \right) = 0 \\
\end{cases}
$$
Uwzględniając powyższy układ równań i warunki z zadania mamy:\\
dla max:
$y = z = -\frac{1}{\sqrt{6}}$
$x = \frac{2}{\sqrt{6}}$
dla min:
$y = z = \frac{1}{\sqrt{6}}$
$x = -\frac{2}{\sqrt{6}}$

\end{document}

