\documentclass{article}
\usepackage[utf8]{inputenc}
\usepackage{polski}


\usepackage{amssymb, amsmath, amsfonts, amsthm, cite, mathtools, enumerate, rotating, hyperref}
\newcommand \eq[1]{\begin{equation} \begin{split}  #1 \end{split} \end{equation}}


\makeatletter
\newcommand\tab[1][1cm]{\hspace*{#1}}
\def\@seccntformat#1{%
  \expandafter\ifx\csname c@#1\endcsname\c@section\else
  \csname the#1\endcsname\quad
  \fi}
\makeatother

\title{}
\date{22.11.2020}
\author{Maurycy Borkowski}
\begin{document}
\maketitle

\section{L5Z27a}
$z = x^2 + y^2$\\
Dla każdego $z$ pole zamknięte w naszej figurze na płaszczyźnie jest równe:
$$
z = x^2 + y^2 = r^2\\
$$
$$
S = \pi r^2 = \pi z
$$
\\Z zasady Cavalieriego:
$$
V = \int Sdz = \int \pi z dz = \frac{\pi z^2}{2} + C
$$

\end{document}

