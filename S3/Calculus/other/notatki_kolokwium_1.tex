\documentclass{article}
\usepackage[utf8]{inputenc}
\usepackage{polski}


\usepackage{amssymb, amsmath, amsfonts, amsthm, cite, mathtools, enumerate, rotating, hyperref}
\newcommand \eq[1]{\begin{equation} \begin{split}  #1 \end{split} \end{equation}}


\makeatletter
\newcommand\tab[1][1cm]{\hspace*{#1}}
\def\@seccntformat#1{%
  \expandafter\ifx\csname c@#1\endcsname\c@section\else
  \csname the#1\endcsname\quad
  \fi}
\makeatother

\title{}
\date{27.10.2020}

\begin{document}
\subsection*{Metoda mnożników Lagrange'a}
Twierdzenie o mnożnikach Lagrange'a. U jest otwartym podzbiorem $\mathbb{R}^n$,
g jest funkcją, a
$$
S = \{x \in U : g(x) = c\}
$$
Jeśli funkcja $f \mid_S$ przyjmuje minimum lokalne w punkcie $x_0$ oraz $\nabla g(x_0) \neq 0$ to:
$\nabla f(x_0) = \lambda \nabla g(x_0)$ dla pewnej $\lambda$\\\\\
W sytuacji, gdy mamy zbiór zwarty o niepustym wnętrzu i brzegu za-
danym jako poziomica (1.19) , procedura znajdowania wartości największej
i najmniejszej funkcji jest następująca:\\\\
1. Znale1ć punkty krytyczne funkcji wewnątrz zbioru, tzn. punkty stacjonarne
oraz punkty, w których nie można obliczyć pochodnych cząstkowych.\\
2. Znale1ć punkty krytyczne funkcji obciętej do brzegu zbioru, np. metodą
mnożników Lagrange'a.\\
3. Obliczyć wartości funkcji w znalezionych punktach.\\
4. Wybrać wartość największą i najmniejszą.\\\\\
Dla wiecej zmiennych:
Załóżmy, że wektory $\nabla g 1 (x 0 ), \nabla g 2 (x 0 ), . . . , \nabla g k (x 0 )$ są liniowo
niezależne. Jeśli funkcja f S posiada ekstremum w punkcie $x_0  \in  S$, to
$$\nabla f (x 0 ) =  \lambda_1 \nabla g_1 (x_0 ) +  \lambda_2 \nabla g_2 (x_0 ) + . . . +  \lambda_k \nabla g_k (x_0 )$$
dla pewnych stałych
$\lambda_1 ,  \lambda_2 , . . . ,  \lambda_k .$
\\\\Tw. o funkcji uwikłanej\\
Uwaga. Aby znale1ć punkt $x_0$ trzeba rozwiązać n + k równań przy n + k niewiadomych: n współrzędnych i k lambd.\\\\
Załóżmy, że funkcja $F: \mathbb{R}^n \rightarrow \mathbb{R}$ jest klasy $C^1$.\\\\ Oraz $F(x_0,z_0) = 0$ oraz $\frac{\partial F}{\partial z} (x_0,z_0) \neq 0$ Wtedy równanie $F(x,z) = 0$ ma jednoznacznie rozwiązanie w pobliżu $(x_0,z_0)$
\\\\Tw. o funkcji odwrotnej\\
Niech $U \subset $ będzie otwartym podzbiorem przestrzeni $\mathbb{R}^n$. Funkcje $f_1,f_2,\dots f_n$ są klasy $C^1$ na U. Załóżmy, że układ * ma rozwiązanie $x = a, y = b$ dla $a \in U$. Jeśli:
$$
\Delta = det \left[\frac{\partial f_i}{\partial x_j}(a) \right] \neq 0
$$
To układ ma jednoznacznie rozwiązanie dla y w pobliżu b i x w pobliżu a.
\end{document}