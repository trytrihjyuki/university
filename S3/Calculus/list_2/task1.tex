\documentclass{article}
\usepackage[utf8]{inputenc}
\usepackage{polski}


\usepackage{amssymb, amsmath, amsfonts, amsthm, cite, mathtools, enumerate, rotating, hyperref}
\newcommand \eq[1]{\begin{equation} \begin{split}  #1 \end{split} \end{equation}}


\makeatletter
\newcommand\tab[1][1cm]{\hspace*{#1}}
\def\@seccntformat#1{%
  \expandafter\ifx\csname c@#1\endcsname\c@section\else
  \csname the#1\endcsname\quad
  \fi}
\makeatother

\title{}
\date{18.10.2020}
\author{Maurycy Borkowski}
\begin{document}
\maketitle

\section{L2Z6}
Mamy równanie:
$$
x^2+y^2+z^2 - 2x + 2y -4z - 10 = 0
$$
$$
(x - 1)^2 + (y + 1)^2 + (z - 2)^2 = 4^2
$$
Powierzchnia jest ograniczona, jest to sfera o promieniu $r=4$.\\
Liczymy, gdzie możemy rozwikłać funkcje:
$$
\frac{\partial F}{\partial z} = 2z - 4 \neq 0 \iff z \neq 2
$$
$$
\frac{\partial F}{\partial x} = 2x- 2
$$
$$
\frac{\partial F}{\partial y} = 2y + 2
$$
$\frac{\partial F}{\partial z} \neq 0$, korzystamy więc z twierdzenia o funkcji uwikłanej i liczymy w otoczeniu rozwiązania:
$$
\frac{\partial z}{\partial x} = \frac{2x-2}{2z-4}
$$
$$
\frac{\partial z}{\partial y} = \frac{2y+2}{2z-4}
$$
Wobec powyższego ekstrema lokalne będą gdy: $\frac{\partial z}{\partial y} = \frac{\partial z}{\partial x}  = 0 \implies x = 1, y = -1 $\\
Liczymy $z$:
$$
z^2-4z-12 = 0
$$
$$
(z-6)(z+2) = 0
$$
Mamy więc:\\
min $z(x,y) = -2$\\
max $z(x,y) = 6$\\
% \section{L2Z19*}
% $x= \rho \sin{\phi}\cos{\theta}$\\
% $y= \rho \sin{\phi}\sin{\theta}$\\
% $z= \rho \cos{\phi}$
% $$
% \frac{\partial(x,y,z)}{\partial(\rho,\phi,\theta)} = {\rho}^2\sin{\phi}
% $$
\end{document}

