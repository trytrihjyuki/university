\documentclass{article}
\usepackage[utf8]{inputenc}
\usepackage{polski}


\usepackage{amssymb, amsmath, amsfonts, amsthm, cite, mathtools, enumerate, rotating, hyperref}
\newcommand \eq[1]{\begin{equation} \begin{split}  #1 \end{split} \end{equation}}


\makeatletter
\newcommand\tab[1][1cm]{\hspace*{#1}}
\def\@seccntformat#1{%
  \expandafter\ifx\csname c@#1\endcsname\c@section\else
  \csname the#1\endcsname\quad
  \fi}
\makeatother

\title{}
\date{20.10.2020}
\author{Maurycy Borkowski}
\begin{document}
\maketitle

\section{L2Z15}
$$
f(x,y) = ((x^2 - y^2)/(x^2+y^2), xy/(x^2+y^2))
$$
Obliczamy jakobian:
$$
\Delta = 
\begin{vmatrix}
\frac{4xy^2}{(x^2+y^2)^2} & \frac{4x^2y}{(x^2+y^2)^2}\\
\frac{y(y^2-x^2)}{(x^2+y^2)^2} & \frac{x(x^2-y^2)}{(x^2+y^2)^2} \\
\end{vmatrix}
= \frac{8x^2y^2(x^2-y^2)}{(x^2+y^2)^2}
$$
$$
\Delta = \frac{8x^2y^2(x^2-y^2)}{(x^2+y^2)^2} = \frac{0}{1} = 0
$$
f nie jest lokalnie odwracalne w pobliżu $(x,y) = (0,1)$ z twierdzenia o funkcji odwrotnej (jakobian niezerowy)
\section{L2Z18*}
$x(r,\theta) = r\cos{\theta} \tab y(r,\theta) = r\sin{\theta}$\\
Liczymy jakobian:
$$
\Delta = 
\begin{vmatrix}
\cos{\varphi_0} & -r_0\sin{\varphi_0}\\
\sin{\varphi_0} & r_0\cos{\varphi_0}\\
\end{vmatrix} =
r(\cos{\varphi_0}^2 + \sin{\varphi_0}^2) = r_0
$$
Ostatnia równość z jedynki trygonometrycznej.\\\\
Z twierdzenia o funkcji odwrotnej: $\Delta = r_0 \neq 0$ aby można było utworzyć funkcje odwrotne: $\varphi(x,y), r(x,y)$.\\\\
Bezpośrednio: $r \neq 0$ ponieważ wtedy $\varphi$ byłaby wyznaczana niejednoznacznie ($r = 0 \implies x = y = 0$ a $\varphi$ może być dowolne)

\end{document}

