\documentclass{article}
\usepackage[utf8x]{inputenc}
\usepackage{polski}
\usepackage{pythonhighlight}
\usepackage[thinlines]{easytable}

\usepackage{amssymb, amsmath, amsfonts, amsthm, cite, mathtools, enumerate, rotating, hyperref,soul, graphicx}
\newcommand \eq[1]{\begin{equation} \begin{split}  #1 \end{split} \end{equation}}

\makeatletter
\newcommand\tab[1][1cm]{\hspace*{#1}}

\renewcommand{\arraystretch}{1.2}

\def\@seccntformat#1{%
  \expandafter\ifx\csname c@#1\endcsname\c@section\else
  \csname the#1\endcsname\quad
  \fi}
\makeatother
\newtheorem{lemma}{Lemat}
\newtheorem{theorem}{Twierdzenie}
\begin{document}
\section{zadanie 5}
Zauważmy, że gra kończy się w co najwyżej $a+b-1$ rundach.\\\\
Bez straty ogólności możemy zawsze grać $a+b-1$ rund i potem wyłaniać zwycięzcę niezależnie od tego, że moglibyśmy zakończyć grę czasami wcześniej. Nie zmienia to wyniku gry a w sumie poniżej uwzględniamy i zliczamy każde możliwe zakończenie dla już przesądzonej gry.\\\\
Sprawiedliwie podzielić stawkę można według wartości oczekiwanej od wygranej dla każdego gracza.\\
Wartość oczekiwana wygranej dla gracza A:
$$
stawka\cdot\sum_{j=a}^{a+b-1} \binom{a+b-1}{j}p^j(1-p)^{a+b-1-j}
$$
dla gracza B:
$$
stawka\cdot\sum_{j=b}^{a+b-1} \binom{a+b-1}{j}(1-p)^jp^{a+b-1-j}
$$
\newpage
\section{zadanie 9}
Załóżmy, że mamy 3 kule w worku (dwie białe, jedną czarną), wyjmujemy najpierw jedną i potem drugą \textbf{(bez zwracania)}.
\begin{center}
  \begin{tabular}{ | l | c | r | r |}
    \hline
     & P(A=czarna) & P(A=biała) & P(A)\\ \hline 
    P(B=czarna) & 0 & $\frac{2}{3}\cdot\frac{1}{2} = \frac{1}{3}$ & $\frac{1}{3}$ \\ \hline
    P(B=biała) & $\frac{1}{3}\cdot\frac{2}{2} = \frac{1}{3}$ & $\frac{2}{3}\cdot\frac{1}{2} = \frac{1}{3}$ & $\frac{2}{3}$ \\ \hline
    P(B) & $\frac{1}{3}$ & $\frac{2}{3}$ & \\
    \hline
  \end{tabular}
\end{center}

Zauważmy, że taki sam rozkład brzegowy tj.:
\begin{align*}
P(A=czarna)=P(B=czarna) = \frac13\\
P(A=biala)=P(B=biala) = \frac23
\end{align*}
Uzyskamy z rozkładu łącznego (A,B) gdzie A,B określa prawdopodbieństwo wylosowania określonego koloru kuli w tym samym worku, ale \textbf{ze zwracaniem}.
\begin{center}
  \begin{tabular}{ | l | c | r | r |}
    \hline
     & P(A=czarna) & P(A=biała) & P(A)\\ \hline 
    P(B=czarna) & $\frac{1}{3}\cdot\frac{1}{3} = \frac{1}{9}$ & $\frac{2}{3}\cdot\frac{1}{3} = \frac{2}{9}$ & $\frac39 =\frac{1}{3}$ \\ \hline
    P(B=biała) & $\frac{1}{3}\cdot\frac{2}{3} = \frac{2}{9}$ & $\frac{2}{3}\cdot\frac{2}{3} = \frac{4}{9}$ & $\frac69 = \frac{2}{3}$ \\ \hline
    P(B) & $\frac39 =\frac{1}{3}$ & $\frac69 = \frac{2}{3}$ & \\
    \hline
  \end{tabular}
\end{center}
\end{document}