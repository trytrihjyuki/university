\documentclass{article}
\usepackage[utf8x]{inputenc}
\usepackage{polski}
\usepackage{pythonhighlight}
\usepackage[thinlines]{easytable}

\usepackage{amssymb, amsmath, amsfonts, amsthm, cite, mathtools, enumerate, rotating, hyperref,soul, graphicx}
\newcommand \eq[1]{\begin{equation} \begin{split}  #1 \end{split} \end{equation}}

\makeatletter
\newcommand\tab[1][1cm]{\hspace*{#1}}

\renewcommand{\arraystretch}{1.2}

\def\@seccntformat#1{%
  \expandafter\ifx\csname c@#1\endcsname\c@section\else
  \csname the#1\endcsname\quad
  \fi}
\makeatother
\newtheorem{lemma}{Lemat}
\newtheorem{theorem}{Twierdzenie}
\begin{document}
\section{zadanie 15}
Oznaczmy $Y = F_X(X)$ (oczywiście $0 \leq Y \leq 1$). Wtedy:
\begin{align*}
F_Y(x) = P(Y\leq x)= P(F_X(X) \leq x) =P(F_X(X) \leq F_X(F_X^{-1}(x)))
\end{align*}
$F_X$ jest ciągła i ściśle rosnąca więc $a \leq b \iff F_X(a) \leq F_X(b)$
$$
F_Y(x) = P(X \leq F_X^{-1}(x)) = F_X(F_X^{-1}(x)) = x
$$
Z wykładu wiemy, że dystrybuanta z $U[0,1]$ to $F_U(x) = x$.\\
Stąd na $x \in [0,1]$ mamy $F_U(x) = F_Y(x)$. \\Dystrybuanta wyznacza rozkład jednoznacznie stąd $F_X(X) = Y \sim U[0,1]$.\\\\\\
Teraz chcemy zrobić \textit{odwrotnie} tj. z $U[0,1]$ dostać np. $exp$\\
Generujemy $U \sim Unif[0,1].$ Dystrybuanta rozkładu wykładniczego: $$F_X(x) = 1-e^{-\lambda x}$$\\
Liczymy funkcję odwrotną dystrybuanty: $$F_X^{-1}(y) = -\frac1\lambda \ln(1-y)$$\\
Dalej, weźmy zmienną losową $T = F_X^{-1}(U)$ Jej dystrybuanta:
$$
F_T(x) = P(T \leq x) = P(F_X^{-1}(U) \leq x) = P(U \leq F_X(x)) = F_U(F_X(x)) = F_X(x)
$$

\newpage
\section{zadanie 17}
Promień $R \sim U[0,2]$, % więc dla $x\in[0,2]$ mamy funkcję gęstości $f_R(x) = \frac 1 2$\\
zmienna losowa będąca polem koła $Y = A(R) =  \pi R^2$\\\\
Mamy od razu: $A^{-1}(x) = \sqrt{\frac x \pi}$ \\Stąd dystrybuanta pola:
$$
F_Y(x) = P(Y \leq x) = P(A(Y) \leq x)
$$
dla $x \geq 0$ mamy ($A$ ściśle rosnąca, ciągła):
$$
F_Y(x) = P(A(R) \leq x) = P(R \leq A^{-1}(x)) = F_R(A^{-1}(x))
$$
Dalej:
$$
f_Y(x) = \frac{d}{dx}F_Y(x) = \frac{d}{dx} \left ( F_R(A^{-1}(x))\right) = F_R^\prime(A^{-1}(x)) \cdot \frac{d}{dx}A^{-1}(x)
$$
Z jednostajnego rozkładu $R$ mamy dla $x \in [0, 4\pi]$:
$$
F_R^\prime(A^{-1}(x)) = f_R(A^{-1}(x)) = \frac 1 2
$$
Obliczamy pochodną ($x\geq0$):
$$
\frac{d}{dx}A^{-1}(x) = \frac{d}{dx}\sqrt{\frac x \pi} = \frac{1}{2\sqrt{x\pi}}
$$
Ostatecznie mamy więc:
$$
f_Y(x) = \frac{1}{4\sqrt{x\pi}}
$$
\end{document}