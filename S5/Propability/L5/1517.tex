\documentclass{article}
\usepackage[utf8x]{inputenc}
\usepackage{polski}
\usepackage{pythonhighlight}

\usepackage{amssymb, amsmath, amsfonts, amsthm, cite, mathtools, enumerate, rotating, hyperref,soul}
\newcommand \eq[1]{\begin{equation} \begin{split}  #1 \end{split} \end{equation}}


\makeatletter
\newcommand\tab[1][1cm]{\hspace*{#1}}
\def\@seccntformat#1{%
  \expandafter\ifx\csname c@#1\endcsname\c@section\else
  \csname the#1\endcsname\quad
  \fi}
\makeatother

\newtheorem{lemma}{Lemat}
\newtheorem{theorem}{Twierdzenie}

\title{AiSD L5}
\date{12.05.2021}
\author{Maurycy Borkowski}
\begin{document}
% \maketitle
\section{zadanie 15.}
$X$ - zmienna losowa opisująca liczbę głosów \textit{za}. \\Oczywiście $X \sim B(n, p)$ gdzie $p$ to rzeczywiste poparcie.\\\\
Oznaczmy przez $m$ oczekiwane poparcie. Z rozkładu dwumianowego:
\begin{align*}
m = \mathbb EX = np\\
\text{oraz}\\
\operatorname{Var}[X] = np(1-p)
\end{align*}
Interesuje nas oszacowanie dla warunku prawdopobieństwa popełnienia błędu względnego większego niż $1\%$ było mniejsze niż $5\%$:
\begin{align}
P\left( \frac{|X-m|}{m} \geq 0.01 \right) \leq 00.5
\iff
P\left( |X-m| \geq 0.01m \right) \leq 00.5
\end{align}
Z nierówności Czebyszewa mamy:
\begin{align*}
P\left( |X-m| \geq 0.01m \right) \leq \frac{np(1-p)}{\left(\frac{np}{100} \right)^2} = 10^4 \cdot \frac{(1-p)}{np}
\end{align*}
stąd aby warunek (1) był zachowany wystraczy by:
\begin{align*}
10^4 \cdot \frac{(1-p)}{np} \leq 0.05
\end{align*}
co daje dolne ograniczenie na $n$:
\begin{align*}
n \geq 2\cdot10^5\cdot\frac{1-p}p
\end{align*}
\clearpage
\setcounter{equation}{0}
\section{zadanie 17.}
Oznaczmy przez $X$ zmienną losową opisującą zysk gracza (= strata kasyna) w $10^6$ rozgrywkach, a $X_i$ w tylko $i$-tej rozgrywce.\\\\
Interesuje nas:
\begin{align}
P(X \geq 10^4)
\end{align}
Skorzystajmy z nierówności Chernoff'a:
\begin{align}
P(X \geq 10^4) \leq \min_{t\geq 0} e^{-t10^4} E(e^{tX})
\end{align}
rozpiszmy wyrażenie $E(e^{tX})$:
\begin{align}
E(e^{tX}) = E\left(e^{t\sum_{i=0}^{10^6}X_i}\right) = \prod_{i=0}^{10^6}E(e^{tX_i}) = \prod_{i=0}^{10^6} \frac{32}{200}e^{2t}+\frac{1}{200}e^{99t} + \frac{167}{200}e^{-t}
\end{align}
wobec powyższego
\begin{align}
\min_{t\geq 0} e^{-t10^4} E(e^{tX}) = \min_{t\geq 0} \left(e^{-t} \left(\frac{32}{200}e^{2t}+\frac{1}{200}e^{99t} + \frac{167}{200}e^{-t}\right)^{100}\right)^{10^4}
\end{align}
wyrażenie do minimalizacji jest nieujemne więc argmin będzie ten sam jak zdejmiemy potęgę:
\begin{align}
\min_{t\geq 0} e^{-t} \left(\frac{32}{200}e^{2t}+\frac{1}{200}e^{99t} + \frac{167}{200}e^{-t}\right)^{100}
\end{align}
Za pomocą programu znajdujemy przybliżenie minimum $t_{min} \approx 0.000575$.\\
Podstawiając argmin do wzoru w nierówności otrzymujemy przybliżenie:
$$
P(X \geq 10^4) \leq \left(e^{-t} \left(\frac{32}{200}e^{2t}+\frac{1}{200}e^{99t} + \frac{167}{200}e^{-t}\right)^{100}\right)^{10^4} \approx (0.9991254742698236)^{10^4} \approx 0.0001586
$$

\clearpage


\end{document}