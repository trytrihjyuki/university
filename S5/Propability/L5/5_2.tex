\documentclass{article}
\usepackage[utf8x]{inputenc}
\usepackage{polski}
\usepackage{pythonhighlight}

\usepackage{amssymb, amsmath, amsfonts, amsthm, cite, mathtools, enumerate, rotating, hyperref,soul}
\newcommand \eq[1]{\begin{equation} \begin{split}  #1 \end{split} \end{equation}}


\makeatletter
\newcommand\tab[1][1cm]{\hspace*{#1}}
\def\@seccntformat#1{%
  \expandafter\ifx\csname c@#1\endcsname\c@section\else
  \csname the#1\endcsname\quad
  \fi}
\makeatother

\newtheorem{lemma}{Lemat}
\newtheorem{theorem}{Twierdzenie}

\title{AiSD L5}
\date{12.05.2021}
\author{Maurycy Borkowski}
\begin{document}
% \maketitle
\section*{5.2}
Oznaczmy:\\$X_j$ - zmienna losowa przyjmująca $1$ jeżeli kupon typu $j$ jest w wybranym zbiorze oraz $0$ jeżeli tego kuponu nie ma w zbiorze.\\\\
$X = \sum_{i=1}^n X_i$, $X$ liczba różnych kuponów w danym zbiorze. Liczymy:
$$
P(X_j=1) = 1-(1-p_j)^k
$$
$X_j$ przyjmuje wartości $\{0,1\}$ stąd, $E(X_j) = P(X_j=1)$ oraz $X_j^2 = X_j$\\\\
Liczymy wartość oczekiwaną $X$:
\begin{align*}
E(X) = E\left(\sum_{i=1}^nX_i\right)=\sum_{i=1}^n\left(1-(1-p_i)^k\right) = n - \sum_{i=1}^n(1-p_i)^k
\end{align*}
Policzmy wariancję $X$:
\begin{align*}
Var(X) = E(X^2) - E(X)^2\\
\end{align*}
\begin{align*}
% \text{Licz}\\
E(X^2) = E\underbrace{\left(\sum_{i=1}^n X_i^2\right)}_{\sum_{i=1}^n X_i=X} + \sum_{\substack{(i,j) \in \{1\ldots n\}\times\{1\ldots n\} \\ i \neq j}} E(X_iX_j) = E(X) - \sum_{\substack{(i,j) \in \{1\ldots n\}\times\{1\ldots n\} \\ i \neq j}} 1 - (1-p_i - p_j)^k
\end{align*}
Podstawiając wyniki otrzymujemy:
$$
Var(X) = \sum_{\substack{(i,j) \in \{1\ldots n\}\times\{1\ldots n\} \\ i \neq j}} (1-p_i - p_j)^k + E(X) - (E(X))^2
- n(n-1)
$$



\clearpage


\end{document}