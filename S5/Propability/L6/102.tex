\documentclass{article}
\usepackage[utf8x]{inputenc}
\usepackage{polski}
\usepackage{pythonhighlight}

\usepackage{amssymb, amsmath, amsfonts, amsthm, cite, mathtools, enumerate, rotating, hyperref,soul}
\newcommand \eq[1]{\begin{equation} \begin{split}  #1 \end{split} \end{equation}}


\makeatletter
\newcommand\tab[1][1cm]{\hspace*{#1}}
\def\@seccntformat#1{%
  \expandafter\ifx\csname c@#1\endcsname\c@section\else
  \csname the#1\endcsname\quad
  \fi}
\makeatother

\newtheorem{lemma}{Lemat}
\newtheorem{theorem}{Twierdzenie}

\title{AiSD L5}
\date{12.05.2021}
\author{Maurycy Borkowski}
\begin{document}
% \maketitle
\section{zadanie 10.}
$X$ - zmienna losowa opisująca liczbę osób \textit{popierających}. \\Oczywiście $X \sim B(n, p)$ gdzie $p$ to rzeczywista frakcja.\\\\
Z rozkładu dwumianowego:
\begin{align*}
\mathbb EX = np\\
\text{oraz}\\
\operatorname{Var}[X] = np(1-p)
\end{align*}
Interesuje nas oszacowanie dla warunku prawdopobieństwa popełnienia błędu bezwzględnego oszacowania frakcji poparcia większego niż $0.01$ było mniejsze niż $0.05$:
\begin{align}
P\left( \left|\frac X n-p\right| \geq 0.01 \right) \leq 00.5
\iff
P\left( |X-np| \geq 0.01n \right) \leq 00.5
\end{align}
Z nierówności Czebyszewa mamy:
\begin{align*}
P\left( |X-np| \geq 0.01n \right) \leq \frac{np(1-p)}{\left(\frac{n}{100} \right)^2} = 10^4 \cdot \frac{p(1-p)}{n}
\end{align*}
stąd aby warunek (1) był zachowany wystraczy by:
\begin{align*}
10^4 \cdot \frac{p(1-p)}{n} \leq 0.05
\end{align*}
co daje dolne ograniczenie na $n$:
\begin{align*}
n \geq 2\cdot10^5\cdot p(1-p)
\end{align*}
\subsection*{a)}
z uwzględnieniem warunku $0 \leq p \leq 0.3$ ostatecznie mamy:
$$
n \geq 2\cdot10^5\cdot 0.21 = 42000
$$
\subsection*{b)}
bez innych ograniczeń niż $0 \leq p \leq 1$ otrzymujemy:
$$
n \geq 2\cdot10^5\cdot 0.25 = 50000
$$
\clearpage
\setcounter{equation}{0}
\section{zadanie 2.}
$O$ - zmienna losowa opisująca liczbę orłów po trzech rzutach
\\Chcemy policzyć warunkową wartość oczekiwaną $E(X|O\leq1)$, rozpisujemy:
$$
E(X|O\leq1) =  P(O=0|O\leq1)E(X|O=0)+ P(O=1|O\leq1)E(X|O=1)
$$
Z $O \sim B(3, 0.5)$ mamy:\\
$P(O=0|O\leq1) = \frac{1}{4}$
$P(O=1|O\leq1) = \frac{3}{4}$\\\\
Z definicji liczymy:
\begin{align*}
E(X|O=0) = \sum_{x=0}^3 xP(X=x|O=0) = 0 + 0 + 0 + 3 = 3\\
E(X|O=1) = \sum_{x=0}^3 xP(X=x|O=1) = 0 + 0 + 2 + 0 = 2\\
\end{align*}
Ostatecznie:
$$
E(X|O\leq1) = \frac{1}{4}\cdot3 + \frac{3}{4}\cdot2 = \frac9 4
$$
\clearpage

\end{document}